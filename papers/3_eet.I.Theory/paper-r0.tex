%%
%%   Version 3.1 of 16 May 2019.
%%

\usepackage{graphicx}% Include figure files
\usepackage{dcolumn}% Align table columns on decimal point
\usepackage{bm}% bold math

% hyphenation
\usepackage{hyphenat}

% tables
\usepackage{multirow}
\usepackage{booktabs}

% mathematics
\usepackage{amsmath}
\usepackage{amsfonts}
\usepackage{amssymb}
\usepackage{amsbsy}
\usepackage{mathrsfs}
\usepackage{upgreek}
\usepackage[cbgreek]{textgreek}

%---------------------------------------------------
% Approximations
\newcommand{\Approx}[2]{\ensuremath{\text{Ap}_{#2} \left[ {#1} \right] }}
% happy integral
\newcommand{\rint}[1]{\mbox{\Large $ \int\limits_{\mbox{\tiny  $#1$}}$}}
% SHORTCUTS
%\newcolumntype{,}{D{.}{,}{2}}
\newcommand{\citee}[1]{\ensuremath{\scriptsize^{\citenum{#1}}}}
\newcommand{\HRule}{\rule{\linewidth}{0.2mm}}
% Quantum notation
\newcommand{\Bra}[1]{\ensuremath{\bigl\langle {#1} \bigl\lvert}}
\newcommand{\Ket}[1]{\ensuremath{\bigr\rvert {#1} \bigr\rangle}}
\newcommand{\BraKet}[2]{\ensuremath{\bigl\langle {#1} \bigl\lvert {#2} \bigr\rangle}}
\newcommand{\tBraKet}[3]{\ensuremath{\bigl\langle {#1} \bigl\lvert {#2} \bigl\lvert {#3} \bigr\rangle}}
%
\newcommand{\bra}[1]{\ensuremath{\bigl( {#1} \bigl\lvert}}
\newcommand{\ket}[1]{\ensuremath{\bigr\rvert {#1} \bigr)}}
\newcommand{\braket}[2]{\ensuremath{\bigl( {#1} \bigl\lvert {#2} \bigr)}}
\newcommand{\tbraket}[3]{\ensuremath{\bigl( {#1} \bigl\lvert {#2} \bigl\lvert {#3} \bigr)}}
% Math
\newcommand{\pd}{\ensuremath{\partial}}
\newcommand{\DR}{\ensuremath{{\rm d} {\bf r}}}
%\newcommand{\BM}[1]{\ensuremath{\mbox{\boldmath${#1}$}}}
\newcommand{\BM}[1]{\bm{#1}}
% Chemistry (formulas)
\newcommand{\ch}[2]{\ensuremath{\mathrm{#1}_{#2}}}
% Math 
\newcommand{\VEC}[1]{\ensuremath{\mathrm{\mathbf{#1}}}}
% vector nabla
\newcommand{\Nabla}{\ensuremath{ \BM{\nabla}}}
% derivative
\newcommand{\FDer}[3]{\ensuremath{
\bigg(
\frac{\partial #1}{\partial #2}
\bigg)_{#3}}}
% diagonal second derivative
\newcommand{\SDer}[3]{\ensuremath{
\biggl(
\frac{\partial^2 #1}{\partial #2^2}
\biggr)_{#3}}}
% off-diagonal second derivative
\newcommand{\SSDer}[4]{\ensuremath{
\biggl(
\frac{\partial^2 #1}{\partial #2 \partial #3}
\biggr)_{#4}}}
% derivatives without bound
% derivative
\newcommand{\fderiv}[2]{\ensuremath{
\frac{\partial #1}{\partial #2}}}
% diagonal second derivative
\newcommand{\sderiv}[2]{\ensuremath{
\frac{\partial^2 #1}{\partial #2^2}
}}
% off-diagonal second derivative
\newcommand{\sderivd}[3]{\ensuremath{
\frac{\partial^2 #1}{\partial #2 \partial #3}
}}
% derivatives for tables
\newcommand{\fderivm}[2]{\ensuremath{
{\partial #1}/{\partial #2}}}
% diagonal second derivative
\newcommand{\sderivm}[2]{\ensuremath{
{\partial^2 #1}/{\partial #2^2}
}}
% off-diagonal second derivative
\newcommand{\sderivdm}[3]{\ensuremath{
{\partial^2 #1}/{\partial #2 \partial #3}
}}
% ERIs and OEIs
\newcommand{\OEIc}[3]{\ensuremath{\left(#1 \lvert #2 \rvert #3 \right)}}
\newcommand{\ERIc}[4]{\ensuremath{\left(#1 #2 \vert #3 #4 \right)}}

% Partial density and potential
\newcommand{\PartPot}[4]{\ensuremath{\frac{#1 #2}{\lvert #3-#4 \rvert }}}

% trace operator
\DeclareMathOperator{\Tr}{Tr}

%\draft % marks overfull lines with a black rule on the right

% Define location of graphics
\graphicspath{{./figures/}}

\begin{document}
\preprint{AIP/123-OEP}

\title{Excitonic Energy Transfer Couplings at Short Distances: Efficient Transfer-Integral Fragmentation Approach
Based on Effective One-Electron Operators}

\author{Bartosz B{\l}asiak}
\email[]{blasiak.bartosz@gmail.com}
\homepage[]{https://www.polonez.pwr.edu.pl}
\author{Robert W. G{\'o}ra}
\author{Marta Cho{\l}uj} 
\author{Joanna D. Bednarska}
\author{Wojciech Bartkowiak}

\affiliation{Department of Physical and Quantum Chemistry, Faculty of Chemistry, 
Wroc{\l}aw University of Science and Technology, 
Wybrze{\.z}e Wyspia{\'n}skiego 27, Wroc{\l}aw 50-370, Poland}

\date{\today}

\begin{abstract}
Bla.
\end{abstract}

\pacs{}

\maketitle

\tableofcontents

\section{\label{s:1}Introduction}

\section{\label{s:2}Theory}

In the simplest version of the TI/CIS approach, the Hamiltonian of the 
complex of molecule $A$ and $B$ is constructed assuming the configuration interaction
singles (CIS) approximation involving the following four configurations
%
\begin{align}
 \Big| \Phi_1 \Big> &= \Big| \Psi_A^{(e)} \otimes \Psi_B^{(g)} \Big> \\
 \Big| \Phi_2 \Big> &= \Big| \Psi_A^{(g)} \otimes \Psi_B^{(e)} \Big> \\
 \Big| \Phi_3 \Big> &= \Big| \Psi_A^{(+)} \otimes \Psi_B^{(-)} \Big> \\
 \Big| \Phi_4 \Big> &= \Big| \Psi_A^{(-)} \otimes \Psi_B^{(+)} \Big> 
\end{align}
%
where $g$ and $e$ superscripts denote the ground and excited state of a molecule,
`+' and `-' label the cationic and anionic state, respectively, 
whereas $ \Big| \Psi_X \otimes \Psi_Y \Big> $ denotes the antisymmetrized Hartree product
of the monomer wavefunctions.
The associated diagonal Hamiltonian matrix elements can be defined as
%
\begin{align}
 \Big< \Phi_1 \Big| \mathscr{H} -E_{0} \Big| \Phi_1 \Big> &\equiv E_1 = E^A_{e\rightarrow g} 
     + \sum_{\mu\nu\in A} \left( P_{\nu\mu}^{A(e)} - P_{\nu\mu}^{A(g)}\right) \times
     \left\{ V_{\mu\nu}^{B({\rm nuc})} + \sum_{\lambda\sigma\in B} P_{\lambda\sigma}^{B(g)} 
     \left[ (\mu\nu | \sigma\lambda) - \frac{1}{2} (\mu\lambda | \sigma\nu) \right] \right\} \\
 \Big< \Phi_2 \Big| \mathscr{H} -E_{0} \Big| \Phi_2 \Big> &\equiv E_2 = E^B_{e\rightarrow g} 
     + \sum_{\mu\nu\in B} \left( P_{\nu\mu}^{B(e)} - P_{\nu\mu}^{B(g)}\right) \times
     \left\{ V_{\mu\nu}^{A({\rm nuc})} + \sum_{\lambda\sigma\in A} P_{\lambda\sigma}^{A(g)} 
     \left[ (\mu\nu | \sigma\lambda) - \frac{1}{2} (\mu\lambda | \sigma\nu) \right] \right\} \\
 \Big< \Phi_3 \Big| \mathscr{H} -E_{0} \Big| \Phi_3 \Big> &\equiv E_3 = 
 -\varepsilon_H^A + \varepsilon_L^B - \big( H^A H^A \big| L^B L^B \big)  \\
 \Big< \Phi_4 \Big| \mathscr{H} -E_{0} \Big| \Phi_4 \Big> &\equiv E_4 = 
  \varepsilon_L^A - \varepsilon_H^B - \big( L^A L^A \big| H^B H^B \big)  
\end{align}
%
The associated off\hyp{}diagonal Hamiltonian matrix elements are
%
\begin{align}
 \Big< \Phi_1 \Big| \mathscr{H} \Big| \Phi_2 \Big> &\equiv V^{\rm Coul} + V^{\rm Exch} + V^{\rm Ovrl}\\
 \Big< \Phi_1 \Big| \mathscr{H} \Big| \Phi_3 \Big> &\equiv V^{\rm ET1} \\
 \Big< \Phi_2 \Big| \mathscr{H} \Big| \Phi_4 \Big> &\equiv V^{\rm ET2} \\
 \Big< \Phi_1 \Big| \mathscr{H} \Big| \Phi_4 \Big> &\equiv V^{\rm HT1} \\
 \Big< \Phi_2 \Big| \mathscr{H} \Big| \Phi_3 \Big> &\equiv V^{\rm HT2} \\
 \Big< \Phi_3 \Big| \mathscr{H} \Big| \Phi_4 \Big> &\equiv V^{\rm CT } 
\end{align}
%
where the Forster-type Coulombic (Coul), Dexter-type exchange (Exch), remaining overlap correction (Ovrl),
as well
as the electron, hole and charge (ET, HT, CT) transfer contributions are defined.

For a closed-shell system, the EET coupling constant for two electronic transitions
can then be given approximately by
%
\begin{equation}
  V \approx V^{\rm Direct} + V^{\rm Inirect}
\end{equation}
%
where the overlap\hyp{}corrected direct and indirect coupling constants are 
%
\begin{align}
  V^{\rm Direct  } &= V^{\rm Coul} + V^{\rm Exch} + V^{\rm Ovrl} \\
  V^{\rm Indirect} &= V^{\rm TI-2} + V^{\rm TI-3}
\end{align}
%
with
%
\begin{align}
 V^{\rm TI-2} &=-\frac{V^{\rm ET1} V^{\rm HT2}}{E_3-E_1} -\frac{V^{\rm ET2} V^{\rm HT1}}{E_4-E_1} \\
 V^{\rm TI-3} &= \frac{V^{\rm CT} \left( V^{\rm ET1} V^{\rm ET2} + V^{\rm HT1} V^{\rm HT2}\right) }{(E_3-E_1)(E_4-E_1)}
\end{align}
%
The exchange-Coulomb coupling takes the form
%
\begin{align}
 V^{\rm Coul} &= \frac{V^{{\rm Coul},(0)}}{1 - S_{12}^2} \\
 V^{\rm Exch} &= \frac{V^{{\rm Exch},(0)}}{1 - S_{12}^2} \\
 V^{\rm Ovrl} &=-\frac{(E_1+E_2)S_{12}}{2(1-S_{12}^2)}
\end{align}
%
The overlap-corrected ET, HT and CT matrix elements read
%
\begin{align}
 V^{\rm ET1} &= \left[ 1 - S_{13}^2 \right]^{-1} \left\{ V^{{\rm ET1},(0)} - \frac{1}{2} (E_1+E_2) S_{13} \right\} \\
 V^{\rm ET2} &= \left[ 1 - S_{24}^2 \right]^{-1} \left\{ V^{{\rm ET2},(0)} - \frac{1}{2} (E_1+E_2) S_{24} \right\} \\
 V^{\rm HT1} &= \left[ 1 - S_{14}^2 \right]^{-1} \left\{ V^{{\rm HT1},(0)} - \frac{1}{2} (E_1+E_2) S_{14} \right\} \\
 V^{\rm HT2} &= \left[ 1 - S_{23}^2 \right]^{-1} \left\{ V^{{\rm HT2},(0)} - \frac{1}{2} (E_1+E_2) S_{23} \right\} \\
 V^{\rm CT } &= \left[ 1 - S_{34}^2 \right]^{-1} \left\{ V^{{\rm CT },(0)} - \frac{1}{2} (E_1+E_2) S_{34} \right\} 
\end{align}
%
In the above equatons, the superscript (0) denotes that the matrix elements are not affected by the
overlap between molecular wavefunctions, and are given by
%
\begin{align}
 V^{{\rm Coul},(0)} &= \sum_{\mu\nu\in A} \sum_{\lambda\sigma\in B} 
  P_{\nu\mu}^{g\rightarrow e(A)} P_{\lambda\sigma}^{g\rightarrow e(B)} 
  (\mu\nu | \sigma\lambda) \\
 V^{{\rm Exch},(0)} &=-\frac{1}{2} \sum_{\mu\nu\in A} \sum_{\lambda\sigma\in B} 
  P_{\nu\mu}^{g\rightarrow e(A)} P_{\lambda\sigma}^{g\rightarrow e(B)} 
  (\mu\lambda | \sigma\nu) \\
 V^{{\rm ET1},(0)} &= t_{H\rightarrow L}^A \left\{ \big( L^A \big| \mathscr{F} \big| L^B \big) 
   + 2 \big( L^A H^A \big| H^A L^B \big) - \big( L^A L^B \big| H^A H^A \big)  \right\} \\
 V^{{\rm ET2},(0)} &= t_{H\rightarrow L}^B \left\{ \big( L^A \big| \mathscr{F} \big| L^B \big) 
   + 2 \big( L^A H^B \big| H^B L^B \big) - \big( L^A L^B \big| H^B H^B \big)  \right\} \\
 V^{{\rm HT1},(0)} &= t_{H\rightarrow L}^A \left\{-\big( H^A \big| \mathscr{F} \big| H^B \big) 
   + 2 \big( H^A L^A \big| L^A H^B \big) - \big( H^A H^B \big| L^A L^A \big)  \right\} \\
 V^{{\rm HT2},(0)} &= t_{H\rightarrow L}^B \left\{-\big( H^A \big| \mathscr{F} \big| H^B \big) 
   + 2 \big( H^A L^B \big| L^B H^B \big) - \big( H^A H^B \big| L^B L^B \big)  \right\} \\
 V^{{\rm CT },(0)} &= 
     2 \big( H^A L^B \big| L^A H^B \big) - \big( H^A H^B \big| L^A L^B \Big) 
\end{align}
%
In the above, $\mathscr{F}$ is the Fock operator whereas $H$ and $L$ denote the 
HOMO and LUMO orbitals, respectively. It was found that the overlap corrections
are quite negligible.


\subsection{\label{s:2.1}Indirect coupling via electron and hole-transfer}

\subsection{\label{s:2.2}Indirect coupling via charge transfer}

\subsection{\label{s:2.3}Approximate pure exchange coupling}

\subsection{\label{s:2.4}Complete model of EET coupling}

\section{\label{s:3}Calculation Details}

All the models that were used to test the theory presented in this work
were implemented in our in\hyp{}house plugin to {\sc Psi4} quantum chemistry program.\cite{Psi4.JCTC.2017}

\section{\label{s:4}Results and Discussion}

\section{\label{s:5}Summary and a few concluding remarks}

Bla.

\begin{acknowledgments}
This project is carried out under POLONEZ programme which has received funding from the European Union's
Horizon~2020 research and innovation programme under the Marie Skłodowska-Curie grant agreement 
No.~665778. This project is funded by National Science Centre, Poland 
(grant~no. 2016/23/P/ST4/01720) within the POLONEZ 3 fellowship.
\end{acknowledgments}

% -----------------------
\bibliography{references}
% -----------------------

\end{document}
