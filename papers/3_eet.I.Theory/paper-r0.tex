%%
%%   Version 3.1 of 16 May 2019.
%%
\documentclass[aip,jcp,amsmath,amssymb,reprint,floatfix]{revtex4-1}

\usepackage{graphicx}% Include figure files
\usepackage{dcolumn}% Align table columns on decimal point
\usepackage{bm}% bold math

% hyphenation
\usepackage{hyphenat}

% tables
\usepackage{multirow}
\usepackage{booktabs}

% mathematics
\usepackage{amsmath}
\usepackage{amsfonts}
\usepackage{amssymb}
\usepackage{amsbsy}
\usepackage{mathrsfs}
\usepackage{upgreek}
\usepackage[cbgreek]{textgreek}

%---------------------------------------------------
% Approximations
\newcommand{\Approx}[2]{\ensuremath{\text{Ap}_{#2} \left[ {#1} \right] }}
% happy integral
\newcommand{\rint}[1]{\mbox{\Large $ \int\limits_{\mbox{\tiny  $#1$}}$}}
% SHORTCUTS
%\newcolumntype{,}{D{.}{,}{2}}
\newcommand{\citee}[1]{\ensuremath{\scriptsize^{\citenum{#1}}}}
\newcommand{\HRule}{\rule{\linewidth}{0.2mm}}
% Quantum notation
\newcommand{\Bra}[1]{\ensuremath{\bigl\langle {#1} \bigl\lvert}}
\newcommand{\Ket}[1]{\ensuremath{\bigr\rvert {#1} \bigr\rangle}}
\newcommand{\BraKet}[2]{\ensuremath{\bigl\langle {#1} \bigl\lvert {#2} \bigr\rangle}}
\newcommand{\tBraKet}[3]{\ensuremath{\bigl\langle {#1} \bigl\lvert {#2} \bigl\lvert {#3} \bigr\rangle}}
%
\newcommand{\bra}[1]{\ensuremath{\bigl( {#1} \bigl\lvert}}
\newcommand{\ket}[1]{\ensuremath{\bigr\rvert {#1} \bigr)}}
\newcommand{\braket}[2]{\ensuremath{\bigl( {#1} \bigl\lvert {#2} \bigr)}}
\newcommand{\tbraket}[3]{\ensuremath{\bigl( {#1} \bigl\lvert {#2} \bigl\lvert {#3} \bigr)}}
% Math
\newcommand{\pd}{\ensuremath{\partial}}
\newcommand{\DR}{\ensuremath{{\rm d} {\bf r}}}
%\newcommand{\BM}[1]{\ensuremath{\mbox{\boldmath${#1}$}}}
\newcommand{\BM}[1]{\bm{#1}}
% Chemistry (formulas)
\newcommand{\ch}[2]{\ensuremath{\mathrm{#1}_{#2}}}
% Math 
\newcommand{\VEC}[1]{\ensuremath{\mathrm{\mathbf{#1}}}}
% vector nabla
\newcommand{\Nabla}{\ensuremath{ \BM{\nabla}}}
% derivative
\newcommand{\FDer}[3]{\ensuremath{
\bigg(
\frac{\partial #1}{\partial #2}
\bigg)_{#3}}}
% diagonal second derivative
\newcommand{\SDer}[3]{\ensuremath{
\biggl(
\frac{\partial^2 #1}{\partial #2^2}
\biggr)_{#3}}}
% off-diagonal second derivative
\newcommand{\SSDer}[4]{\ensuremath{
\biggl(
\frac{\partial^2 #1}{\partial #2 \partial #3}
\biggr)_{#4}}}
% derivatives without bound
% derivative
\newcommand{\fderiv}[2]{\ensuremath{
\frac{\partial #1}{\partial #2}}}
% diagonal second derivative
\newcommand{\sderiv}[2]{\ensuremath{
\frac{\partial^2 #1}{\partial #2^2}
}}
% off-diagonal second derivative
\newcommand{\sderivd}[3]{\ensuremath{
\frac{\partial^2 #1}{\partial #2 \partial #3}
}}
% derivatives for tables
\newcommand{\fderivm}[2]{\ensuremath{
{\partial #1}/{\partial #2}}}
% diagonal second derivative
\newcommand{\sderivm}[2]{\ensuremath{
{\partial^2 #1}/{\partial #2^2}
}}
% off-diagonal second derivative
\newcommand{\sderivdm}[3]{\ensuremath{
{\partial^2 #1}/{\partial #2 \partial #3}
}}
% ERIs and OEIs
\newcommand{\OEIc}[3]{\ensuremath{\left(#1 \lvert #2 \rvert #3 \right)}}
\newcommand{\ERIc}[4]{\ensuremath{\left(#1 #2 \vert #3 #4 \right)}}

% Partial density and potential
\newcommand{\PartPot}[4]{\ensuremath{\frac{#1 #2}{\lvert #3-#4 \rvert }}}

% trace operator
\DeclareMathOperator{\Tr}{Tr}

%\draft % marks overfull lines with a black rule on the right

% Define location of graphics
\graphicspath{{./figures/}}

\begin{document}
\preprint{AIP/123-OEP}

\title{Excitonic Energy Transfer Couplings at Short Distances: Efficient Transfer-Integral Fragmentation Approach
Based on Effective One-Electron Operators}

\author{Bartosz B{\l}asiak}
\email[]{blasiak.bartosz@gmail.com}
\homepage[]{https://www.polonez.pwr.edu.pl}
\author{Robert W. G{\'o}ra}
\author{Marta Cho{\l}uj} 
\author{Joanna D. Bednarska}
\author{Wojciech Bartkowiak}

\affiliation{Department of Physical and Quantum Chemistry, Faculty of Chemistry, 
Wroc{\l}aw University of Science and Technology, 
Wybrze{\.z}e Wyspia{\'n}skiego 27, Wroc{\l}aw 50-370, Poland}

\date{\today}

\begin{abstract}
What is EET? Short distances. Write about cost of current models. In this work, we developed a new model 
that is
based in the original TI/CIS model by Fujimoto. Write about cost reduction by introducing OEP's.
\end{abstract}

\pacs{}

\maketitle

\tableofcontents

\section{\label{s:1}Introduction}

Here why we want to care about EET at short distances. Why we want to devlop new model?
Provide literature review
of methods, then mention particularly the TDFI and TI/CIS model of Fujimoto, which
is the starting point. 

\subsection{\label{s:2.0}Fujimoto's model of EET Coupling: Review}

In the simplest version of the TI/CIS approach, the Hamiltonian of the 
complex of molecule $A$ and $B$ is constructed assuming the configuration interaction
approximation involving the following four configurations
%
\begin{subequations}
\begin{align}
 \big| \Phi_1 \big> &= \big| \Psi_A^{(e)} \otimes \Psi_B^{(g)} \big> \;, \\
 \big| \Phi_2 \big> &= \big| \Psi_A^{(g)} \otimes \Psi_B^{(e)} \big> \;, \\
 \big| \Phi_3 \big> &= \big| \Psi_A^{(+)} \otimes \Psi_B^{(-)} \big> \;, \\
 \big| \Phi_4 \big> &= \big| \Psi_A^{(-)} \otimes \Psi_B^{(+)} \big> \;,
\end{align}
\end{subequations}
%
where $g$ and $e$ superscripts denote the ground and excited state of a molecule,
`+' and `-' label the cationic and anionic state, respectively, 
whereas
%
\begin{equation}
\big| \Psi_X \otimes \Psi_Y \big> \equiv \mathscr{A} \Big\{ \big| \Psi_X \big> \otimes \big| \Psi_Y \big> \Big\}
\end{equation}
%
denotes the antisymmetrized Hartree product
of the monomer wavefunctions with $\mathscr{A}$ being the standard antisymmetrizer operator. 
Instead of diagonalizing the Hamiltonian in such basis,
Fujimoto used the perturbation theory to obtain the approximate expression for the EET coupling
constant,
%
\begin{multline}
  V \approx \left< \Phi_1 \vert \mathscr{H} \vert \Phi_2 \right> 
   - \sum_{n=3,4} \frac{\left< \Phi_1 \vert \mathscr{H} \vert \Phi_n \right> 
                        \left< \Phi_n \vert \mathscr{H} \vert \Phi_2 \right>}{E_n - E_1} \\
   + \sum_{m,n=3,4}^{m\ne n}
     \frac{\left< \Phi_1 \vert \mathscr{H} \vert \Phi_m \right>
           \left< \Phi_m \vert \mathscr{H} \vert \Phi_n \right>
           \left< \Phi_n \vert \mathscr{H} \vert \Phi_2 \right>}{(E_m-E_1)(E_n-E_1)} \;,
\end{multline}
%
where $E_n \equiv \left< \Phi_n \vert \mathscr{H} -E_{0} \vert \Phi_n \right>$
and $E_0$ is the ground state energy of the aggregate.
The first two site energies are given by
%
\begin{subequations}
\begin{align}
 E_1 \equiv \big< \Phi_1 \big| \mathscr{H} -E_{0} \big| \Phi_1 \big> &= E^A_{e\rightarrow g} 
  + \Delta E^A_{e\rightarrow g}(B) \;, \\
 E_2 \equiv \big< \Phi_2 \big| \mathscr{H} -E_{0} \big| \Phi_2 \big> &= E^B_{e\rightarrow g} 
  + \Delta E^B_{e\rightarrow g}(A) \;,
\end{align}
\end{subequations}
%
where $E^X_{e\rightarrow g}$ is the excitation energy of isolated molecule $X$ whereas 
$\Delta E^X_{e\rightarrow g}(Y)$ accounts for the environmental effect due to other molecule,
%
\begin{equation}
 \Delta E^X_{e\rightarrow g}(Y) \cong \int \left[ \rho^{X}_{{\rm el},(e)}({\bf r}) - \rho^{X}_{{\rm el},(g)}({\bf r}) \right]
         v^{{\rm eff},Y}({\bf r}) \; d{\bf r}
\end{equation}
%
with the effective one\hyp{}electron potential given by
%
\begin{multline}
 v^{{\rm eff},Y}({\bf r}_1) \equiv v^{Y}_{\rm nuc}({\bf r})
  + \sum_{ij}^{\rm Occ} \int d{\bf r}_2 \times \\
 \Big\{
 \frac{\phi_i^*({\bf r}_1) \phi_i({\bf r}_1)
                    \phi_j^*({\bf r}_2) \phi_j({\bf r}_2)}{\vert {\bf r}_1 - {\bf r}_2 \vert}
  - \frac{1}{2}
   \frac{\phi_i^*({\bf r}_1 \phi_j({\bf r}_1 
                 \phi_j^*({\bf r}_2 \phi_i({\bf r}_2)}{\vert {\bf r}_1 - {\bf r}_2 \vert}
  \Big\} \;.
\end{multline}
%
If the basis functions 3 and 4 are approximated as HOMO and LUMO orbitals of monomers,
the remaining two site energies are given by
%
\begin{subequations}
\begin{align}
 E_3 &\equiv \big< \Phi_3 \big| \mathscr{H} -E_{0} \big| \Phi_3 \big> \approx \nonumber \\ 
 &\qquad\qquad -\varepsilon_H^A + \varepsilon_L^B - \big( \phi_H^A \phi_H^A \big| \phi_L^B \phi_L^B \big)  \;, \\
 E_4 &\equiv \big< \Phi_4 \big| \mathscr{H} -E_{0} \big| \Phi_4 \big> \approx \nonumber \\
 &\qquad\qquad \varepsilon_L^A - \varepsilon_H^B - \big( \phi_L^A \phi_L^A \big| \phi_H^B \phi_H^B \big)  \;.
\end{align}
\end{subequations}
%
In the above, $\varepsilon_j^X$ is the energy of the $j$th Hartree-Fock orbital $\phi_j^X$ associated with molecule $X$,
and the two\hyp{}electron integral is defined by
%
\begin{equation}
	\braket{\alpha\beta}{\gamma\delta} \equiv
	\iint 
	\frac{ \phi_\alpha^{*}({\bf r}_1) \phi_\beta({\bf r}_1) 
	       \phi_\gamma^{*}({\bf r}_2) \phi_\delta({\bf r}_2) }{ \vert {\bf r}_1 - {\bf r}_2 \vert}
	d{\bf r}_1 d{\bf r}_2  \;.
\end{equation}
%

%First, in Section II we briefly review the Fujimoto's TI/CIS model of EET coupling
%which is treated in this work as a reference.
%Then, further in this Section, we derive the OEP model based on the TI/CIS model (here referred to as the
%which is validated in Section III against the TI/CIS method for a few model complexes. 
First, in Section II we derive the OEP model based on the TI/CIS model (here referred to as the
OEP-TI/CIS model). Next, we validate it against the TI/CIS method for a few model complexes
in Section III. 
Finally, we conclude our work in Section IV.

\section{\label{s:2}OEP-Based Model}


%The associated off\hyp{}diagonal Hamiltonian matrix elements are
%%
%\begin{align}
% \Big< \Phi_1 \Big| \mathscr{H} \Big| \Phi_2 \Big> &\equiv V^{\rm Coul} + V^{\rm Exch} + V^{\rm Ovrl}\\
% \Big< \Phi_1 \Big| \mathscr{H} \Big| \Phi_3 \Big> &\equiv V^{\rm ET1} \\
% \Big< \Phi_2 \Big| \mathscr{H} \Big| \Phi_4 \Big> &\equiv V^{\rm ET2} \\
% \Big< \Phi_1 \Big| \mathscr{H} \Big| \Phi_4 \Big> &\equiv V^{\rm HT1} \\
% \Big< \Phi_2 \Big| \mathscr{H} \Big| \Phi_3 \Big> &\equiv V^{\rm HT2} \\
% \Big< \Phi_3 \Big| \mathscr{H} \Big| \Phi_4 \Big> &\equiv V^{\rm CT } 
%\end{align}
%%
%where the Forster-type Coulombic (Coul), Dexter-type exchange (Exch), remaining overlap correction (Ovrl),
%as well
%as the electron, hole and charge (ET, HT, CT) transfer contributions are defined.


%In the above equatons, the superscript (0) denotes that the matrix elements are not affected by the
%overlap between molecular wavefunctions, and are given by
%%
%\begin{align}
% V^{{\rm Coul},(0)} &= \sum_{\mu\nu\in A} \sum_{\lambda\sigma\in B} 
%  P_{\nu\mu}^{g\rightarrow e(A)} P_{\lambda\sigma}^{g\rightarrow e(B)} 
%  (\mu\nu | \sigma\lambda) \\
% V^{{\rm Exch},(0)} &=-\frac{1}{2} \sum_{\mu\nu\in A} \sum_{\lambda\sigma\in B} 
%  P_{\nu\mu}^{g\rightarrow e(A)} P_{\lambda\sigma}^{g\rightarrow e(B)} 
%  (\mu\lambda | \sigma\nu) \\
% V^{{\rm HT1},(0)} &= t_{H\rightarrow L}^A \left\{-\big( H^A \big| \mathscr{F} \big| H^B \big) 
%   + 2 \big( H^A L^A \big| L^A H^B \big) - \big( H^A H^B \big| L^A L^A \big)  \right\} \\
% V^{{\rm HT2},(0)} &= t_{H\rightarrow L}^B \left\{-\big( H^A \big| \mathscr{F} \big| H^B \big) 
%   + 2 \big( H^A L^B \big| L^B H^B \big) - \big( H^A H^B \big| L^B L^B \big)  \right\} \\
% V^{{\rm CT },(0)} &= 
%     2 \big( H^A L^B \big| L^A H^B \big) - \big( H^A H^B \big| L^A L^B \Big) 
%\end{align}
%%
%In the above, $\mathscr{F}$ is the Fock operator whereas $H$ and $L$ denote the 
%HOMO and LUMO orbitals, respectively. It was found that the overlap corrections
%are quite negligible.

To account for the EET coupling via charge transfer states 

\subsection{\label{s:2.1}Indirect coupling via electron and hole-transfer}

The matrix elements that are needed for the electron transfer (ET) phenomena
is
%
\begin{multline}\label{e:v0-et1}
 V^{{\rm ET1},(0)} \equiv \left< \Phi_1 \vert \mathscr{H} \vert \Phi_3 \right> \approx 
 t_{H\rightarrow L}^A \times \Big\{ 
 \big( \phi_L^A \big| \mathscr{F} \big| \phi_L^B \big) \\
   + 2 \big( \phi_L^A \phi_H^A \big| \phi_H^A \phi_L^B \big) - \big( \phi_L^A \phi_L^B \big| \phi_H^A \phi_H^A \big) 
 \Big\} \;,
\end{multline}
%
along with its twin term, $V^{{\rm ET2},(0)} \equiv \left< \Phi_2 \vert \mathscr{H} \vert \Phi_4 \right>$,
which can be found in the original work of Fujimoto.
In the above, $\mathscr{F}$ is the Fock operator of the entire molecular aggregate.
Here, we approximate the Fock operator by constructing it from the unperturbed
Fock operators of isolated monomers,
%
\begin{equation}
 \mathscr{F} \approx \mathscr{F}^{A}_0 + \mathscr{F}^{B}_0 \;,
\end{equation}
%
where
%
\begin{equation}
 \mathscr{F}^{X}_0 = \hat{T}^{X} + \hat{V}^{X}_{\rm nuc} + \hat{J}({\bf P}^{X}_g) - \frac{1}{2} \hat{K}({\bf P}^{X}_g) \;.
\end{equation}
%
Note that, without loss of generality, one can rewrite Eq.~\eqref{e:v0-et1}
as
%
\begin{multline}
 V^{{\rm ET1},(0)} = t_{H\rightarrow L}^A \times \Big\{ 
 \tbraket{\phi_L^A}{\mathscr{F}}{\phi_L^B} \\
  + 2 \tbraket{\phi_H^A}{\hat{v}_{LH}^A}{\phi_L^B}
  -   \tbraket{\phi_L^A}{\hat{v}_{HH}^A}{\phi_L^B}
 \Big\} \;,
\end{multline}
%
in which the effective potential operator\cite{Blasiak.Bednarska.Choluj.Bartkowiak.XXXX} is defined by
%
\begin{equation}
 \hat{v}_{ij}^X = \int d{\bf r} \Ket{{\bf r}} v_{ij}^X({\bf r})
 \Bra{{\bf r}}
\end{equation}
%
with its spatial form given as
%
\begin{equation}
 v_{ij}^X({\bf r}) \equiv \int d{\bf r}' \frac{\phi_i^*({\bf r}') \phi_j({\bf r}') }{\vert {\bf r} - {\bf r}'\vert} \;.
\end{equation}
%
Note here that $\phi_j({\bf r}) \equiv \BraKet{{\bf r}}{\phi_j}$.
Now, 
by using the OEP technique\cite{Blasiak.Bednarska.Choluj.Bartkowiak.XXXX}
for overlap\hyp{}like effective potential operator matrix elements of the type AB
one can gather operators associated to a particular molecule and group them
into one effective one\hyp{}electron operator
as
%
\begin{equation}
 \bra{\phi_L^A} \left\{ \mathscr{F}_0^A - \hat{v}^A_{HH} \right\}
 + \bra{\phi_H^A} 2 \hat{v}^A_{LH} 
 \cong \sum_{\xi\in A}^{\rm DF} \bra{\xi} V^{A,{\rm ET1}}_{\xi,HL}
\end{equation}
%
and
%
\begin{equation}
 \mathscr{F}_0^B \ket{\phi_L^B} \cong \sum_{\eta\in B}^{\rm DF} V^{B,{\rm ET1}}_{\eta,L} \ket{\eta} 
\end{equation}
%
In the above transformations, that involve the generalized density fitting
in an auxiliary basis set space,
$V^{A,{\rm ET1}}_{\xi,HL}$ and $V^{B,{\rm ET1}}_{\eta,L}$
are the OEP matrix elements associated with the above
effective potentials, separately defined for molecule $A$ and $B$. This allows one to
further recast $V^{{\rm ET1},(0)}$ into
%
\begin{multline}\label{e:v0-et1-oep}
 V^{{\rm ET1},(0)} \cong t_{H\rightarrow L}^A \Big\{ 
 \sum_{\xi\in A}^{\rm DF} S_{\xi L}^{AB} V^{A,{\rm ET1}}_{\xi,HL} +
 \sum_{\eta\in B}^{\rm DF} S_{\eta L}^{BA} V^{B,{\rm ET1}}_{\eta,L}
 \Big\} \;,
\end{multline}
%
which has a particularly simple form as compared to the original expression
in Eq.~\eqref{e:v0-et1} that involves four\hyp{}center electron repulsion integrals.
Note that if 
the OEP matrices are considered
as effective fragment parameters, %pre\hyp{}computed only once.
only overlap integrals between auxiliary basis functions and LUMO orbitals of molecule
$A$ and $B$ are necessary to compute, i.e.,
%
\begin{equation}
 S^{XY}_{\xi U} = \sum_{\beta\in Y} S_{\xi\beta}^{XY} C_{\beta U}^Y \text{ for $\xi\in X$} \;,
\end{equation}
%
in which 
%
\begin{equation}
 S_{\xi\beta}^{XY} = \int d{\bf r} \varphi_\xi^*({\bf r}) \varphi_\beta({\bf r})
\end{equation}
%
and $C_{\beta U}^Y$ is the SCF LCAO-MO matrix for isolated molecule $Y$.
It is straightforward to show that the twin term ET2 is
%
\begin{multline}\label{e:v0-et2-oep}
 V^{{\rm ET2},(0)} \cong t_{H\rightarrow L}^B \Big\{ 
 \sum_{\eta\in B}^{\rm DF} S_{\eta L}^{BA} V^{B,{\rm ET1}}_{\eta,HL} +
 \sum_{\xi\in A}^{\rm DF} S_{\xi L}^{AB} V^{A,{\rm ET1}}_{\xi,L}
 \Big\} \;.
\end{multline}
%

Similar considerations apply to the hole transfer (HT) matrix elements.
The expression
%
\begin{multline}\label{e:v0-ht1}
 V^{{\rm HT1},(0)} \equiv \left< \Phi_1 \vert \mathscr{H} \vert \Phi_4 \right> \approx 
 t_{H\rightarrow L}^A \times \Big\{ 
-\big(\phi_H^A \big| \mathscr{F} \big| \phi_H^B \big) \\
   + 2 \big( \phi_H^A \phi_L^A \big| \phi_L^A \phi_H^B \big) - \big( \phi_H^A \phi_H^B \big| \phi_L^A \phi_L^A \big) 
 \Big\} 
\end{multline}
%
can be recast as
%
\begin{multline}\label{e:v0-ht1-oep}
 V^{{\rm HT1},(0)} \cong t_{H\rightarrow L}^A \Big\{ 
 \sum_{\xi\in A}^{\rm DF} S_{\xi H}^{AB} V^{A,{\rm HT1}}_{\xi,HL} +
 \sum_{\eta\in B}^{\rm DF} S_{\eta H}^{BA} V^{B,{\rm HT1}}_{\eta,H}
 \Big\} \;,
\end{multline}
%
with the effective potential matrices given by
%
%
\begin{subequations}
\begin{align}
 \bra{\phi_H^A} \left\{ -\mathscr{F}_0^A - \hat{v}^A_{LL} \right\}
 + \bra{\phi_L^A} 2 \hat{v}^A_{HL} 
 &\cong \sum_{\xi\in A}^{\rm DF} \bra{\xi} V^{A,{\rm HT1}}_{\xi,HL} \\
 -\mathscr{F}_0^B \ket{\phi_H^B} &\cong \sum_{\eta\in B}^{\rm DF} V^{B,{\rm ET1}}_{\eta,H} \ket{\eta} 
\end{align}
\end{subequations}
%
The twin term HT2 is accordingly
%
\begin{multline}\label{e:v0-ht2-oep}
 V^{{\rm HT2},(0)} \cong t_{H\rightarrow L}^B \Big\{ 
 \sum_{\eta\in B}^{\rm DF} S_{\eta H}^{BA} V^{B,{\rm HT2}}_{\eta,HL} +
 \sum_{\xi\in A}^{\rm DF} S_{\xi H}^{AB} V^{A,{\rm HT2}}_{\xi,H}
 \Big\} \;.
\end{multline}
%
As can be seen, to compute Hamiltonian matrix elements
associated with the electron and hole transfer, eight in total
different OEP matrices need to be pre\hyp{}computed and stored in a file.
Note that each of these matrices is actually just a vector of length
equal to the size of the auxiliary basis set chosen. Therefore,
it can be predicted that the computational cost of evaluation of the OEP\hyp{}based Hamiltonian
matrix elements is negligible as compared to the original TI/CIS method.
In Appendix~\ref{a:gdf-formulae}, explicit working formulae
for the OEP matrix elements are given in terms of the AO's.

\subsection{\label{s:2.2}Indirect coupling via charge transfer}

\subsection{\label{s:2.3}Approximate pure exchange coupling}

\subsection{\label{s:2.4}Complete model of EET coupling}

To summarize, the overall form of the OEP-TI/CIS model is still given as in the original TI/CIS model
by
%
\begin{equation}
  V \approx V^{\rm Direct} + V^{\rm Inirect} \;,
\end{equation}
%
where the so called `direct' and `indirect' coupling constants are respectively
%
\begin{subequations}
\begin{align}
  V^{\rm Direct  } &\equiv V^{\rm Coul} + V^{\rm Exch} + V^{\rm Ovrl} \;,\\
  V^{\rm Indirect} &\equiv V^{\rm TI-2} + V^{\rm TI-3} \;,
\end{align}
\end{subequations}
%
with
%
\begin{subequations}
\begin{align}
 V^{\rm TI-2} &\equiv-\frac{V^{\rm ET1} V^{\rm HT2}}{E_3-E_1} -\frac{V^{\rm ET2} V^{\rm HT1}}{E_4-E_1}  \;,\\
 V^{\rm TI-3} &\equiv \frac{V^{\rm CT} \left( V^{\rm ET1} V^{\rm ET2} + V^{\rm HT1} V^{\rm HT2}\right) }{(E_3-E_1)(E_4-E_1)} \;.
\end{align}
\end{subequations}
%
However, 
the differences between the original Fujimoto's model and the newly derived OEP\hyp{}based model 
are: 
%
\begin{enumerate}
 \item The Hamiltonian matrix elements are evaluated from the fragment effective 
parameters, for which only the one\hyp{}electron integrals are needed. Therefore, the computational
cost is significantly reduced as compared to the parent TI/CIS model. In particular, 
$V^{\rm Coul}$ is evaluated by using the TrCAMM method developed previously,
$V^{\rm Exch}$ and $V^{\rm CT}$ are computed from the Mulliken approximated
formulae in Eqs. XXX, and XXX, as well as $V^{\rm ET1}$, $V^{\rm ET2}$,
$V^{\rm HT1}$ and $V^{\rm HT2}$ are treated by the generalized density fitting
with OEP effective matrices as effective parameters spanned in auxiliary AO basis set 
(Eqs.XXX); 
 \item The relaxation effects due to the intermolecular interactions are not included
in the current OEP\hyp{}based model.
Note that in the original TI/CIS model the basis states and all the one\hyp{}particle density matrices are built from the
self\hyp{}consistently adjusted fragment densities according to the DFI method.
\end{enumerate}
%
\paragraph*{Overlap corrections to the Hamiltonian.}
Since in the TI/CIS method the basis states are constructed from the monomer wavefunctions,
the overlap effects should be properly taken into account. It was reported previously
that overlap correction to the $\left< \Phi_1 \vert \mathscr{H} \vert \Phi_2 \right>$ 
matrix element brings negligible contributions to the 
total EET coupling (c.f. the `overlap' contribution in Ref.). On the other hand,
the importance of the overlap effects on the other off\hyp{}diagonal Hamiltonian matrix elements were not directly studied. 
After the overlap correction, the constituents of the direct coupling take the form
%
\begin{subequations}\label{e:ovrl-direct}
\begin{align}
 V^{\rm Coul} &= \frac{V^{{\rm Coul},(0)}}{1 - S_{12}^2} \;, \\
 V^{\rm Exch} &= \frac{V^{{\rm Exch},(0)}}{1 - S_{12}^2} \;, \\
 V^{\rm Ovrl} &=-\frac{(E_1+E_2)S_{12}}{2(1-S_{12}^2)}   \;.
\end{align}
\end{subequations}
%
The overlap\hyp{}corrected ET, HT and CT matrix elements read
%
\begin{equation}\label{e:ovrl-indirect}
 V^{\rm t} = \left[ 1 - S_{\rm t}^2 \right]^{-1} \left\{ V^{{\rm t},(0)} - \frac{1}{2} (E_1+E_2) S_{\rm t} \right\} \;,
\end{equation}
%
where `t' denotes one of the matrix element types (ET1, ET2, HT1, HT2, CT) 
and $S_{\rm t}$ is the appropriate overlap integral between basis states.
Since evaluation
of Eqs.~\eqref{e:ovrl-direct} and \eqref{e:ovrl-indirect} 
is computationally inexpensive, we shall consider the overlap\hyp{}corrected
Hamiltonian in the present work.


\section{\label{s:4}Results and Discussion}

All the models that were used to test the theory presented in this work
were implemented in our in\hyp{}house plugin to {\sc Psi4} quantum chemistry program.\cite{Psi4.JCTC.2017}


\section{\label{s:5}Summary and a few concluding remarks}

Bla.

\begin{acknowledgments}
This project is carried out under POLONEZ programme which has received funding from the European Union's
Horizon~2020 research and innovation programme under the Marie Skłodowska-Curie grant agreement 
No.~665778. This project is funded by National Science Centre, Poland 
(grant~no. 2016/23/P/ST4/01720) within the POLONEZ 3 fellowship.
\end{acknowledgments}

\appendix

\section{Explicit formulae for ET and HT OEP matrix elements\label{a:gdf-formulae}}

As developed in Ref.\cite{Blasiak.Bednarska.Choluj.Bartkowiak.XXXX}, 
the matrix elements of the OEP operators
are computed from the generalized density fitting scheme, according to which
%
\begin{equation}
 V^X_{\xi, M} = \sum_{\xi'\in X}^{\rm DF} 
                \sum_{\varepsilon\in X}^{\rm RI}
                [{\bf R}^{-1}]_{\xi\xi'} R_{\xi'\varepsilon} H^X_{\varepsilon, M} \;,
\end{equation}
%
where
%
\begin{equation}
 H^X_{\varepsilon, M} = \sum_{\varepsilon'\in X}^{\rm RI}
                        [{\bf S}^{-1}]_{\varepsilon\varepsilon'} a_{\varepsilon'}^{X; M} \;.
\end{equation}
%
In the above equations, the auxiliary matrices are defined by
%
\begin{equation}
 S_{\eta\xi}  = \int 
                       \varphi^*_\eta({\bf r}) \varphi_\xi({\bf r}) 
                 d{\bf r} 
\end{equation}
%
and
%
\begin{equation}
 R_{\eta\xi}  = \iint 
                       \frac{ \varphi^*_\eta({\bf r}_1) \varphi_\xi({\bf r}_2) } 
                            {\vert {\bf r}_1 - {\bf r}_2\vert}  
                 d{\bf r}_1 d{\bf r}_2 \;.
\end{equation}
%
The elements of vector ${\bf a}^X$ for $M=$ ET1, ET2, HT1 and HT2 and $X=A, B$ are given by
%
\begin{subequations}
\begin{align}
a_{\alpha}^{A;{\rm ET1}} &= \sum_\beta C_{\beta L}^A F_{\alpha\beta}^A \nonumber \\
 &+\sum_{\beta\gamma\delta} \braket{\alpha\beta}{\gamma\delta}
 \left\{ 2C^A_{\beta  H} C^A_{\gamma L} - C^A_{\beta  L} C^A_{\gamma H}\right\} C^A_{\delta H} \;,\\
%
a_{\alpha}^{B;{\rm ET1}} &= \sum_\beta C_{\beta L}^B F_{\alpha\beta}^B \;,\\
%
a_{\alpha}^{B;{\rm ET2}} &= \sum_\beta C_{\beta L}^B F_{\alpha\beta}^B \nonumber \\
 &+\sum_{\beta\gamma\delta} \braket{\alpha\beta}{\gamma\delta}
 \left\{ 2C^B_{\beta  H} C^B_{\gamma L} - C^B_{\beta  L} C^B_{\gamma H}\right\} C^B_{\delta H} \;,\\
%
a_{\alpha}^{A;{\rm ET2}} &= \sum_\beta C_{\beta L}^A F_{\alpha\beta}^A \;,\\
%
a_{\alpha}^{A;{\rm HT1}} &=-\sum_\beta C_{\beta H}^A F_{\alpha\beta}^A \nonumber \\
 &+\sum_{\beta\gamma\delta} \braket{\alpha\beta}{\gamma\delta}
 \left\{ 2C^A_{\beta  L} C^A_{\gamma H} - C^A_{\beta  H} C^A_{\gamma L}\right\} C^A_{\delta L} \;,\\
%
a_{\alpha}^{B;{\rm HT1}} &=-\sum_\beta C_{\beta L}^B F_{\alpha\beta}^B \;,\\
%
a_{\alpha}^{B;{\rm HT2}} &=-\sum_\beta C_{\beta H}^B F_{\alpha\beta}^B \nonumber \\
 &+\sum_{\beta\gamma\delta} \braket{\alpha\beta}{\gamma\delta}
 \left\{ 2C^B_{\beta  L} C^B_{\gamma H} - C^B_{\beta  H} C^B_{\gamma L}\right\} C^B_{\delta L} \;,\\
%
a_{\alpha}^{A;{\rm HT2}} &=-\sum_\beta C_{\beta L}^A F_{\alpha\beta}^A \;.
\end{align}
\end{subequations}
%
% -----------------------
\bibliography{references}
% -----------------------

\end{document}
