%%
%%   Version 3.1 of 16 May 2019.
%%

\usepackage{graphicx}% Include figure files
\usepackage{dcolumn}% Align table columns on decimal point
\usepackage{bm}% bold math

% hyphenation
\usepackage{hyphenat}

% tables
\usepackage{multirow}
\usepackage{booktabs}

% mathematics
\usepackage{amsmath}
\usepackage{amsfonts}
\usepackage{amssymb}
\usepackage{amsbsy}
\usepackage{mathrsfs}
\usepackage{upgreek}
\usepackage[cbgreek]{textgreek}

%---------------------------------------------------
% Approximations
\newcommand{\Approx}[2]{\ensuremath{\text{Ap}_{#2} \left[ {#1} \right] }}
% happy integral
\newcommand{\rint}[1]{\mbox{\Large $ \int\limits_{\mbox{\tiny  $#1$}}$}}
% SHORTCUTS
%\newcolumntype{,}{D{.}{,}{2}}
\newcommand{\citee}[1]{\ensuremath{\scriptsize^{\citenum{#1}}}}
\newcommand{\HRule}{\rule{\linewidth}{0.2mm}}
% Quantum notation
\newcommand{\Bra}[1]{\ensuremath{\bigl\langle {#1} \bigl\lvert}}
\newcommand{\Ket}[1]{\ensuremath{\bigr\rvert {#1} \bigr\rangle}}
\newcommand{\BraKet}[2]{\ensuremath{\bigl\langle {#1} \bigl\lvert {#2} \bigr\rangle}}
\newcommand{\tBraKet}[3]{\ensuremath{\bigl\langle {#1} \bigl\lvert {#2} \bigl\lvert {#3} \bigr\rangle}}
%
\newcommand{\bra}[1]{\ensuremath{\bigl( {#1} \bigl\lvert}}
\newcommand{\ket}[1]{\ensuremath{\bigr\rvert {#1} \bigr)}}
\newcommand{\braket}[2]{\ensuremath{\bigl( {#1} \bigl\lvert {#2} \bigr)}}
\newcommand{\tbraket}[3]{\ensuremath{\bigl( {#1} \bigl\lvert {#2} \bigl\lvert {#3} \bigr)}}
% Math
\newcommand{\pd}{\ensuremath{\partial}}
\newcommand{\DR}{\ensuremath{{\rm d} {\bf r}}}
%\newcommand{\BM}[1]{\ensuremath{\mbox{\boldmath${#1}$}}}
\newcommand{\BM}[1]{\bm{#1}}
% Chemistry (formulas)
\newcommand{\ch}[2]{\ensuremath{\mathrm{#1}_{#2}}}
% Math 
\newcommand{\VEC}[1]{\ensuremath{\mathrm{\mathbf{#1}}}}
% vector nabla
\newcommand{\Nabla}{\ensuremath{ \BM{\nabla}}}
% derivative
\newcommand{\FDer}[3]{\ensuremath{
\bigg(
\frac{\partial #1}{\partial #2}
\bigg)_{#3}}}
% diagonal second derivative
\newcommand{\SDer}[3]{\ensuremath{
\biggl(
\frac{\partial^2 #1}{\partial #2^2}
\biggr)_{#3}}}
% off-diagonal second derivative
\newcommand{\SSDer}[4]{\ensuremath{
\biggl(
\frac{\partial^2 #1}{\partial #2 \partial #3}
\biggr)_{#4}}}
% derivatives without bound
% derivative
\newcommand{\fderiv}[2]{\ensuremath{
\frac{\partial #1}{\partial #2}}}
% diagonal second derivative
\newcommand{\sderiv}[2]{\ensuremath{
\frac{\partial^2 #1}{\partial #2^2}
}}
% off-diagonal second derivative
\newcommand{\sderivd}[3]{\ensuremath{
\frac{\partial^2 #1}{\partial #2 \partial #3}
}}
% derivatives for tables
\newcommand{\fderivm}[2]{\ensuremath{
{\partial #1}/{\partial #2}}}
% diagonal second derivative
\newcommand{\sderivm}[2]{\ensuremath{
{\partial^2 #1}/{\partial #2^2}
}}
% off-diagonal second derivative
\newcommand{\sderivdm}[3]{\ensuremath{
{\partial^2 #1}/{\partial #2 \partial #3}
}}
% ERIs and OEIs
\newcommand{\OEIc}[3]{\ensuremath{\left(#1 \lvert #2 \rvert #3 \right)}}
\newcommand{\ERIc}[4]{\ensuremath{\left(#1 #2 \vert #3 #4 \right)}}

% Partial density and potential
\newcommand{\PartPot}[4]{\ensuremath{\frac{#1 #2}{\lvert #3-#4 \rvert }}}

% trace operator
\DeclareMathOperator{\Tr}{Tr}

%\draft % marks overfull lines with a black rule on the right

% Define location of graphics
\graphicspath{{./figures/}}

\begin{document}
\preprint{AIP/123-OEP}

\title{Excitonic Energy Transfer Couplings at Short Distances: Efficient Transfer-Integral Fragmentation Approach
Based on Effective One-Electron Operators}

\author{Bartosz B{\l}asiak}
\email[]{blasiak.bartosz@gmail.com}
\homepage[]{https://www.polonez.pwr.edu.pl}
\author{Robert W. G{\'o}ra}
\author{Marta Cho{\l}uj} 
\author{Joanna D. Bednarska}
\author{Wojciech Bartkowiak}

\affiliation{Department of Physical and Quantum Chemistry, Faculty of Chemistry, 
Wroc{\l}aw University of Science and Technology, 
Wybrze{\.z}e Wyspia{\'n}skiego 27, Wroc{\l}aw 50-370, Poland}

\date{\today}

\begin{abstract}
What is EET? Short distances. Write about cost of current models. In this work, we developed a new model 
that is
based in the original TI/CIS model by Fujimoto. Write about cost reduction by introducing OEP's.
\end{abstract}

\pacs{}

\maketitle

\tableofcontents

\section{\label{s:1}Introduction}

Here why we want to care about EET at short distances. Why we want to devlop new model?
Provide literature review
of methods, then mention particularly the TDFI and TI/CIS model of Fujimoto, which
is the starting point. 

\subsection{\label{s:2.0}Fujimoto's model of EET Coupling: Review}

In the simplest version of the TI/CIS approach, the Hamiltonian of the 
complex of molecule $A$ and $B$ is constructed assuming the configuration interaction
approximation involving the following four configurations
%
\begin{subequations}
\begin{align}
 \big| \Phi_1 \big> &= \big| \Psi_A^{(e)} \otimes \Psi_B^{(g)} \big> \;, \\
 \big| \Phi_2 \big> &= \big| \Psi_A^{(g)} \otimes \Psi_B^{(e)} \big> \;, \\
 \big| \Phi_3 \big> &= \big| \Psi_A^{(+)} \otimes \Psi_B^{(-)} \big> \;, \\
 \big| \Phi_4 \big> &= \big| \Psi_A^{(-)} \otimes \Psi_B^{(+)} \big> \;,
\end{align}
\end{subequations}
%
where $g$ and $e$ superscripts denote the ground and excited state of a molecule,
`+' and `-' label the cationic and anionic state, respectively, 
whereas
%
\begin{equation}
\big| \Psi_X \otimes \Psi_Y \big> \equiv \mathscr{A} \Big\{ \big| \Psi_X \big> \otimes \big| \Psi_Y \big> \Big\}
\end{equation}
%
denotes the antisymmetrized Hartree product
of the monomer wavefunctions with $\mathscr{A}$ being the standard antisymmetrizer operator. 
Instead of diagonalizing the Hamiltonian in such basis,
Fujimoto used the perturbation theory to obtain the approximate expression for the EET coupling
constant,
%
\begin{multline}\label{e:v0-fujimoto}
  V \approx \left< \Phi_1 \vert \mathscr{H} \vert \Phi_2 \right> 
   - \sum_{n=3,4} \frac{\left< \Phi_1 \vert \mathscr{H} \vert \Phi_n \right> 
                        \left< \Phi_n \vert \mathscr{H} \vert \Phi_2 \right>}{E_n - E_1} \\
   + \sum_{m,n=3,4}^{m\ne n}
     \frac{\left< \Phi_1 \vert \mathscr{H} \vert \Phi_m \right>
           \left< \Phi_m \vert \mathscr{H} \vert \Phi_n \right>
           \left< \Phi_n \vert \mathscr{H} \vert \Phi_2 \right>}{(E_m-E_1)(E_n-E_1)} \;,
\end{multline}
%
where $E_n \equiv \left< \Phi_n \vert \mathscr{H} -E_{0} \vert \Phi_n \right>$
and $E_0$ is the ground state energy of the aggregate.
The first two site energies are given by
%
\begin{subequations}
\begin{align}
 E_1 \equiv \big< \Phi_1 \big| \mathscr{H} -E_{0} \big| \Phi_1 \big> &= E^A_{e\rightarrow g} 
  + \Delta E^A_{e\rightarrow g}(B) \;, \\
 E_2 \equiv \big< \Phi_2 \big| \mathscr{H} -E_{0} \big| \Phi_2 \big> &= E^B_{e\rightarrow g} 
  + \Delta E^B_{e\rightarrow g}(A) \;,
\end{align}
\end{subequations}
%
where $E^X_{e\rightarrow g}$ is the excitation energy of isolated molecule $X$ whereas 
$\Delta E^X_{e\rightarrow g}(Y)$ accounts for the environmental effect due to the other molecule,
%
\begin{equation}
 \Delta E^X_{e\rightarrow g}(Y) \cong \int \left[ \rho^{X}_{{\rm el},(e)}({\bf r}) - \rho^{X}_{{\rm el},(g)}({\bf r}) \right]
         v^{{\rm eff},Y}({\bf r}) \; d{\bf r}
\end{equation}
%
with the effective one\hyp{}electron potential given by
%
\begin{multline}
 v^{{\rm eff},Y}({\bf r}_1) \equiv v^{Y}_{\rm nuc}({\bf r})
  + \sum_{ij}^{\rm Occ} \int d{\bf r}_2 \times \\
 \Big\{
 \frac{\phi_i^*({\bf r}_1) \phi_i({\bf r}_1)
                    \phi_j^*({\bf r}_2) \phi_j({\bf r}_2)}{\vert {\bf r}_1 - {\bf r}_2 \vert}
  - \frac{1}{2}
   \frac{\phi_i^*({\bf r}_1) \phi_j({\bf r}_1)
                 \phi_j^*({\bf r}_2) \phi_i({\bf r}_2)}{\vert {\bf r}_1 - {\bf r}_2 \vert}
  \Big\} \;.
\end{multline}
%
If the basis functions 3 and 4 are approximated as HOMO and LUMO orbitals of monomers,
the remaining two site energies are given by
%
\begin{subequations}
\begin{align}
 E_3 &\equiv \big< \Phi_3 \big| \mathscr{H} -E_{0} \big| \Phi_3 \big> \approx \nonumber \\ 
 &\qquad\qquad -\varepsilon_H^A + \varepsilon_L^B - \big( \phi_H^A \phi_H^A \big| \phi_L^B \phi_L^B \big)  \;, \\
 E_4 &\equiv \big< \Phi_4 \big| \mathscr{H} -E_{0} \big| \Phi_4 \big> \approx \nonumber \\
 &\qquad\qquad \varepsilon_L^A - \varepsilon_H^B - \big( \phi_L^A \phi_L^A \big| \phi_H^B \phi_H^B \big)  \;.
\end{align}
\end{subequations}
%
In the above, $\varepsilon_j^X$ is the energy of the $j$th Hartree-Fock orbital $\phi_j^X$ associated with molecule $X$,
and the two\hyp{}electron integral is defined by
%
\begin{equation}
	\braket{\alpha\beta}{\gamma\delta} \equiv
	\iint 
	\frac{ \phi_\alpha^{*}({\bf r}_1) \phi_\beta({\bf r}_1) 
	       \phi_\gamma^{*}({\bf r}_2) \phi_\delta({\bf r}_2) }{ \vert {\bf r}_1 - {\bf r}_2 \vert}
	d{\bf r}_1 d{\bf r}_2  \;.
\end{equation}
%

The first term in Eq.~\eqref{e:v0-fujimoto} is the, so called, `direct' coupling, which is 
composed of two contributions:
%
\begin{equation}
 \left< \Phi_1 \vert \mathscr{H} \vert \Phi_2 \right> = V^{{\rm Coul},(0)} + V^{{\rm Exch},(0)}
\end{equation}
%
where the F{\"o}rster (or Coulomb) coupling is given by
%
\begin{equation}
 V^{{\rm Coul},(0)} = \sum_{\mu\nu\in A} \sum_{\lambda\sigma\in B} 
  P_{\nu\mu}^{g\rightarrow e(A)} P_{\lambda\sigma}^{g\rightarrow e(B)} 
  \braket{\mu\nu}{\sigma\lambda}  \;,
\end{equation}
%
whereas the Dexter (or exchange) coupling is
%
\begin{equation}
 V^{{\rm Exch},(0)} =-\frac{1}{2} \sum_{\mu\nu\in A} \sum_{\lambda\sigma\in B} 
  P_{\nu\mu}^{g\rightarrow e(A)} P_{\lambda\sigma}^{g\rightarrow e(B)} 
  \braket{\mu\nu}{\sigma\lambda}  \;.
\end{equation}
%
In the above equations, ${\bf P}^{g\rightarrow e(X)}$ is the transition total one\hyp{}
particle density matrix of molecule $X$ in AO representation. $V^{{\rm Coul},(0)}$,
which is usually a leading term for singlet-singlet EET (but vanishes for triplet-triplet EET)
can be efficiently evaluated by various multipole expansion models published to date.
$V^{{\rm Exch},(0)}$ is known to be negligible, even for close intermolecular contacts.
However, unlike the direct coupling, the second and third term in Eq.~\eqref{e:v0-fujimoto}, which is collectively
referred to by Fujimoto as the `indirect' coupling, are not negligible at short intermolecular distances.
Unfortunately, evaluation of the indirect coupling is quite expensive and requires evaluation
of the electron repulsion integrals (ERI's).

%First, in Section II we briefly review the Fujimoto's TI/CIS model of EET coupling
%which is treated in this work as a reference.
%Then, further in this Section, we derive the OEP model based on the TI/CIS model (here referred to as the
%which is validated in Section III against the TI/CIS method for a few model complexes. 
Since the TI/CIS model was found very accurate to estimate EET coupling constants for many organic
molecules, we consider this model as a starting point to develop computationally more effcient
scheme that does not involve electron repulsion integrals at all.
First, in Section II we derive the OEP model based on the TI/CIS model (here referred to as the
OEP-TI/CIS model). Next, we validate it against the TI/CIS method for a few model complexes
in Section III. 
Finally, we conclude our work in Section IV.

\section{\label{s:2}OEP-Based Model}


%The associated off\hyp{}diagonal Hamiltonian matrix elements are
%%
%\begin{align}
% \Big< \Phi_1 \Big| \mathscr{H} \Big| \Phi_2 \Big> &\equiv V^{\rm Coul} + V^{\rm Exch} + V^{\rm Ovrl}\\
% \Big< \Phi_1 \Big| \mathscr{H} \Big| \Phi_3 \Big> &\equiv V^{\rm ET1} \\
% \Big< \Phi_2 \Big| \mathscr{H} \Big| \Phi_4 \Big> &\equiv V^{\rm ET2} \\
% \Big< \Phi_1 \Big| \mathscr{H} \Big| \Phi_4 \Big> &\equiv V^{\rm HT1} \\
% \Big< \Phi_2 \Big| \mathscr{H} \Big| \Phi_3 \Big> &\equiv V^{\rm HT2} \\
% \Big< \Phi_3 \Big| \mathscr{H} \Big| \Phi_4 \Big> &\equiv V^{\rm CT } 
%\end{align}
%%
%where the Forster-type Coulombic (Coul), Dexter-type exchange (Exch), remaining overlap correction (Ovrl),
%as well
%as the electron, hole and charge (ET, HT, CT) transfer contributions are defined.


%In the above equatons, the superscript (0) denotes that the matrix elements are not affected by the
%overlap between molecular wavefunctions, and are given by
%%
% V^{{\rm CT },(0)} &= 
%     2 \big( H^A L^B \big| L^A H^B \big) - \big( H^A H^B \big| L^A L^B \Big) 
%\end{align}
%%

To account for the EET coupling via charge transfer states ...

\subsection{\label{s:2.1}Indirect coupling via electron and hole-transfer}

The matrix elements that are needed for the electron transfer (ET) phenomena
is
%
\begin{multline}\label{e:v0-et1}
 V^{{\rm ET1},(0)} \equiv \left< \Phi_1 \vert \mathscr{H} \vert \Phi_3 \right> \approx 
 t_{H\rightarrow L}^A \times \Big\{ 
 \big( \phi_L^A \big| \mathscr{F} \big| \phi_L^B \big) \\
   + 2 \big( \phi_L^A \phi_H^A \big| \phi_H^A \phi_L^B \big) - \big( \phi_L^A \phi_L^B \big| \phi_H^A \phi_H^A \big) 
 \Big\} \;,
\end{multline}
%
along with its twin term, $V^{{\rm ET2},(0)} \equiv \left< \Phi_2 \vert \mathscr{H} \vert \Phi_4 \right>$,
which can be found in the original work of Fujimoto.
In the above, $\mathscr{F}$ is the Fock operator of the entire molecular aggregate.
Here, we approximate the Fock operator by constructing it from the unperturbed
Fock operators of isolated monomers,
%
\begin{equation}
% \mathscr{F} \approx \mathscr{F}^{A}_0 + \mathscr{F}^{B}_0 \;,
 \mathscr{F} \approx \hat{T} + 
   \sum_X^{A,B} \left\{ 
      \hat{V}^{X}_{\rm nuc} + \hat{J}({\bf P}^{X}_g) - \frac{1}{2} \hat{K}({\bf P}^{X}_g)
 \right\} \;,
\end{equation}
%
where $\hat{T}$ is the kinetic energy operator, 
$\hat{V}^{X}_{\rm nuc}$ is the operator associated with the electrostatic potential energy
between nuclei and electrons in molecule $X$, and the Coulomb and exchange operators
are defined by
%
\begin{subequations}
\begin{align}
 \hat{J}({\bf P}^{X}_g) &\equiv \sum_{\sigma\lambda} P^{X(g)}_{\lambda\sigma}
           \iint \frac{\varphi_\lambda({\bf r}_1) \varphi^*_\sigma({\bf r}_1) }{\vert {\bf r}_1 - {\bf r}_2 \vert}
           \square \; d{\bf r}_1 d{\bf r}_2 \;,\\
 \hat{K}({\bf P}^{X}_g) &\equiv \sum_{\sigma\lambda} P^{X(g)}_{\lambda\sigma}
           \iint \frac{\varphi_\lambda({\bf r}_1) \varphi^*_\sigma({\bf r}_2) }{\vert {\bf r}_1 - {\bf r}_2 \vert}
           \square \; d{\bf r}_1 d{\bf r}_2 \;.
\end{align} 
\end{subequations}
%
In the above definitions, the `$\square$' symbol reminds that $\hat{J}$ and $\hat{K}$ are the integral operators
and is given by
%
\begin{equation}
 \square \equiv \iint \Ket{{\bf r}} \Bra{{\bf r}'} \; d{\bf r} d{\bf r}' \;.
\end{equation}
%
Note that the kinetic energy operator cannot be decomposed into particular molecules. However, 
consider the following partitioning of the total Fock operator:
%
\begin{equation}
 \mathscr{F} \approx \mathscr{G}^{A}_0 + \mathscr{G}^{B}_0 \;,
\end{equation}
%
in which the auxiliary operator is defined as
%
\begin{equation}
 \mathscr{G}^{X}_0 = \mathscr{F}^{X}_0 - \frac{1}{2}\hat{T}^{X} \;.
\end{equation}
%
Here, the kinetic energy operator was symmetrically partitioned in between two molecules.
%
Now,
note that, without loss of generality, one can rewrite Eq.~\eqref{e:v0-et1}
as
%
\begin{multline}
 V^{{\rm ET1},(0)} = t_{H\rightarrow L}^A \times \Big\{ 
 \tbraket{\phi_L^A}{\left( \mathscr{G}^A_0 + \mathscr{G}^B_0 \right)}{\phi_L^B} \\
  + 2 \tbraket{\phi_H^A}{\hat{v}_{LH}^A}{\phi_L^B}
  -   \tbraket{\phi_L^A}{\hat{v}_{HH}^A}{\phi_L^B}
 \Big\} \;,
\end{multline}
%
in which the effective potential operator\cite{Blasiak.Bednarska.Choluj.Bartkowiak.XXXX} is defined by
%
\begin{equation}
 \hat{v}_{ij}^X = \int d{\bf r} \Ket{{\bf r}} v_{ij}^X({\bf r})
 \Bra{{\bf r}}
\end{equation}
%
with its spatial form given as
%
\begin{equation}
 v_{ij}^X({\bf r}) \equiv \int d{\bf r}' \frac{\phi_i^*({\bf r}') \phi_j({\bf r}') }{\vert {\bf r} - {\bf r}'\vert} \;.
\end{equation}
%
Note here that $\phi_j({\bf r}) \equiv \BraKet{{\bf r}}{\phi_j}$.
Now, 
by using the OEP technique\cite{Blasiak.Bednarska.Choluj.Bartkowiak.XXXX}
for overlap\hyp{}like effective potential operator matrix elements of the type AB
one can gather operators associated to a particular molecule and group them
into one effective one\hyp{}electron operator
as
%
\begin{equation}
 \bra{\phi_L^A} \left\{ \mathscr{G}_0^A - \hat{v}^A_{HH} \right\}
 + \bra{\phi_H^A} 2 \hat{v}^A_{LH} 
 \cong \sum_{\xi\in A}^{\rm DF} \bra{\xi} V^{A;{\rm ET}}_{\xi;HL}
\end{equation}
%
and
%
\begin{equation}
 \mathscr{G}_0^B \ket{\phi_L^B} \cong \sum_{\eta\in B}^{\rm DF} V^{B;{\rm ET}}_{\eta;L} \ket{\eta} 
\end{equation}
%
In the above transformations, that involve the generalized density fitting
in an auxiliary basis set space,
$V^{A;{\rm ET}}_{\xi;HL}$ and $V^{B;{\rm ET}}_{\eta;L}$
are the OEP matrix elements associated with the above
effective potentials, separately defined for molecule $A$ and $B$. This allows one to
further recast $V^{{\rm ET1},(0)}$ into
%
\begin{multline}\label{e:v0-et1-oep}
 V^{{\rm ET1},(0)} \cong t_{H\rightarrow L}^A \Big\{ 
 \sum_{\xi\in A}^{\rm DF} S_{\xi L}^{AB} V^{A;{\rm ET}}_{\xi;HL} +
 \sum_{\eta\in B}^{\rm DF} S_{\eta L}^{BA} V^{B;{\rm ET}}_{\eta;L}
 \Big\} \;,
\end{multline}
%
which has a particularly simple form as compared to the original expression
in Eq.~\eqref{e:v0-et1} that involves four\hyp{}center electron repulsion integrals.
Note that if 
the OEP matrices are considered
as effective fragment parameters, %pre\hyp{}computed only once.
only overlap integrals between auxiliary basis functions and LUMO orbitals of molecule
$A$ and $B$ are necessary to be computed, i.e.,
%
\begin{equation}
 S^{XY}_{\xi U} = \sum_{\beta\in Y} S_{\xi\beta}^{XY} C_{\beta U}^Y \text{ for $\xi\in X$} \;,
\end{equation}
%
in which 
%
\begin{equation}
 S_{\xi\beta}^{XY} = \int d{\bf r} \varphi_\xi^*({\bf r}) \varphi_\beta({\bf r})
\end{equation}
%
and $C_{\beta U}^X$ is the SCF LCAO-MO matrix for isolated molecule $X$.
It is straightforward to show that the twin term ET2 is
%
\begin{multline}\label{e:v0-et2-oep}
 V^{{\rm ET2},(0)} \cong t_{H\rightarrow L}^B \Big\{ 
 \sum_{\eta\in B}^{\rm DF} S_{\eta L}^{BA} V^{B;{\rm ET}}_{\eta;HL} +
 \sum_{\xi\in A}^{\rm DF} S_{\xi L}^{AB} V^{A;{\rm ET}}_{\xi;L}
 \Big\} \;.
\end{multline}
%
Note that OEP matrices for ET1 and ET2 terms share similar structure.
Therefore a joint superscript `ET' is used here to label the OEP matrices
and additional subscripts
$L$ and $HL$ denote the dependence on only LUMO and on both LUMO and HOMO orbitals, respectively.

Similar considerations apply to the hole transfer (HT) matrix elements.
The expression
%
\begin{multline}\label{e:v0-ht1}
 V^{{\rm HT1},(0)} \equiv \left< \Phi_1 \vert \mathscr{H} \vert \Phi_4 \right> \approx 
 t_{H\rightarrow L}^A \times \Big\{ 
-\big(\phi_H^A \big| \mathscr{F} \big| \phi_H^B \big) \\
   + 2 \big( \phi_H^A \phi_L^A \big| \phi_L^A \phi_H^B \big) - \big( \phi_H^A \phi_H^B \big| \phi_L^A \phi_L^A \big) 
 \Big\} 
\end{multline}
%
can be recast as
%
\begin{multline}\label{e:v0-ht1-oep}
 V^{{\rm HT1},(0)} \cong t_{H\rightarrow L}^A \Big\{ 
 \sum_{\xi\in A}^{\rm DF} S_{\xi H}^{AB} V^{A;{\rm HT}}_{\xi;HL} +
 \sum_{\eta\in B}^{\rm DF} S_{\eta H}^{BA} V^{B;{\rm HT}}_{\eta;H}
 \Big\} \;,
\end{multline}
%
with the effective potential matrices given by
%
%
\begin{subequations}
\begin{align}
 \bra{\phi_H^A} \left\{ -\mathscr{G}_0^A - \hat{v}^A_{LL} \right\}
 + \bra{\phi_L^A} 2 \hat{v}^A_{HL} 
 &\cong \sum_{\xi\in A}^{\rm DF} \bra{\xi} V^{A,{\rm HT}}_{\xi,HL} \\
 -\mathscr{G}_0^B \ket{\phi_H^B} &\cong \sum_{\eta\in B}^{\rm DF} V^{B,{\rm HT}}_{\eta,H} \ket{\eta} 
\end{align}
\end{subequations}
%
The twin term HT2 is accordingly
%
\begin{multline}\label{e:v0-ht2-oep}
 V^{{\rm HT2},(0)} \cong t_{H\rightarrow L}^B \Big\{ 
 \sum_{\eta\in B}^{\rm DF} S_{\eta H}^{BA} V^{B;{\rm HT}}_{\eta;HL} +
 \sum_{\xi\in A}^{\rm DF} S_{\xi H}^{AB} V^{A;{\rm HT}}_{\xi;H}
 \Big\} \;.
\end{multline}
%
As can be seen, to compute Hamiltonian matrix elements
associated with the electron and hole transfer, eight in total
different OEP matrices need to be pre\hyp{}computed and stored in a file.
Note that each of these matrices is actually just a vector of length
equal to the size of the auxiliary basis set chosen. Therefore,
it can be predicted that the computational cost of evaluation of the OEP\hyp{}based Hamiltonian
matrix elements is negligible as compared to the original TI/CIS method.
In Appendix~\ref{a:gdf-formulae}, explicit working formulae
for the OEP matrix elements are given in terms of the AO's.

\subsection{\label{s:2.2}Indirect coupling via charge transfer}

The Hamiltonian matrix element that is associated with the charge transfer process
is
%
\begin{equation}\label{e:v0-ct}
 V^{{\rm CT },(0)} = 
     2 \braket{\phi_H^A \phi_L^B}{\phi_L^A \phi_H^B}
      -\braket{\phi_H^A \phi_H^B}{\phi_L^A \phi_L^B} \;.
\end{equation}
%
Since it involves ERI's of type $\braket{AB}{AB}$ OEP's cannot be directly defined here.
However, Mulliken approximation allows one to express ERI's in terms of the
overlap and Coulomb integrals, i.e.,
%
\begin{equation}
 \braket{ij}{kl} \approx \frac{1}{4} S_{ij}S_{kl}
 \left[ \braket{ii}{kk} + \braket{ii}{ll} + \braket{jj}{kk} + \braket{jj}{ll}\right] \;.
\end{equation}
%
Application of the above approximation to Eq.~\eqref{e:v0-ct} yields
%
\begin{multline}\label{e:v0-ct.mulliken}
 V^{{\rm CT },(0)} \approx \frac{1}{2} S_{HL}^{AB} S_{LH}^{AB} \times \\
  \left[ r_{HL}^A + r_{HL}^B + \braket{\phi_H^A \phi_H^A}{\phi_H^B \phi_H^B} 
                             + \braket{\phi_L^A \phi_L^A}{\phi_L^B \phi_L^B}\right] \\
 -\frac{1}{4} S_{HH}^{AB} S_{LL}^{AB} \times \\
  \left[ r_{HL}^A + r_{HL}^B + \braket{\phi_H^A \phi_H^A}{\phi_L^B \phi_L^B} 
                             + \braket{\phi_L^A \phi_L^A}{\phi_H^B \phi_H^B}\right]  \;,
\end{multline}
%
where 
%
\begin{equation} 
 r_{HL}^X \equiv \braket{\phi_H^X \phi_H^X}{\phi_L^X \phi_L^X} \;.
\end{equation}
%
First of all, note that $r_{HL}^A$ and $r_{HL}^B$ are just constant numbers
associated with the isolated monomers and 
independent on the molecular aggregate. Also, the remaining four ERI's
can be considered as Coulombic interactions between HOMO and LUMO orbitals
of unperturbed molecular wavefunctions.
Then, applying distributed multipole expansion and truncating it on the monopole
one finds that
%
\begin{equation} 
 \braket{\phi_H^A \phi_H^A}{\phi_H^B \phi_H^B} \approx \frac{1}{\vert {\bf r}_H^A - {\bf r}_H^B \vert} \;,
\end{equation}
%
where the charge centroids of HOMO orbitals are given by
%
\begin{equation} 
 {\bf r}_H^X = \tbraket{\phi_H^X}{\hat{\bf r}}{\phi_H^X} \;.
\end{equation}
%
Note that, once occupied molecular orbitals are localized, the above approximation
should yield correct description. LUMO orbitals are usually not much localized.
Here, we use the distributed cumulative atomic charges to represent them, %cite CAMM paper
%
\begin{equation} 
 q^X_{x,L} =-\sum_{\alpha\in x} \sum_\beta S_{\alpha\beta} C_{\alpha L}^X C_{\beta L}^X
\end{equation}
%
and one can find that assuming this one obtains
%
\begin{equation} 
 \braket{\phi_L^A \phi_L^A}{\phi_L^B \phi_L^B} 
 \approx \sum_{x\in A}^{\rm At} \sum_{y\in B}^{\rm At}
 \frac{q_{x;L}^A q_{y;L}^B}{\vert {\bf r}_x - {\bf r}_y \vert} \;.
\end{equation}
%
The two remaining ERI's can be then approximated as follows:
%
\begin{equation} 
 \braket{\phi_H^A \phi_H^A}{\phi_L^B \phi_L^B} 
 \approx -\sum_{y\in B}^{\rm At}
 \frac{q_{y;L}^B}{\vert {\bf r}_H^A - {\bf r}_y \vert} \;,
\end{equation}
%
and similarly for the twin ERI. To summarize the derivation, the CT matrix element
is approximately given by
%
\begin{multline}\label{e:v0-ct.mulliken.working}
 V^{{\rm CT },(0)} \approx \frac{1}{2} S_{HL}^{AB} S_{LH}^{AB} \times 
  \Big[ r_{HL}^A + r_{HL}^B \\ 
 + \frac{1}{\vert {\bf r}_H^A - {\bf r}_H^B \vert} 
 + \sum_{x\in A}^{\rm At} \sum_{y\in B}^{\rm At} 
 \frac{q_{x;L}^A q_{y;L}^B}{\vert {\bf r}_x - {\bf r}_y \vert} \Big] \\
 -\frac{1}{4} S_{HH}^{AB} S_{LL}^{AB} \times 
  \Big[ r_{HL}^A + r_{HL}^B \\
 -\sum_{y\in B}^{\rm At}
 \frac{q_{y;L}^B}{\vert {\bf r}_H^A - {\bf r}_y \vert} 
 -\sum_{x\in A}^{\rm At}
 \frac{q_{x;L}^A}{\vert {\bf r}_H^B - {\bf r}_x \vert} \Big]\;.
\end{multline}
%
Thus, the computational cost of the above expression is negligible as
compared to the original formula in Eq.~\eqref{e:v0-ct}.

\subsection{\label{s:2.3}Approximate pure exchange coupling}

Mulliken approximation can also be used to treat the pure exchange coupling (Eq.).
It is straightforward to show that
%
\begin{multline}\label{e:v0-exch-mulliken}
V^{{\rm Exch},(0)} \approx -\frac{1}{8} \sum_{\mu\nu\in A} \sum_{\lambda\sigma\in B} 
 P_{\nu\mu}^{g\rightarrow e(A)} P_{\lambda\sigma}^{g\rightarrow e(B)} 
 S_{\mu\lambda}  S_{\sigma\nu} \times \\
 \left[ \braket{\mu\mu}{\sigma\sigma} + \braket{\lambda\lambda}{\nu\nu} 
      + \braket{\mu\mu}{\nu\nu} + \braket{\lambda\lambda}{\sigma\sigma}\right] \;.
\end{multline}
%
However, ERI's that involve AO's in two different molecules can be safely neglected.
That leads to the following simplified expression:
%
\begin{multline}\label{e:v0-exch.mulliken.working}
V^{{\rm Exch},(0)} \approx 
  -\frac{1}{8} \Big\{ \sum_{\mu\nu\in A} P_{\nu\mu}^{g\rightarrow e(A)} G_{\mu\nu}^A
    [{\bf s}^{AB} {\bf P}^{g\rightarrow e(B)} {\bf s}^{BA} ]_{\mu\nu} \\
  + \sum_{\sigma\lambda\in B} P_{\lambda\sigma}^{g\rightarrow e(B)} G_{\lambda\sigma}^B
    [{\bf s}^{BA} {\bf P}^{g\rightarrow e(A)} {\bf s}^{AB} ]_{\sigma\lambda} 
  \Big\}
\end{multline}
%
with $G_{\mu\nu} \equiv \braket{\mu\mu}{\nu\nu}$. Note that $G_{\mu\nu}^A$ and $G_{\lambda\sigma}^B$
are just constant matrices that can be considered as effective fragment parameters.

\subsection{\label{s:2.4}Complete model of EET coupling}

To summarize, the overall form of the OEP-TI/CIS model is still given as in the original TI/CIS model
by
%
\begin{equation}
  V \approx V^{\rm Direct} + V^{\rm Inirect} \;,
\end{equation}
%
where the direct and indirect coupling constants are respectively
%
\begin{subequations}
\begin{align}
  V^{\rm Direct  } &\equiv V^{\rm Coul} + V^{\rm Exch} + V^{\rm Ovrl} \;,\\
  V^{\rm Indirect} &\equiv V^{\rm TI-2} + V^{\rm TI-3} \;,
\end{align}
\end{subequations}
%
with
%
\begin{subequations}
\begin{align}
 V^{\rm TI-2} &\equiv-\frac{V^{\rm ET1} V^{\rm HT2}}{E_3-E_1} -\frac{V^{\rm ET2} V^{\rm HT1}}{E_4-E_1}  \;,\\
 V^{\rm TI-3} &\equiv \frac{V^{\rm CT} \left( V^{\rm ET1} V^{\rm ET2} + V^{\rm HT1} V^{\rm HT2}\right) }{(E_3-E_1)(E_4-E_1)} \;.
\end{align}
\end{subequations}
%
In the above, the direct EET coupling contributions are
%
\begin{subequations}\label{e:ovrl-direct}
\begin{align}
 V^{\rm Coul} &= \frac{V^{{\rm Coul},(0)}}{1 - S_{12}^2} \;, \\
 V^{\rm Exch} &= \frac{V^{{\rm Exch},(0)}}{1 - S_{12}^2} \;, \\
 V^{\rm Ovrl} &=-\frac{(E_1+E_2)S_{12}}{2(1-S_{12}^2)}   \;.
\end{align}
\end{subequations}
%
whereas the indirect EET coupling is evaluated from
the overlap\hyp{}corrected ET, HT and CT matrix elements that read
%
\begin{equation}\label{e:ovrl-indirect}
 V^{\rm t} = \left[ 1 - S_{\rm t}^2 \right]^{-1} \left\{ V^{{\rm t},(0)} - \frac{1}{2} (E_1+E_2) S_{\rm t} \right\} \;,
\end{equation}
%
where `t' denotes one of the matrix element types (ET1, ET2, HT1, HT2, CT) 
and $S_{\rm t}$ is the appropriate overlap integral between basis states.

Despite the same general form,
the differences between the original Fujimoto's model and the newly derived OEP\hyp{}based model 
are significant:
%
\begin{enumerate}
 \item The Hamiltonian matrix elements are evaluated from the fragment effective 
parameters, for which only the one\hyp{}electron integrals are needed. Therefore, the computational
cost is significantly reduced as compared to the parent TI/CIS model. In particular, 
$V^{{\rm Coul},(0)}$ is evaluated by using the TrCAMM method developed previously,
$V^{{\rm Exch},(0)}$ and $V^{{\rm CT},(0)}$ are computed from the Mulliken approximated
formulae in Eqs.~\eqref{e:v0-exch.mulliken.working} and \eqref{e:v0-ct.mulliken.working}, respectively,
as well as $V^{\rm ET1}$, $V^{\rm ET2}$,
$V^{{\rm HT1},(0)}$ and $V^{{\rm HT2},(0)}$ are treated by the generalized density fitting
with OEP effective matrices as effective parameters spanned in auxiliary AO basis set 
(Eqs.~\eqref{e:v0-et1-oep}, \eqref{e:v0-et2-oep}, \eqref{e:v0-ht1-oep}, \eqref{e:v0-ht2-oep}
and Eqs.~\eqref{e:a:voep}-\eqref{e:a:aoep}). Note that evaluation
of Eqs.~\eqref{e:ovrl-direct} and \eqref{e:ovrl-indirect} 
is computationally inexpensive because it involves computing only approximate overlap
integrals between basis states;
 \item The relaxation effects due to the intermolecular interactions are not included
in the current OEP\hyp{}based model.
Note that in the original TI/CIS model the basis states and all the one\hyp{}particle density matrices are built from the
self\hyp{}consistently adjusted fragment densities according to the DFI method.
\end{enumerate}
%
%\paragraph*{Overlap corrections to the Hamiltonian.}
%Since in the TI/CIS method the basis states are constructed from the monomer wavefunctions,
%the overlap effects should be properly taken into account. It was reported previously
%that overlap correction to the $\left< \Phi_1 \vert \mathscr{H} \vert \Phi_2 \right>$ 
%matrix element brings negligible contributions to the 
%total EET coupling (c.f. the `overlap' contribution in Ref.). On the other hand,
%the importance of the overlap effects on the other off\hyp{}diagonal Hamiltonian matrix elements were not directly studied. 
%After the overlap correction, the constituents of the direct coupling take the form
%
%Since evaluation
%of Eqs.~\eqref{e:ovrl-direct} and \eqref{e:ovrl-indirect} 
%is computationally inexpensive, we shall consider the overlap\hyp{}corrected
%Hamiltonian in the present work.

\section{\label{s:4}Results and Discussion}

All the models that were used to test the theory presented in this work
were implemented in our in\hyp{}house plugin to {\sc Psi4} quantum chemistry program.\cite{Psi4.JCTC.2017}


\section{\label{s:5}Summary and a few concluding remarks}

Bla.

\begin{acknowledgments}
This project is carried out under POLONEZ programme which has received funding from the European Union's
Horizon~2020 research and innovation programme under the Marie Skłodowska-Curie grant agreement 
No.~665778. This project is funded by National Science Centre, Poland 
(grant~no. 2016/23/P/ST4/01720) within the POLONEZ 3 fellowship.
\end{acknowledgments}

\appendix

\section{Explicit formulae for ET and HT OEP matrix elements\label{a:gdf-formulae}}

As developed in Ref.\cite{Blasiak.Bednarska.Choluj.Bartkowiak.XXXX}, 
the matrix elements of the OEP operators
are computed from the generalized density fitting scheme, according to which
%
\begin{equation}\label{e:a:voep}
 V^X_{\xi, M} = \sum_{\xi'\in X}^{\rm DF} 
                \sum_{\varepsilon\in X}^{\rm RI}
                [{\bf R}^{-1}]_{\xi\xi'} R_{\xi'\varepsilon} H^{X;M}_{\varepsilon; N} \;,
\end{equation}
%
where
%
\begin{equation}
 H^{X;M}_{\varepsilon; N} = \sum_{\varepsilon'\in X}^{\rm RI}
                        [{\bf S}^{-1}]_{\varepsilon\varepsilon'} a_{\varepsilon';N}^{X; M} \;.
\end{equation}
%
In the above equations, the auxiliary matrices are defined by
%
\begin{equation}
 S_{\eta\xi}  = \int 
                       \varphi^*_\eta({\bf r}) \varphi_\xi({\bf r}) 
                 d{\bf r} 
\end{equation}
%
and
%
\begin{equation}
 R_{\eta\xi}  = \iint 
                       \frac{ \varphi^*_\eta({\bf r}_1) \varphi_\xi({\bf r}_2) } 
                            {\vert {\bf r}_1 - {\bf r}_2\vert}  
                 d{\bf r}_1 d{\bf r}_2 \;.
\end{equation}
%
The elements of vector ${\bf a}^{X;M}_N$ for $M=$ ET1, ET2, HT1 and HT2 and $N=$ H, L and HL are given by
%
\begin{subequations}\label{e:a:aoep}
\begin{align}
%
a_{\alpha;L}^{X;{\rm ET}} &= \sum_\beta C_{\beta L}^X G_{\alpha\beta}^X \;,\\
%
a_{\alpha;HL}^{X;{\rm ET}} &= a_{\alpha;L}^{X;{\rm ET}} \nonumber \\
 &+\sum_{\beta\gamma\delta} \braket{\alpha\beta}{\gamma\delta}
 \left\{ 2C^X_{\beta  H} C^X_{\gamma L} - C^X_{\beta  L} C^X_{\gamma H}\right\} C^X_{\delta H} \;,\\
%
a_{\alpha;H}^{X;{\rm HT}} &=-\sum_\beta C_{\beta H}^X G_{\alpha\beta}^X \;,\\
%
a_{\alpha;HL}^{X;{\rm HT}} &=a_{\alpha;H}^{X;{\rm HT}} \nonumber \\
 &+\sum_{\beta\gamma\delta} \braket{\alpha\beta}{\gamma\delta}
 \left\{ 2C^X_{\beta  L} C^X_{\gamma H} - C^X_{\beta  H} C^X_{\gamma L}\right\} C^X_{\delta L} \;.
%
%a_{\alpha}^{A;{\rm ET1}} &= \sum_\beta C_{\beta L}^A F_{\alpha\beta}^A \nonumber \\
% &+\sum_{\beta\gamma\delta} \braket{\alpha\beta}{\gamma\delta}
% \left\{ 2C^A_{\beta  H} C^A_{\gamma L} - C^A_{\beta  L} C^A_{\gamma H}\right\} C^A_{\delta H} \;,\\
%%
%a_{\alpha}^{B;{\rm ET1}} &= \sum_\beta C_{\beta L}^B F_{\alpha\beta}^B \;,\\
%%
%a_{\alpha}^{B;{\rm ET2}} &= \sum_\beta C_{\beta L}^B F_{\alpha\beta}^B \nonumber \\
% &+\sum_{\beta\gamma\delta} \braket{\alpha\beta}{\gamma\delta}
% \left\{ 2C^B_{\beta  H} C^B_{\gamma L} - C^B_{\beta  L} C^B_{\gamma H}\right\} C^B_{\delta H} \;,\\
%%
%a_{\alpha}^{A;{\rm ET2}} &= \sum_\beta C_{\beta L}^A F_{\alpha\beta}^A \;,\\
%%
%a_{\alpha}^{A;{\rm HT1}} &=-\sum_\beta C_{\beta H}^A F_{\alpha\beta}^A \nonumber \\
% &+\sum_{\beta\gamma\delta} \braket{\alpha\beta}{\gamma\delta}
% \left\{ 2C^A_{\beta  L} C^A_{\gamma H} - C^A_{\beta  H} C^A_{\gamma L}\right\} C^A_{\delta L} \;,\\
%%
%a_{\alpha}^{B;{\rm HT1}} &=-\sum_\beta C_{\beta H}^B F_{\alpha\beta}^B \;,\\
%%
%a_{\alpha}^{B;{\rm HT2}} &=-\sum_\beta C_{\beta H}^B F_{\alpha\beta}^B \nonumber \\
% &+\sum_{\beta\gamma\delta} \braket{\alpha\beta}{\gamma\delta}
% \left\{ 2C^B_{\beta  L} C^B_{\gamma H} - C^B_{\beta  H} C^B_{\gamma L}\right\} C^B_{\delta L} \;,\\
%%
%a_{\alpha}^{A;{\rm HT2}} &=-\sum_\beta C_{\beta H}^A F_{\alpha\beta}^A \;.
\end{align}
\end{subequations}
%
Here, the auxiliary one\hyp{}electron ${\mathscr{G}}^X$ operator represented
in space spanned by the intermediate and primary AO basis sets
needs to be computed. The respective contributions to ${\bf a}$ vectors can be
written in a convenient to implement form as
%
\begin{multline}
 \sum_\beta C_{\beta U}^X G_{\alpha\beta}^X = 
 \sum_\beta C_{\beta U}^X \left( \frac{1}{2} T_{\alpha\beta}  + V_{{\rm nuc};\alpha\beta}^X \right)\\
   + \sum_{\beta\gamma\delta} \braket{\alpha\beta}{\gamma\delta}
     \left[ P_{\delta\gamma}^{X(g)} C_{\beta U}^X - \frac{1}{2} P_{\beta\gamma}^{X(g)} C_{\delta U}^X \right] \;,
\end{multline}
%
where
%
\begin{subequations}
\begin{align}
 T_{\alpha\beta} &= -\frac{1}{2} \int \varphi^*_\alpha({\bf r}) \nabla^2_{{\bf r}} \varphi_\beta({\bf r})\; d{\bf r} 
 \text{ and}\\ 
 V_{{\rm nuc};\alpha\beta}^X &= -\sum_{x\in X} Z_x \int 
           \frac{\varphi^*_\alpha({\bf r})  \varphi_\beta({\bf r}) }{\vert {\bf r} - {\bf r}_x \vert} d{\bf r} \;.
\end{align}
\end{subequations}
%
In the above, $Z_x$ is the atomic number of the $x$th atom and $\nabla^2_{{\bf r}}\equiv 
\frac{\partial^2}{\partial x^2} + \frac{\partial^2}{\partial y^2} + \frac{\partial^2}{\partial z^2}$.
% -----------------------
\bibliography{references}
% -----------------------

\end{document}
