%%
%%   Version 3.1 of 16 May 2019.
%%

\usepackage{graphicx}% Include figure files
\usepackage{dcolumn}% Align table columns on decimal point
\usepackage{bm}% bold math

% hyphenation
\usepackage{hyphenat}

% tables
\usepackage{multirow}
\usepackage{booktabs}

% mathematics
\usepackage{amsmath}
\usepackage{amsfonts}
\usepackage{amssymb}
\usepackage{amsbsy}
\usepackage{mathrsfs}
\usepackage{upgreek}
\usepackage[cbgreek]{textgreek}

%---------------------------------------------------
% Approximations
\newcommand{\Approx}[2]{\ensuremath{\text{Ap}_{#2} \left[ {#1} \right] }}
% happy integral
\newcommand{\rint}[1]{\mbox{\Large $ \int\limits_{\mbox{\tiny  $#1$}}$}}
% SHORTCUTS
%\newcolumntype{,}{D{.}{,}{2}}
\newcommand{\citee}[1]{\ensuremath{\scriptsize^{\citenum{#1}}}}
\newcommand{\HRule}{\rule{\linewidth}{0.2mm}}
% Quantum notation
\newcommand{\Bra}[1]{\ensuremath{\bigl\langle {#1} \bigl\lvert}}
\newcommand{\Ket}[1]{\ensuremath{\bigr\rvert {#1} \bigr\rangle}}
\newcommand{\BraKet}[2]{\ensuremath{\bigl\langle {#1} \bigl\lvert {#2} \bigr\rangle}}
\newcommand{\tBraKet}[3]{\ensuremath{\bigl\langle {#1} \bigl\lvert {#2} \bigl\lvert {#3} \bigr\rangle}}
%
\newcommand{\bra}[1]{\ensuremath{\bigl( {#1} \bigl\lvert}}
\newcommand{\ket}[1]{\ensuremath{\bigr\rvert {#1} \bigr)}}
\newcommand{\braket}[2]{\ensuremath{\bigl( {#1} \bigl\lvert {#2} \bigr)}}
\newcommand{\tbraket}[3]{\ensuremath{\bigl( {#1} \bigl\lvert {#2} \bigl\lvert {#3} \bigr)}}
% Math
\newcommand{\pd}{\ensuremath{\partial}}
\newcommand{\DR}{\ensuremath{{\rm d} {\bf r}}}
%\newcommand{\BM}[1]{\ensuremath{\mbox{\boldmath${#1}$}}}
\newcommand{\BM}[1]{\bm{#1}}
% Chemistry (formulas)
\newcommand{\ch}[2]{\ensuremath{\mathrm{#1}_{#2}}}
% Math 
\newcommand{\VEC}[1]{\ensuremath{\mathrm{\mathbf{#1}}}}
% vector nabla
\newcommand{\Nabla}{\ensuremath{ \BM{\nabla}}}
% derivative
\newcommand{\FDer}[3]{\ensuremath{
\bigg(
\frac{\partial #1}{\partial #2}
\bigg)_{#3}}}
% diagonal second derivative
\newcommand{\SDer}[3]{\ensuremath{
\biggl(
\frac{\partial^2 #1}{\partial #2^2}
\biggr)_{#3}}}
% off-diagonal second derivative
\newcommand{\SSDer}[4]{\ensuremath{
\biggl(
\frac{\partial^2 #1}{\partial #2 \partial #3}
\biggr)_{#4}}}
% derivatives without bound
% derivative
\newcommand{\fderiv}[2]{\ensuremath{
\frac{\partial #1}{\partial #2}}}
% diagonal second derivative
\newcommand{\sderiv}[2]{\ensuremath{
\frac{\partial^2 #1}{\partial #2^2}
}}
% off-diagonal second derivative
\newcommand{\sderivd}[3]{\ensuremath{
\frac{\partial^2 #1}{\partial #2 \partial #3}
}}
% derivatives for tables
\newcommand{\fderivm}[2]{\ensuremath{
{\partial #1}/{\partial #2}}}
% diagonal second derivative
\newcommand{\sderivm}[2]{\ensuremath{
{\partial^2 #1}/{\partial #2^2}
}}
% off-diagonal second derivative
\newcommand{\sderivdm}[3]{\ensuremath{
{\partial^2 #1}/{\partial #2 \partial #3}
}}
% ERIs and OEIs
\newcommand{\OEIc}[3]{\ensuremath{\left(#1 \lvert #2 \rvert #3 \right)}}
\newcommand{\ERIc}[4]{\ensuremath{\left(#1 #2 \vert #3 #4 \right)}}

% Partial density and potential
\newcommand{\PartPot}[4]{\ensuremath{\frac{#1 #2}{\lvert #3-#4 \rvert }}}

% trace operator
\DeclareMathOperator{\Tr}{Tr}

%\draft % marks overfull lines with a black rule on the right

% Define location of graphics
\graphicspath{{./figures/}}

\begin{document}
\preprint{AIP/123-OEP}

\title{Ab Initio Effective One-Electron Potentials 
for Repulsive and Charge-Transfer Interaction Energies at Hartree-Fock Level}

\author{Bartosz B{\l}asiak}
\email[]{blasiak.bartosz@gmail.com}
\homepage[]{https://www.polonez.pwr.edu.pl}

\author{Marta Cho{\l}uj} 
\author{Joanna Bednarska}
\author{Wojciech Bartkowiak}

\affiliation{Department of Physical and Quantum Chemistry, Faculty of Chemistry, 
Wroc{\l}aw University of Science and Technology, 
Wybrze{\.z}e Wyspia{\'n}skiego 27, Wroc{\l}aw 50-370, Poland}

\date{\today}

\begin{abstract}
The concept of the effective one-electron potentials has been useful for many decades
in efficient description of electronic structure of chemical systems, especially extended
molecular aggregates such as interacting molecules in condensed phases. 
In this Work, \emph{ab initio} effective potentials are studied for a number of theories
of the intemolecular interaction energy.
\end{abstract}

\pacs{}

\maketitle

\tableofcontents

\section{\label{s:1}Introduction}
\section{\label{s:2}Effective One-Electron Potentials: Configuration Interaction}

The time\hyp{}independent Schr{\"o}dinger equation provides the stationary solution to the
electronic state
of a molecular system in terms of its $N$\hyp{}electron wavefunction, $\Psi$,
%
\begin{equation} \label{e:schrodinger}
 E = \tBraKet{\Psi}{\mathscr{H}}{\Psi} \;,
\end{equation}
%
for a quantum Hamiltonian, $\mathscr{H} = \hat{h}(1) + \hat{o}(2)$,
with $\hat{h}(1)$ being the core Hamiltonian one\hyp{}electron operator
and $\hat{o}(2)$ the two\hyp{}electron repulsion operator.
From the above solution, any property $P$, understood as an expectation value of the
associated operator $\mathscr{P}$, can be computed according to
%
\begin{equation} \label{e:prop}
 P = \tBraKet{\Psi}{\mathscr{P}}{\Psi} \;.
\end{equation}
%
The exact solution to Eq.~\eqref{e:schrodinger} is formally given by
the configuration interaction (CI) expansion around the reference 
(approximate) wavefunction, $\Psi^{(0)}$,
typically the solution to the Hartree\hyp{}Fock (HF) problem,
%
\begin{equation} \label{e:ci}
 \Ket{\Psi} = \Ket{\Psi^{(0)}} + \sum_{ra} C_{r}^{a} \Ket{\Psi_{r}^{a}} + 
	 \sum_{rsab} C_{rs}^{ab} \Ket{\Psi_{rs}^{ab}} + \ldots
\end{equation}
%
In the above equation,
$\Ket{\Psi_{rs\ldots}^{ab\ldots}}$ are called the CI configurations
with the associated CI coefficients $C_{rs\ldots}^{ab\ldots}$, created by `exciting'
one or more electrons from the $a$th ($b$th) occupied molecular orbital to the $a$th ($b$th)
virtual (unoccupied) molecular orbital, obtained from the HF solution.
By inserting Eq.~\eqref{e:ci} into Eq.~\eqref{e:prop} 
and using the Slater\hyp{}Condon rules for evaluating the one\hyp{} and two\hyp{}electron
operator matrix element between
the CI configurations, it can be shown that the following holds
%
\begin{equation} \label{e:exp-val-series}
	P =
	\sum_{ij} P_{ij} \tBraKet{i}{\hat{h}}{j}
	+ \sum_{ijkl} P_{ijkl} \BraKet{ij}{kl}  \;,
\end{equation}
%
where $\tBraKet{i}{\hat{h}}{j} \equiv \int \phi^*_i({\bf r}) \hat{h} \phi_j({\bf r}) d{\bf r} $,
$\BraKet{ij}{kl}$ is the electron\hyp{}repulsion integral (ERI) defined by
%
\begin{equation}
	\BraKet{ij}{kl} \equiv
	\iint 
	\frac{ \phi_i^{*}({\bf r}_1) \phi_j({\bf r}_1) 
	       \phi_k^{*}({\bf r}_2) \phi_l({\bf r}_2) }{ \vert {\bf r}_1 - {\bf r}_2 \vert}
	d{\bf r}_1 d{\bf r}_2
\end{equation}
%
whereas the second\hyp{}rank tensor ${\bf P}^{(2)}$ 
and the fourth\hyp{}rank tensor ${\bf P}^{(4)}$ 
are well defined and characteristic for the property of interest.
%In this work we are interested in cases when

%$\Psi^{AB}$. In this Section we consider theories to calculate the effect of the interaction
%of molecules $A$ and $B$ on the property $P$ knowing only the unperturbed wavefunctions,
%$\Psi^{A}$ and $\Psi^{B}$.

\subsection{\label{ss:2.1}Intermolecular Interaction-Induced Property: Localized Basis Set Approximation}

Within a supermolecular approch, the part of a property that arises due to the
intermolecular interaction between $A$ and $B$ is given by
%
\begin{equation} \label{e:delta-p}
	\Delta^{AB} P = \tBraKet{\Psi^{AB}}{\mathscr{P}}{\Psi^{AB}} - 
	    \tBraKet{\Psi^{A}}{\mathscr{P}}{\Psi^{A}} -
	    \tBraKet{\Psi^{B}}{\mathscr{P}}{\Psi^{B}}\;.
\end{equation}
%
where $\Psi^{AB}$ and $\Psi^{A(B)}$ are the wavefunctions of the interacting molecular aggregate
and the unperturbed (non\hyp{}interacting) molecules, respectively.

Consider an approximation in which the basis functions $\phi_i$ 
are localized on either molecule. 
For this case, application of Eq.~\eqref{e:exp-val-series}
to Eq.~\eqref{e:delta-p} yields the partitioning of the interaction\hyp{}induced property,
for which the one\hyp{}electron part reads
%
\begin{equation} \label{e:delta-p-part-1el}
	\sum_{ij} \Big\{ 
	C^{AA}_{ij}  \BraKet{A}{A} + %\tBraKet{i^A}{\hat{h}}{j^A} + 
	C^{BB}_{ij}  \BraKet{B}{B} + %\tBraKet{i^B}{\hat{h}}{j^B} + 2 
2	C^{AB}_{ij}  \BraKet{A}{B}   %\tBraKet{i^A}{\hat{h}}{j^B} 
	\Big\}
\end{equation}
%
and the two\hyp{}electron part reads
%
\begin{multline} \label{e:delta-p-part-2el}
	\sum_{ijkl} \Big\{  
	C^{AAAA}_{ijkl} \BraKet{AA}{AA} +    %\BraKet{i^Aj^A}{k^Al^A} +
	C^{BBBB}_{ijkl} \BraKet{BB}{BB} + \\ %\BraKet{i^Bj^B}{k^Bl^B} +
	{s} \left[ C^{AABB}_{ijkl} \BraKet{AA}{BB} \right] + 
	{s} \left[ C^{ABAB}_{ijkl} \BraKet{AB}{AB} \right] + \\
	{s} \left[ C^{AAAB}_{ijkl} \BraKet{AA}{AB} \right] +
	{s} \left[ C^{ABBB}_{ijkl} \BraKet{AB}{BB} \right]
	\Big\}
\end{multline}
%
To simplify the resulting expressions
in Eqs.~\eqref{e:delta-p-part-1el} and \eqref{e:delta-p-part-2el},
the shorthand notation $\BraKet{A}{B} \equiv \tBraKet{i^A}{\hat{h}}{j^B}$ and 
$\BraKet{AB}{CD} \equiv \BraKet{i^Aj^B}{k^Cl^D}$
was introduced, and
the real orbitals 
were assumed to simplify the symmetry of the one\hyp{} and two\hyp{}electron integrals.
The operator $s$ permutes and sums over all the combinations of the basis function indices
given their partitioning scheme between molecules $A$ and $B$.

Therefore, the interaction\hyp{}induced property can be recast in a following way


\subsection{\label{ss:2.2}Effective One-Electron Potential}

The one\hyp{}electron potential $v^{\rm eff}({\bf r})$
produced by a certain effective one\hyp{}electron charge density distribution $\rho^{\rm eff}({\bf r})$
can be defined as
%
\begin{equation}
	v^{\rm eff}({\bf r}) = \int \frac{ \rho^{\rm eff}({\bf r}') }{ \vert {\bf r}' - {\bf r} \vert} d{\bf r}'
\end{equation}
%
where ${\bf r}$ is a spatial coordinate. The density can be expanded in terms of an effective
one\hyp{}particle density matrix represented in a certain basis of orbitals, $\BM{\phi}({\bf r})$,
%
\begin{equation}
	\rho^{\rm eff}({\bf r}) = \sum_{\alpha\beta} D_{\alpha\beta}^{\rm eff} 
	\phi_\alpha({\bf r}) \phi_\beta^{*}({\bf r})
\end{equation}
%
Based on that, the operator form of the effective potential 
can be written by
%
\begin{equation}
	\hat{v}^{\rm eff} = \sum_{\alpha\beta} 
	\ket{\alpha} 
	\left[
	\int
	\hat{J}^{\rm}_{\alpha\beta}({\bf r}) 
	\rho^{\rm eff}({\bf r})
	d{\bf r} 
	\right]
	\bra{\beta}
\end{equation}
%
with the Coulomb operator defined by
%
\begin{equation}
	\hat{J}^{\rm}_{\alpha\beta}({\bf r}) f({\bf r}) \equiv
	f({\bf r}) 
	\int
	\frac{ \phi_\alpha^{*}({\bf r}') \phi_\beta({\bf r}') }{ \vert {\bf r}' - {\bf r} \vert} d{\bf r}'
\end{equation}
%
for any one\hyp{}electron function $f({\bf r})$. The matrix element of the effective potential operator
is therefore given by
%
\begin{equation}
	\tbraket{\alpha}{ \hat{v}^{\rm eff} }{\beta}
	= \sum_{\gamma\delta} \braket{\alpha\beta}{\gamma\delta} D^{\rm eff}_{\gamma\delta}
\end{equation}
%
In the above equation and throughout the work, the electron repulsion integral (ERI)
is defined according to
%
\begin{equation}
	\braket{\alpha\beta}{\gamma\delta} \equiv
	\iint 
	\frac{ \phi_\alpha^{*}({\bf r}_1) \phi_\beta({\bf r}_1) 
	       \phi_\gamma^{*}({\bf r}_2) \phi_\delta({\bf r}_2) }{ \vert {\bf r}_1 - {\bf r}_2 \vert}
	d{\bf r}_1 d{\bf r}_2
\end{equation}
%

\subsection{\label{ss:2.3}}

%n brief, the main principle
%is to rewrite the arbirtary function $f$ as\footnote{
%I use Coulomb notation for ERIs, i.e.,
%$\ERIc{\phi_i}{\phi_j}{\phi_k}{\phi_l} \equiv \iint d{\bf r}_1 d{\bf r}_2
%\phi_i^{*}({\bf r}_1)\phi_j({\bf r}_1) \frac{1}{\lvert {\bf r}_1 - {\bf r}_2 \rvert} \phi_k^{*}({\bf r}_2)\phi_l({\bf r}_2)
%$
%}
%
\begin{subequations}
 \begin{align}
 f\left[ 
   \ERIc{\phi_i^A}{\phi_j^A}{\phi_k^B}{\phi_l^B}
 \right] &= \OEIc{\phi_i^A}{v_{kl}^B}{\phi_j^A} \rightarrow \text{ multipole expansion or density fitting}\\
%
 f\left[ 
   \ERIc{\phi_i^A}{\phi_j^B}{\phi_k^B}{\phi_l^B}
 \right] &= \OEIc{\phi_i^A}{v_{kl}^B}{\phi_j^B} \rightarrow \text{ density fitting}
 \end{align}
\end{subequations}
%
where $A$ and $B$ denote different molecules and $\phi_i$ is the $i$th molecular orbital
or basis function.
$v_{kl}^B$ is the effective one-electron potential derived from the partial
effective densities $\rho_{kl}^B({\bf r}) = \phi_k^B({\bf r})\phi_l^B({\bf r})$.
The summations over $k$ and $l$ can be incorporated into the generalized one-electron potential
$v_{\text{eff}}^B$
to produce
%
\begin{equation}
 \sum_{ij}\sum_{kl\in B} f\left[ 
   \ERIc{\phi_i}{\phi_j}{\phi_k^B}{\phi_l^B}
 \right] = \sum_{ij} \OEIc{\phi_i}{v_{\text{eff}}^B}{\phi_j} 
\end{equation}
%
Thus, the total computational effort is extremely reduced from the fourth-fold
sum involving evaluation of ERIs to the two-fold sums of cheaper one-electron integrals.
It is also possible to generalize the above expression even further by
summing over all possible functions $f_t$
%
\begin{equation} \label{e:ft-reduction}
 \sum_t \sum_{ij}\sum_{kl\in B} f_t\left[ 
   \ERIc{\phi_i}{\phi_j}{\phi_k^B}{\phi_l^B}
 \right] = \sum_{ij} \OEIc{\phi_i}{v_{\text{eff}}^B}{\phi_j} 
\end{equation}
%
The above design has the advantage that the effective potentials are fully first-principles
and no extensive case-dependent fitting procedures are necessary.
Only the effective potentials of \emph{independent} fragment $B$ need to be determined once and for all
and stored in a file. Note also, that, in principle, there is no approximation 
made here at that moment.

\section{\label{s:3}Calculation Details}
\section{\label{s:4}Results and Discussion}
\section{\label{s:5}Summary and a few concluding remarks}


\begin{acknowledgments}
This project is carried out under POLONEZ programme which has received funding from the European Union's
Horizon~2020 research and innovation programme under the Marie Skłodowska-Curie grant agreement 
No.~665778. This project is funded by National Science Centre, Poland 
(grant~no. 2016/23/P/ST4/01720) within the POLONEZ 3 fellowship.
\end{acknowledgments}


% -----------------------
\bibliography{references}
% -----------------------

\end{document}
