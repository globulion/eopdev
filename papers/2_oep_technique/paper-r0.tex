%%
%%   Version 3.1 of 16 May 2019.
%%

\usepackage{graphicx}% Include figure files
\usepackage{dcolumn}% Align table columns on decimal point
\usepackage{bm}% bold math

% hyphenation
\usepackage{hyphenat}

% tables
\usepackage{multirow}
\usepackage{booktabs}

% mathematics
\usepackage{amsmath}
\usepackage{amsfonts}
\usepackage{amssymb}
\usepackage{amsbsy}
\usepackage{mathrsfs}
\usepackage{upgreek}
\usepackage[cbgreek]{textgreek}

%---------------------------------------------------
% Approximations
\newcommand{\Approx}[2]{\ensuremath{\text{Ap}_{#2} \left[ {#1} \right] }}
% happy integral
\newcommand{\rint}[1]{\mbox{\Large $ \int\limits_{\mbox{\tiny  $#1$}}$}}
% SHORTCUTS
%\newcolumntype{,}{D{.}{,}{2}}
\newcommand{\citee}[1]{\ensuremath{\scriptsize^{\citenum{#1}}}}
\newcommand{\HRule}{\rule{\linewidth}{0.2mm}}
% Quantum notation
\newcommand{\Bra}[1]{\ensuremath{\bigl\langle {#1} \bigl\lvert}}
\newcommand{\Ket}[1]{\ensuremath{\bigr\rvert {#1} \bigr\rangle}}
\newcommand{\BraKet}[2]{\ensuremath{\bigl\langle {#1} \bigl\lvert {#2} \bigr\rangle}}
\newcommand{\tBraKet}[3]{\ensuremath{\bigl\langle {#1} \bigl\lvert {#2} \bigl\lvert {#3} \bigr\rangle}}
%
\newcommand{\bra}[1]{\ensuremath{\bigl( {#1} \bigl\lvert}}
\newcommand{\ket}[1]{\ensuremath{\bigr\rvert {#1} \bigr)}}
\newcommand{\braket}[2]{\ensuremath{\bigl( {#1} \bigl\lvert {#2} \bigr)}}
\newcommand{\tbraket}[3]{\ensuremath{\bigl( {#1} \bigl\lvert {#2} \bigl\lvert {#3} \bigr)}}
% Math
\newcommand{\pd}{\ensuremath{\partial}}
\newcommand{\DR}{\ensuremath{{\rm d} {\bf r}}}
%\newcommand{\BM}[1]{\ensuremath{\mbox{\boldmath${#1}$}}}
\newcommand{\BM}[1]{\bm{#1}}
% Chemistry (formulas)
\newcommand{\ch}[2]{\ensuremath{\mathrm{#1}_{#2}}}
% Math 
\newcommand{\VEC}[1]{\ensuremath{\mathrm{\mathbf{#1}}}}
% vector nabla
\newcommand{\Nabla}{\ensuremath{ \BM{\nabla}}}
% derivative
\newcommand{\FDer}[3]{\ensuremath{
\bigg(
\frac{\partial #1}{\partial #2}
\bigg)_{#3}}}
% diagonal second derivative
\newcommand{\SDer}[3]{\ensuremath{
\biggl(
\frac{\partial^2 #1}{\partial #2^2}
\biggr)_{#3}}}
% off-diagonal second derivative
\newcommand{\SSDer}[4]{\ensuremath{
\biggl(
\frac{\partial^2 #1}{\partial #2 \partial #3}
\biggr)_{#4}}}
% derivatives without bound
% derivative
\newcommand{\fderiv}[2]{\ensuremath{
\frac{\partial #1}{\partial #2}}}
% diagonal second derivative
\newcommand{\sderiv}[2]{\ensuremath{
\frac{\partial^2 #1}{\partial #2^2}
}}
% off-diagonal second derivative
\newcommand{\sderivd}[3]{\ensuremath{
\frac{\partial^2 #1}{\partial #2 \partial #3}
}}
% derivatives for tables
\newcommand{\fderivm}[2]{\ensuremath{
{\partial #1}/{\partial #2}}}
% diagonal second derivative
\newcommand{\sderivm}[2]{\ensuremath{
{\partial^2 #1}/{\partial #2^2}
}}
% off-diagonal second derivative
\newcommand{\sderivdm}[3]{\ensuremath{
{\partial^2 #1}/{\partial #2 \partial #3}
}}
% ERIs and OEIs
\newcommand{\OEIc}[3]{\ensuremath{\left(#1 \lvert #2 \rvert #3 \right)}}
\newcommand{\ERIc}[4]{\ensuremath{\left(#1 #2 \vert #3 #4 \right)}}

% Partial density and potential
\newcommand{\PartPot}[4]{\ensuremath{\frac{#1 #2}{\lvert #3-#4 \rvert }}}

% trace operator
\DeclareMathOperator{\Tr}{Tr}

%\draft % marks overfull lines with a black rule on the right

% Define location of graphics
\graphicspath{{./figures/}}

\begin{document}
\preprint{AIP/123-OEP}

\title{Ab Initio Effective One-Electron Potentials Revisited:
General Theory and Example Applications for Charge-Transfer Energy Calculations}

\author{Bartosz B{\l}asiak}
\email[]{blasiak.bartosz@gmail.com}
\homepage[]{https://www.polonez.pwr.edu.pl}

\author{Marta Cho{\l}uj} 
\author{Joanna D. Bednarska}
\author{Wojciech Bartkowiak}

\affiliation{Department of Physical and Quantum Chemistry, Faculty of Chemistry, 
Wroc{\l}aw University of Science and Technology, 
Wybrze{\.z}e Wyspia{\'n}skiego 27, Wroc{\l}aw 50-370, Poland}

\date{\today}

\begin{abstract}
The concept of the effective one-electron potentials has been useful for many decades
in efficient description of electronic structure of chemical systems, especially extended
molecular aggregates such as interacting molecules in condensed phases. 
%In this Work, \emph{ab initio} effective potentials are studied for a number of theories
%of the intemolecular interaction energy.
\end{abstract}

\pacs{}

\maketitle

\tableofcontents

\section{\label{s:1}Introduction}

\section{\label{s:2}Effective One-Electron Potentials: General Theory}

The one\hyp{}electron Coulomb static effective potential $v^{\rm eff}({\bf r})$
produced by a certain effective one\hyp{}electron charge density distribution $\rho^{\rm eff}({\bf r})$
is 
%
\begin{equation} \label{e:v-eff}
	v^{\rm eff}({\bf r}) = \int \frac{ \rho^{\rm eff}({\bf r}') }{ \vert {\bf r}' - {\bf r} \vert} d{\bf r}' \;,
\end{equation}
%
where ${\bf r}$ is a spatial coordinate. 
For convenience, the total effective density can be split into nuclear and electronic contributions,
%
\begin{equation} \label{e:rho-eff}
 \rho^{\rm eff}({\bf r}) = \lambda \rho^{\rm eff}_{\rm nuc}({\bf r}) + \rho^{\rm eff}_{\rm el}({\bf r}) \;,
\end{equation}
%
where $\lambda$ is a certain parameter and is assumed either 1 or 0 in this work.
Formally, the electronic part can be expanded in terms of an effective
one\hyp{}particle density matrix represented in a certain basis of orbitals, $\BM{\phi}({\bf r})$,
%
\begin{equation} \label{e:d-spectral}
	\rho^{\rm eff}_{\rm el}({\bf r}) = \sum_{\alpha\beta} D_{\alpha\beta}^{\rm eff} 
	\phi_\alpha({\bf r}) \phi_\beta^{*}({\bf r})  \;.
\end{equation}
%
Based on that, the operator form of the effective potential 
can be written as
%
\begin{equation} \label{e:oep-operator}
	\hat{v}^{\rm eff} = 
        \lambda \hat{v}_{\rm nuc} +
        %\sum_{\alpha\beta} 
	%\Ket{\alpha} 
        \int d{\bf r} \Ket{{\bf r}} 
        v^{\rm eff}({\bf r})
	%\left[
	%\int
	%\hat{J}^{\rm}_{\alpha\beta}({\bf r}') 
	%\rho^{\rm eff}({\bf r}')
	%d{\bf r}' 
	%\right]
	%\Bra{\beta}  
        \Bra{{\bf r}}
\end{equation}
%
with the nuclear
%and Coulomb operators 
operator
defined by
%
%\begin{subequations}
%\begin{align}
\begin{equation}
        \hat{v}_{\rm nuc} \equiv \sum_x^{\rm At}  
                     \int d{\bf r} \Ket{{\bf r}} 
                     \frac{Z_x}{\vert {\bf r} - {\bf r}_x \vert}
                     \Bra{{\bf r}} \;.
	%\hat{J}^{\rm}_{\alpha\beta}({\bf r}) f({\bf r}) &\equiv
	%f({\bf r}) 
	%\int
	%\frac{ \phi_\alpha^{*}({\bf r}') \phi_\beta({\bf r}') }{ \vert {\bf r}' - {\bf r} \vert} d{\bf r}'
\end{equation}
%\end{align}
%\end{subequations}
%
%for any one\hyp{}electron function $f({\bf r})$. 
and $\BraKet{\bf r}{\alpha} = \varphi_\alpha({\bf r})$.
The matrix element of the effective potential operator
is therefore given by
%
\begin{equation}
	\tBraKet{\alpha}{ \hat{v}^{\rm eff} }{\beta}
	= \lambda \sum_x^{\rm At} W_{\alpha\beta}^{(x)} +
        \sum_{\gamma\delta} \BraKet{\alpha\beta}{\gamma\delta} D^{\rm eff}_{\gamma\delta}  \;,
\end{equation}
%
where 
%
\begin{equation}
 W_{\alpha\beta}^{(x)} = 
 Z_x \int \frac{\phi^*_\alpha({\bf r}) \phi_\beta({\bf r})}{\vert {\bf r} - {\bf r}_x \vert} d{\bf r} \;,
\end{equation}
%
$Z_x$ is the atomic number of the $x$th atom,
and the symbol `At' denotes all atoms that contribute to the effective potential.
In the above equation and throughout the work, 
%
\begin{equation}
\tBraKet{\alpha}{\mathscr{O}(1)}{\beta} \equiv \int \phi^*_\alpha({\bf r}) \mathscr{O}(1) \phi_\beta({\bf r}) d{\bf r} 
\end{equation}
%
for any one\hyp{}electron operator $\mathscr{O}(1)$ 
and the electron repulsion integral (ERI)
is defined according to
%
\begin{equation}
	\BraKet{\alpha\beta}{\gamma\delta} \equiv
	\iint 
	\frac{ \phi_\alpha^{*}({\bf r}_1) \phi_\beta({\bf r}_1) 
	       \phi_\gamma^{*}({\bf r}_2) \phi_\delta({\bf r}_2) }{ \vert {\bf r}_1 - {\bf r}_2 \vert}
	d{\bf r}_1 d{\bf r}_2  \;.
\end{equation}
%


\section{\label{s:3}Effective One-Electron Potentials: Configuration Interaction}

The time\hyp{}independent Schr{\"o}dinger equation provides the stationary solution to the
electronic state
of a molecular system in terms of its $N$\hyp{}electron wavefunction, $\Psi$,
%
\begin{equation} \label{e:schrodinger}
 E = \tBraKet{\Psi}{\mathscr{H}}{\Psi} \;,
\end{equation}
%
for a quantum Hamiltonian, $\mathscr{H} = \hat{h}(1) + \hat{o}(2)$,
with $\hat{h}(1)$ being the core Hamiltonian one\hyp{}electron operator,
$\hat{h}(1) = \hat{v}_{\rm nuc} -\frac{1}{2}\sum_e^{\rm el} \nabla_e^2$,
and $\hat{o}(2)$ the two\hyp{}electron repulsion operator.
From the above solution, any property $P$, understood as an expectation value of the
associated operator $\mathscr{P}$, can be computed according to
%
\begin{equation} \label{e:prop}
 P = \tBraKet{\Psi}{\mathscr{P}}{\Psi} \;.
\end{equation}
%
The exact solution to Eq.~\eqref{e:schrodinger} is formally given by
the configuration interaction (CI) expansion around the reference 
(approximate) wavefunction, $\Psi^{(0)}$,
typically the solution to the Hartree\hyp{}Fock (HF) problem\cite{Roothaan.RevModPhys.1951},
%
\begin{equation} \label{e:ci}
 \Ket{\Psi} = \Ket{\Psi^{(0)}} + \sum_{ra} C_{r}^{a} \Ket{\Psi_{r}^{a}} + 
	 \sum_{rsab} C_{rs}^{ab} \Ket{\Psi_{rs}^{ab}} + \ldots
\end{equation}
%
In the above equation,
$\Ket{\Psi_{rs\ldots}^{ab\ldots}}$ are called the CI configurations
with the associated CI coefficients $C_{rs\ldots}^{ab\ldots}$, created by `exciting'
one or more electrons from the $a$th ($b$th) occupied molecular orbital to the $a$th ($b$th)
virtual (unoccupied) molecular orbital, obtained from the HF solution.
By inserting Eq.~\eqref{e:ci} into Eq.~\eqref{e:prop} 
and using the Slater\hyp{}Condon rules for evaluating the one\hyp{} and two\hyp{}electron
operator matrix element between
the CI configurations, it can be shown that the following holds
%
\begin{equation} \label{e:exp-val-series}
	P =
	\sum_{ij} P_{ij} \tBraKet{i}{\hat{h}}{j}
	+ \sum_{ijkl} P_{ijkl} \BraKet{ij}{kl}  \;,
\end{equation}
%
%where $\tBraKet{i}{\hat{h}}{j} \equiv \int \phi^*_i({\bf r}) \hat{h} \phi_j({\bf r}) d{\bf r} $
%$\BraKet{ij}{kl}$ is the electron\hyp{}repulsion integral (ERI) defined by
%%
%\begin{equation}
%	\BraKet{ij}{kl} \equiv
%	\iint 
%	\frac{ \phi_i^{*}({\bf r}_1) \phi_j({\bf r}_1) 
%	       \phi_k^{*}({\bf r}_2) \phi_l({\bf r}_2) }{ \vert {\bf r}_1 - {\bf r}_2 \vert}
%	d{\bf r}_1 d{\bf r}_2  \;,
%\end{equation}
%%
where the second\hyp{}rank tensor ${\bf P}^{(2)}$ 
and the fourth\hyp{}rank tensor ${\bf P}^{(4)}$ 
are well defined and characteristic for the property of interest.
In fact, they are certain functions of the CI coefficients, i.e.,
%
\begin{equation}
{\bf P} = {\bf P}
\left(
 \left\{
  C_{r}^{a}, C_{rs}^{ab}, \ldots 
 \right\} 
\right)
\end{equation}
%
and, in principle, are known as long as the solution to the CI problem 
or its approximation is available.

%In this work we are interested in cases when

%$\Psi^{AB}$. In this Section we consider theories to calculate the effect of the interaction
%of molecules $A$ and $B$ on the property $P$ knowing only the unperturbed wavefunctions,
%$\Psi^{A}$ and $\Psi^{B}$.


\subsection{\label{ss:2.1}Intermolecular Interaction-Induced Property}

Consider a molecular aggregate consisting of two interacting 
molecules $A$ and $B$. The part of property that arises due to the
intermolecular interaction between $A$ and $B$ can be calculated
based on the supermolecular approach,
%
\begin{multline} \label{e:delta-p}
	\Delta^{AB} P = \tBraKet{\Psi^{AB}}{\mathscr{P}}{\Psi^{AB}} \\- 
	\left(
	    \tBraKet{\Psi^{A}}{\mathscr{P}}{\Psi^{A}} +
	    \tBraKet{\Psi^{B}}{\mathscr{P}}{\Psi^{B}}
	\right)
	\;,
\end{multline}
%
where $\Psi^{AB}$ and $\Psi^{A(B)}$ are the wavefunctions of the interacting molecular aggregate
and the unperturbed (non\hyp{}interacting) molecules, respectively.
The notation $\Delta^{AB}$ emphasizes on the two\hyp{}body character of the interaction\hyp{}induced
property. In this work we focus on re\hyp{}expressing $\Delta^{AB}P$ in terms of separate
fragments associated with either molecule. In general case when the basis set
is complete, such an operation is not possible. However,
assuming an approximation
in which the basis functions $\phi_i$ 
are localized on either molecule it allows one to partition the basis set space
and the resulting 2\hyp{} and 4\hyp{}index tensor elements
into subsets `belonging' either to molecule $A$ or to molecule $B$.
Such an approximation is essentially a foundation of any fragment\hyp{}based strategy
and, in particular, designing \emph{ab initio} force fields such as EFP2
of SolEFP.

Application of Eq.~\eqref{e:exp-val-series}
to Eq.~\eqref{e:delta-p} and adopting the localized basis set approximation
yields the partitioning of the interaction\hyp{}induced property,
%
\begin{equation} \label{e:delta-p-part}
 \Delta^{AB} P = \Delta^{AB} P(1) + \Delta^{AB} P(2) \;,
\end{equation}
%
for which the one\hyp{}electron part reads
%
\begin{multline} \label{e:delta-p-part-1el}
 \Delta^{AB} P(1) \approx
	\sum_{ij} \Big\{ 
	P^{AA}_{ij}  \tBraKet{A}{\hat{h}}{A} +    %\tBraKet{i^A}{\hat{h}}{j^A} + 
	P^{BB}_{ij}  \tBraKet{B}{\hat{h}}{B} \\ + %\tBraKet{i^B}{\hat{h}}{j^B} + 2 
2	P^{AB}_{ij}  \tBraKet{A}{\hat{h}}{B}      %\tBraKet{i^A}{\hat{h}}{j^B} 
	\Big\}
\end{multline}
%
and the two\hyp{}electron part is given by
%
\begin{multline} \label{e:delta-p-part-2el}
 \Delta^{AB} P(2) \approx
	\sum_{ijkl} \Big\{ 
	P^{AAAA}_{ijkl} \BraKet{AA}{AA} \\ +    %\BraKet{i^Aj^A}{k^Al^A} +
	P^{BBBB}_{ijkl} \BraKet{BB}{BB} +  %\BraKet{i^Bj^B}{k^Bl^B} +
	{s} \left[ P^{AABB}_{ijkl} \BraKet{AA}{BB} \right] \\ + 
	{s} \left[ P^{ABAB}_{ijkl} \BraKet{AB}{AB} \right] + 
	{s} \left[ P^{AAAB}_{ijkl} \BraKet{AA}{AB} \right] \\ +
	{s} \left[ P^{ABBB}_{ijkl} \BraKet{AB}{BB} \right]
	\Big\} \;.
\end{multline}
%
To simplify the resulting expressions
in Eqs.~\eqref{e:delta-p-part-1el} and \eqref{e:delta-p-part-2el},
the shorthand notation $\tBraKet{A}{\hat{h}}{B} \equiv \tBraKet{i^A}{\hat{h}}{j^B}$ and 
$\BraKet{AB}{CD} \equiv \BraKet{i^Aj^B}{k^Cl^D}$
was introduced, and
the real orbitals 
were assumed to simplify the symmetry of the one\hyp{} and two\hyp{}electron integrals.
The operator $s$ permutes and sums over all the combinations of the basis function indices
given their partitioning scheme between molecules $A$ and $B$.
All the summation components from Eqs.~\eqref{e:delta-p-part-1el} and \eqref{e:delta-p-part-2el} 
can be grouped into the three
sub\hyp{}categories: (i) Coulomb\hyp{}like terms
with integrals of the type $\BraKet{AA}{BB}$, (ii) overlap\hyp{}like terms with $\BraKet{AA}{AB}$
and $\BraKet{AB}{BB}$ and (iii) exchange\hyp{}like terms with $\BraKet{AB}{AB}$.

%Therefore, the interaction\hyp{}induced property can be recast in a following way

\subsection{\label{ss:2.3}Incorporating Electron Repulsion Integrals into Effective Potentials}

Consider now an arbirtary functional $\mathcal{F}$ that explicitly depends on the 
ERI's. In this work, OEP's are defined by the following transformation
%
 \begin{equation}
 \mathcal{F}
 \left[ 
   \BraKet{ij}{k^Al^A}
	 \right] = \tBraKet{i}{\hat{v}_{kl}^A}{j}  \;,
 \end{equation}
%
where 
% $A$ and $B$ denote different molecules and $\phi_i$ is the $i$th molecular orbital
%or basis function.
$\hat{v}_{kl}^A$ is the effective one-electron potential operator given by Eq.~\eqref{e:oep-operator} 
with the
effective density $\rho_{kl}^A({\bf r}) \equiv \phi_k^A({\bf r})\phi_l^A({\bf r})$.
The summations over $k$ and $l$ can be incorporated into the total effective one-electron potential operator
$\hat{v}_{\text{eff}}^A$
to produce
%
\begin{equation}
	\sum_{ij}\sum_{kl\in A} \mathcal{F}\left[ 
   \BraKet{ij}{k^Al^A}
 \right] = \sum_{ij} \tBraKet{i}{\hat{v}_{\text{eff}}^A}{j}  \;.
\end{equation}
%
Thus, the total computational effort is, in principle, reduced from the fourth-fold
sum involving evaluation of ERI's to the two-fold sums of cheaper one-electron integrals.
It is also possible to generalize the above expression even further by
summing over all possible functionals ${\mathcal{F}}_t$
%
\begin{equation} \label{e:ft-reduction}
	\sum_t \sum_{ij}\sum_{kl\in A} {\mathcal{F}}_t\left[ 
   \BraKet{ij}{k^Al^A}
 \right] = \sum_{ij} \tBraKet{i}{\hat{v}_{\text{eff}}^A}{j} \;.
\end{equation}
%
The above design has the advantage that it opens the possibility to define first\hyp{}principles
effective fragments as long as the $P_{ij}$ and $P_{ijkl}$ 
from Eq.~\eqref{e:exp-val-series} are computable and can be approximately
partitioned in between the interacting fragments.
%the resulting effective potentials are fully first\hyp{}principles
%and no extensive case\hyp{}dependent fitting procedures are necessary as long as the $P_{ij}$ and $P_{ijkl}$
%are computable.
%Only the effective potentials of \emph{independent} fragment $A$ need to be determined once and for all
%and stored in a file. 
%Note also, that, in principle, there is no approximation 
%made here at that moment.

\subsection{\label{ss:2.4}Definition of Interaction-Induced Property Effective Potentials}

The above technique can be now applied to the CI expansion of the interaction\hyp{}induced property
under the localized basis set approximation 
from Eq.~\eqref{e:delta-p-part}.
%Coulomb\hyp{}like and overlap\hyp{}like 
%terms from Eq.~\eqref{e:delta-p-part-2el}. 
For example, the Coulomb\hyp{}like contributions
can be rewritten without making any further approximation as
%
\begin{equation}
 \sum_{ijkl} 
	{s} \left[ P^{AABB}_{ijkl} \BraKet{AA}{BB} \right]
 \equiv
	\sum_{kl\in B} 
	\tBraKet{k^B}{\hat{v}^{{\rm eff},A}}{l^B}
\end{equation}
%
with
%
\begin{equation}
 \hat{v}^{{\rm eff},A} 
	\equiv \sum_{ij\in A} {s} \left[ P^{AABB}_{ijkl} \hat{v}^{AA}_{ij} \right]  \;.
\end{equation}
%
Similarly for the overlap\hyp{}like term we have
%
\begin{equation}
 \sum_{ijkl} 
	{s} \left[ P^{AAAB}_{ijkl} \BraKet{AA}{AB} \right]
 \equiv
	\sum_{k\in A} \sum_{l\in B} 
	\tBraKet{k^A}{\hat{v}^{{\rm eff},A}}{l^B}
\end{equation}
%
and the associated effective potential reads
%
\begin{equation}
 \hat{v}^{{\rm eff},A} 
	\equiv \sum_{ij\in A} {s} \left[ P^{AAAB}_{ijkl} \hat{v}^{AA}_{ij} \right] \;.
\end{equation}
%
Note that the exchange\hyp{}like terms cannot be represented in terms of OEP's.
Gathering the above arguments, $\Delta^{AB}P$ adopts the following form
%
\begin{multline} \label{e:ci-oep-master}
 \Delta^{AB}P \approx \Delta_{\rm ex}^{AB} P
	+ \sum_{i\in A} \sum_{j\in B} 
	  \tBraKet{i}{ \left\{ \hat{v}^{{\rm eff},AB}_{1(ij)} + \hat{v}^{{\rm eff},BA}_{1(ij)}  \right\} }{j}
	 \\ +
	\sum_{i\in A} \sum_{k\in A}
	  \tBraKet{i}{\hat{v}^{{\rm eff},BA}_{2(ik)} }{k}
	 +
        \sum_{j\in B} \sum_{l\in B}
          \tBraKet{j}{\hat{v}^{{\rm eff},AB}_{2(jl)} }{l}
\end{multline}
%
where the exchange\hyp{}like component is
%
\begin{equation}
	\Delta_{\rm ex}^{AB} P = 
	s \sum_{ik\in A} \sum_{jl\in B}
	P^{ABAB}_{ijkl} \BraKet{ij}{kl} \;,
\end{equation}
%
and
the overlap\hyp{}like and the Coulomb\hyp{}like 
OEP's are defined as follows
%
\begin{subequations}
\begin{align} \label{e:oep-exch-like}
	\hat{v}^{{\rm eff},BA}_{1(ij)} &\equiv P_{ij}^{AB} \hat{h}
	+ s \sum_{mk\in A}  P^{AAAB}_{mkij} \hat{v}_{mk}^{AA} \;, \\
	%
	\hat{v}^{{\rm eff},BA}_{2(ik)} &\equiv P_{ik}^{AA} \hat{h} 
        + s \sum_{mk'\in A} P^{AAAA}_{ikmk'} \hat{v}_{mk'}^{AA}  \nonumber
        \\ \label{e:oep-coul-like}
	& \qquad\qquad+ \frac{1}{2}
        s \sum_{jl'\in B}  P^{AABB}_{ik'jl} \hat{v}_{jl'}^{BB} \;,
\end{align}
\end{subequations}
%
and accordingly for the twin operators $\hat{v}^{{\rm eff},AB}_{\text{$1$ and $2$}}$.

\subsection{\label{s.242}Separability of P-Coefficients}

Until that point, the effective potentials from Eqs.~\eqref{e:oep-exch-like}
and \eqref{e:oep-coul-like} are not separable in terms of the interacting molecules
because the $\bf P$ tensors are derived from the CI wavefunction of
the entire interacting system $AB$. However, if there exist an approximate
separation as, for example,
%
\begin{align}
 P^{AAAB}_{mkij} &\approx X^A_{mk} Y^{AB}_{ij} \;, \\
 P^{AABB}_{ikjl} &\approx Z^A_{ik} Y^{AB}_{} Z^{ B}_{jl} \;,
\end{align}
%
and if the matrices ${\bf X}^{A(B)}$, 
${\bf Z}^{A(B)}$ and ${\bf P}^{AA(BB)}$ as well as the tensors
${\bf P}^{AAAA(BBBB)}$ 
are properties of the unperturbed (isolated) molecules,
further rearrangements in Eq.~\eqref{e:ci-oep-master} are possible
leading to computationally more efficient expressions.

Note also that the matrices ${\bf Y}^{AB}$ and ${\bf P}^{AB}$ depend on their instantaneous 
relative positions.

%then the molecule\hyp{}specific OEP's can be defined.

\section{\label{s.334}Practical OEP-Based Calculations}

There are two general cases of basis function partitioning scheme
in which it is possible to define an effective one\hyp{}electron
potential. They are listed in Table~\ref{t:oep-matrix-element-types}.
The first type is a Coulomb\hyp{}like contribution, for which multipole expansion or density fitting can be used.
The second type is an overlap\hyp{}like contribution, which can be approximated via the density fitting. 
%
{
\renewcommand{\arraystretch}{1.4}
\begin{table}[b]
\caption[Types of matrix elements with OEP operators]
{{\bf Types of matrix elements with OEP operators\footnotemark[1]}
}
\label{t:oep-matrix-element-types}
\begin{ruledtabular}
\begin{tabular}{lcccc}
Matrix element      &&            `Overlap-like'                &&            `Coulomb-like'               \\ 
                    && $\tBraKet{i}{\hat{v}^{A}_{\rm eff}}{j} $ && $\tBraKet{j}{\hat{v}^{A}_{\rm eff}}{l}$ \\ 
	\cline{1-5}
Partitioning scheme &&            $i\in A, j\in B$              &&               $j,l\in B$                \\
Associated ERI's    &&            $\BraKet{AA}{AB}$             &&               $\BraKet{AA}{BB}$         \\
DF\footnotemark[2]/RI\footnotemark[3] Form    
&& $\sum_{\xi\in A}^{\rm Aux} v^A_{i\xi} S^{AB}_{\xi j} $  
&& $\sum_{\xi\zeta\in A}^{\rm Aux} S^{BA}_{j\xi} v^A_{\xi\zeta} S^{AB}_{\zeta l} $ \\
DMTP\footnotemark[4] Form                     
&& --  &  &  $\rho_{jl}^B \odot \rho_{\rm eff}^A$ \\
\end{tabular}
\end{ruledtabular}
%
\footnotetext[1]{Test footnotemark.}
\footnotetext[2]{Density Fitting}
\footnotetext[3]{Resolution of Identity}
\footnotetext[4]{Distributed Multipole Expansion}
%
\end{table}
}
%

\subsection{\label{s.333}Distributed Multipole Expansion of OEP's}

This scheme is most applicable in the case of matrix elements of the type
$
 \tBraKet{j^B}{\hat{v}_{\rm eff}^A}{l^B}
$
because it can be considered as a Coulombic interaction between $\rho^B_{jl}$
and $\rho^A_{\rm eff}$. 
In general, given certain two effective one\hyp{}electron density distributions,
the associated effective Coulombic interaction energy can be estimated from the classical formula
according to
%
\begin{equation}
 \Delta^{(XY)}_{\rm eff} E_{\rm Coul} = \iint \frac{\rho_{\rm eff}^X({\bf r}_1) \rho_{\rm eff}^Y({\bf r}_2)}
 {\vert {\bf r}_1 - {\bf r}_2 \vert} 
d{\bf r}_1 d{\bf r}_2 \;.
\end{equation}
%
The above double integral can be approximated by the
distributed multipole (DMTP)
expansions of the respective densities as follows
%
\begin{multline}
  \Delta^{(XY)}_{\rm eff} E_{\rm Coul} \approx
 \rho_{\rm eff}^X \odot \rho_{\rm eff}^Y \equiv
 \sum_{u\in A} \sum_{w\in B} \Big\{ 
 T^{(0)}_{uw}
 q_{\rm eff}^{(u)}  q_{\rm eff}^{(w)} \\
 - {\bf T}^{(1)}_{uw} \cdot 
   \left[ q_{\rm eff}^{(u)} {\BM\upmu}_{\rm eff}^{(w)} - q_{\rm eff}^{(w)} {\BM\upmu}_{\rm eff}^{(u)} \right]
 - {\bf T}^{(2)}_{uw} : 
  {\BM\upmu}_{\rm eff}^{(u)}  \otimes {\BM\upmu}_{\rm eff}^{(w)} 
 \ldots
 \Big\}
\end{multline}
%
In the above equations, $q_{\rm eff}^{(u)}$ and ${\BM\upmu}^{(u)}_{\rm eff}$ 
are the effective distributed monopole (charge)
and dipole moment, respectively, centered on the $u$th site, 
whereas ${{\bf T}^{(r)}_{uw}}$ are the so called interaction tensors or rank $r$ 
that can be found elsewhere.
The symbol `$\odot$' denotes the sum of all the tensor contractions
performed over the DMTP's of molecule $X$ and $Y$ to yield the associated interaction energy.
The choice of the distribution centres as well as the truncation order of the multipole expansion
is crucial in compromising the accuracy and computational cost of the resulting expressions.
There is many ways in which this can be achieved, e.g., through the distributed multipole analysis (DMA)
of Stone and Anderton, %cite
the cumulative atomic multipole moments of Sokalski and Poirier, %cite
the localised distributed multipole expansion (LMTP) of Etchtebest et al. %cite
or schemes based on fitting to electrostatic potential.

\subsection{\label{s.3344}Density Fitting of OEP's}

This scheme is most useful in the case of matrix elements of the type
$
 \tBraKet{i^A}{\hat{v}_{\rm eff}^A}{j^B}
$.
To get the \emph{ab initio} representation of such an overlap\hyp{}like matrix element,
one can use a procedure similar to
the typical density fitting or resolution of identity, both of which are nowadays widely used 
to compute electron\hyp{}repulsion integrals (ERI's) more efficiently. 

\subsubsection{Density Fitting in Nearly-Complete Space}

An arbitrary one\hyp{}electron potential of molecule $A$ acting on any state vector 
associated with molecule $A$ can be expanded in an auxiliary space centered 
on $A$ as
%
\begin{equation}
   \hat{v}^{A}\Ket{i} = \sum_{\xi\eta}^{\rm RI} \hat{v}^{A}\Ket{\xi} [{\bf S}^{-1}]_{\xi\eta} \BraKet{\eta}{i}
\end{equation}
%
under the necessary assumption that the auxiliary basis set is nearly complete,
i.e., 
$\sum_{\xi\eta}^{\rm RI} \Ket{\xi}[{\bf S}^{-1}]_{\xi\eta} \Bra{\eta} \cong 1$.
In the equations above, 
the `RI' symbol denotes a certain auxiliary basis set that fulfills the resolution of identity. 
%In a special case when the basis set is orthogonal (e.g., molecular orbitals)
%the above relation simplifies to
%\begin{equation}
%   v\vert i) = \sum_{\xi} v\vert \xi) ( \xi \vert i)
%\end{equation}
Such a general expansion can be obtained by 
utilizing the density fitting
in the nearly\hyp{}complete space,
%
\begin{equation}
 \hat{v}^{A}\Ket{i} = \sum_{\xi}^{\rm RI} V^A_{i\xi} \Ket{\xi} \;.
\end{equation}
%
In the above equation,
the matrix ${\bf V}^A$
is the projection of the state vector $\hat{v}^{A} \Ket{i}$
onto the complete basis $\{ \varphi_\xi \}$.
Let $Z_i[{\bf V}^A]$ be the least\hyp{}squares objective function 
%
\begin{equation} \label{e:z-compl}
 Z_i[{\bf V}^A] = \int \vert \Xi_i({\bf r}) \vert^2 d{\bf r}
\end{equation}
%
with the error density defined by
%
\begin{equation}
 \Xi_i({\bf r}) = v^A({\bf r}) \phi_i({\bf r}) - \sum_\xi^{\rm RI} V^A_{i\xi} \varphi_\xi({\bf r}) \;.
\end{equation}
%
By requiring that
%
\begin{equation} \label{e:z-necessary-requirement}
 \frac{\partial Z_i[{\bf V}^A]}{\partial V^A_{i\mu}} = 0 \text{ for all $\mu$}
\end{equation}
%
one finds the coefficients of the $i$th row of ${\bf V}^A$ to be
%
\begin{equation} \label{e:gdf-coml.v}
% {\bf v}_i = {\bf a}^{(i)} \cdot {\bf S}^{-1}
  V^A_{i\xi} = \sum_\eta^{\rm RI} [{\bf S}^{-1}]_{\xi\eta} a^{(i)}_\eta \;,
\end{equation}
%
where the auxiliary matrices are given by
%
\begin{subequations}
\begin{align}
 a^{(i)}_\eta &= \int \varphi^*_\eta({\bf r}) \hat{v}^A \phi_i({\bf r}) d{\bf r} \;,\\  
 S_{\eta\xi}  &= \int \varphi^*_\eta({\bf r}) \varphi_\xi({\bf r}) d{\bf r} \;.
\end{align}
\end{subequations}
%
%Since matrix elements of an OEP operator in auxiliary space can be computed 
%in the same way as the matrix elements with any other basis function, 
%one can formally write the following identity
%\begin{equation}
% (X \vert v\vert i) = \sum_{\xi\eta} S_{X \xi} [{\bf S}^{-1}]_{\xi\eta} (\eta\vert v\vert i)
%\end{equation}
%where $ X $ is an arbitrary orbital.
%When the other orbital
%does not belong to molecule $A$ but to the (changing) environment, it is 
%straightforward to compute the resulting matrix element, which is simply given as  
%\begin{equation}
%   (j_{\in B} \vert v^A \vert i_{\in A}) = \sum_\xi {S_{j\xi}} {G_{i\xi}}
%\end{equation}
%where $j$ denotes the other (environmental) basis function.
%
The working formula for $a^{(i)}_\eta$ can be found by applying 
potential form from Eq.~\eqref{e:v-eff}
along with the spectral representation of the effective density from Eq.~\eqref{e:d-spectral} 
which finally gives
%
\begin{equation}
 a^{(i)}_\eta = \lambda \sum_{x} W_{\eta i}^{(x)} + 
 \sum_{\alpha\beta} D^A_{\alpha\beta} 
  \BraKet{\alpha\beta}{\eta i} \;.
%  \iint \frac{ \phi_\alpha^*({\bf r}_1)  \phi_\beta({\bf r}_1) \varphi^*_\eta({\bf r}_2) \phi_i({\bf r}_2)}
% {\vert {\bf r}_1 - {\bf r}_2 \vert }
% d{\bf r}_1 d{\bf r}_2
\end{equation}
%

\subsubsection{Density Fitting in Incomplete Space}

Density fitting scheme from previous section has practical disadvantage of a nearly\hyp{}complete basis set
being usually very large (spanned by large amount of basis set vectors). 
Since most of basis sets used in quantum chemistry do not form a nealy complete
set, it is beneficial to design a modified scheme in which it is possible to obtain the effective 
matrix elements of the OEP operator in a incomplete auxiliary space. This can be achieved by minimizing 
the following objective function
%
\begin{equation} \label{e:z-incompl}
	Z_i[{\bf V}^A] = \iint 
        \frac{ \Xi_i^*({\bf r}_1) \Xi_i({\bf r}_2) }{\vert {\bf r}_1 - {\bf r}_2 \vert}  
         d{\bf r}_1 d{\bf r}_2  \;.
\end{equation}
%
with the error density defined by
%
\begin{equation}
 \Xi_i({\bf r}) = v^A({\bf r}) \phi_i({\bf r}) - \sum_\xi^{\rm DF} V^A_{i\xi} \varphi_\xi({\bf r}) \;.
\end{equation}
%
where the symbol `DF' denotes the generally incomplete auxiliary basis set.
From the requirement given in Eq.~\eqref{e:z-necessary-requirement}
one obtains
%
\begin{equation} \label{e:v-noncompl}
% {\bf G}^{(i)} = {\bf b}^{(i)} \cdot {\bf A}^{-1}
  V^A_{i\xi} = \sum_\eta^{\rm DF} [{\bf R}^{-1}]_{\xi\eta} b^{(i)}_\eta \;,
\end{equation}
%
where 
%
\begin{subequations}
\begin{align}
 b^{(i)}_\eta &= \iint 
                       \frac{ \varphi^*_\eta({\bf r}_1) \hat{v} \phi_i({\bf r}_2) } 
                            {\vert {\bf r}_1 - {\bf r}_2\vert}  
                 d{\bf r}_1 d{\bf r}_2 \;, \\
 R_{\eta\xi}  &= \iint 
                       \frac{ \varphi^*_\eta({\bf r}_1) \varphi_\xi({\bf r}_2) } 
                            {\vert {\bf r}_1 - {\bf r}_2\vert}  
                 d{\bf r}_1 d{\bf r}_2 \;.
\end{align}
\end{subequations}
%
%The symbol $ \vert\vert $ is to denote the operator $ r_{12}^{-1}$ and double integration over $ {\bf r}_1 $
%and $ {\bf r}_2 $. 
Note that, while $R_{\eta\xi}$ is a typical 2\hyp{}center ERI 
that can be evaluated by standard means,
$b^{(i)}_\eta$ matrix elements are not at all trivial to calculate
because the OEP operator, which contains integration over an electron coordinate,
is present inside the double integral. Therefore, the following triple integral,
%
\begin{equation} \label{e:triple-integral}
 b^{(i)}_\eta = \iiint 
           \frac{ \varphi^*_\eta({\bf r}_1) \phi_i({\bf r}_2)  \rho^{\rm eff}({\bf r}_3) }
            {\vert {\bf r}_1 - {\bf r}_2 \vert \vert {\bf r}_3 - {\bf r}_2 \vert}
           d{\bf r}_1 d{\bf r}_2 d{\bf r}_3 \;,
\end{equation}
%
has to be computed.
Obtaining all the necessery integrals of this kind directly 
by performing integrations in Eq.~\eqref{e:triple-integral} is very costly 
and impractical even for medium sized molecules. 
However, one can introduce the effective potential in order to eliminate one integration. 
This can be achieved by performing additional density fitting in a nearly complete intermediate basis 
%
\begin{equation}
 \hat{v}^{\rm eff} \Ket{i} = \sum_\varepsilon^{\rm RI} H_{i\varepsilon} \Ket{\varepsilon} \;,
\end{equation}
%
%where the symbol `RI' denotes the intermediate basis from a chosen resolution of identity.
The working equation is therefore given by
%
\begin{equation}
 b^{(i)}_\eta = \sum_\varepsilon^{\rm RI} H_{i\varepsilon} R_{\varepsilon\eta} \;,
\end{equation}
%
%with
%%
%\begin{equation}
% R_{\varepsilon\eta} \equiv \iint 
%                       \frac{ \varphi^*_\varepsilon({\bf r}_1) \varphi_\eta({\bf r}_2) } 
%                            {\vert {\bf r}_1 - {\bf r}_2\vert}  
%                 d{\bf r}_1 d{\bf r}_2 \;,
%\end{equation}
%%
which can be easily evaluated by noting that
$H_{i\varepsilon}$ are given by Eq.~\eqref{e:gdf-coml.v}.
Thus, the matrix elements of the OEP operator can be found
by using Eq.~\eqref{e:v-noncompl}.
%Therefore, in order to use this generalized density fitting scheme
%one must to compute two\hyp{}centre electron repulsion integrals
%as well as four\hyp{}centre asymmetric electron repulsion integrals of the type $ (\alpha\beta\gamma||\eta) $.
%The evaluation of such integrals is discussed in Appendix~\ref{a:mcmurchie-davidson}.

It is emphasized here that, as long as the state vector $\Ket{i}$, OEP operator $\hat{v}^A$ 
and the auxiliary and intermediate basis sets depend solely on one unperturbed molecule $A$, the matrix elements
$V^A_{i\xi}$ can be calculated just once and stored in a file as effective fragment parameters.


\section{\label{s:4}Calculation Details}

All the models that were used to test the theory presented in this work
were implemented in our in\hyp{}house plugin to {\sc Psi4} quantum chemistry program.\cite{Psi4.JCTC.2017}

\section{\label{s:5}Results and Discussion}
\subsection{\label{ss.5.1}Pauli-Repulsion Interaction Energy}

At short intermolecular interactions, wavefunction overlap increases
which causes the electrons from either molecules to move apart to minimize the repulsive interactions.
The resulting interaction energy can be estimated based on perturbation theory approaches.
In this section, we demonstrate the use of OEP operators to significanlty reduce the computational
cost of such calculations, that can be used in fragment\hyp{}based modelling of extended molecular
aggregates in terms of force field methods.

As a starting point we consider perturbation theory model of Murrell et al.,
in which repulsive energy is originally given by
%
\begin{equation}
 \Delta^{AB} E^{\rm Rep} = 
 \Delta^{AB} E^{\rm Rep}(S^{-1}) + 
 \Delta^{AB} E^{\rm Rep}(S^{-2}) \;,
\end{equation}
%
where the first\hyp{}order term with respect to the intermolecular overlap is
%
\begin{multline} \label{e:rep.murrell-etal.S1}
  \Delta^{AB}  E^{\rm Rep}(S^{-1}) = -2\sum_{i\in A}^{\rm Occ} \sum_{j\in B}^{\rm Occ}
               S_{ij} \Big\{
           V^A_{ij} 
         + V^B_{ji} \\
 + \sum_{k\in A}^{\rm Occ} \left[ 2\BraKet{ij}{kk} - \BraKet{ik}{jk} \right] 
 + \sum_{l\in B}^{\rm Occ} \left[ 2\BraKet{ij}{ll} - \BraKet{il}{jl} \right]
                \Big\}
\end{multline}
%
and the second\hyp{}order term is given in Eq.(XXX) in Ref. %cite
In this work, we focus to simplify only the former term because this is usually
the bottleneck of evaluating the Pauli repulsive contributions to
forces in molecular dynamics when using the effective potential method, whereas
very inexpensive approximation to the latter term has already been worked out previously. %cite
Note that the Coulomb and exchange integrals can be re\hyp{}cast as follows:
%
\begin{subequations}
 \begin{align}
 \BraKet{ij}{kk} &\equiv \sum_{\mu\in A} 
     C_{\mu i}^A \tBraKet{\mu}{\hat{v}_{kk}}{j} \;, \\
 \BraKet{ik}{jk} &\equiv \sum_{\mu\in A} 
     C_{\mu k}^A \tBraKet{\mu}{\hat{v}_{ik}}{j} \;.
 \end{align}
\end{subequations}
%
%Similarly, the exchange integrals reduce to
%
%\begin{equation}
% \BraKet{ik}{jk} \equiv \sum_{\mu\in A} 
%     C_{\mu k}^A \tBraKet{\mu}{\hat{v}_{ik}}{j} \;.
%\end{equation}
%
In the above equations, the auxiliary potential operators are given by
%
\begin{equation}
  \hat{v}_{ik} \equiv \int d{\bf r} \Ket{{\bf r}} 
        \left[
        \int d{\bf r}' \frac{\phi_i^{*}({\bf r}') \phi_k({\bf r}')}{\vert {\bf r}' - {\bf r}\vert}
        \right] \Bra{{\bf r}} \;.
\end{equation}
%
Now, using the prescription in Eq.~\eqref{e:ft-reduction} one can define a joint OEP
operator from nuclear, Coulomb and exchange parts as
%
\begin{equation}
 V^A_{ij} + 
 \sum_{k\in A}^{\rm Occ} 
  \left\{ 2\BraKet{ij}{kk} - \BraKet{ik}{jk} \right\}
\equiv \sum_{\mu\in A} \tBraKet{\mu}{ 
\hat{v}^{A[\mu i]}_{\rm eff}
 }{j}
\end{equation}
%
with
%
\begin{equation}
 \hat{v}^{A[\mu i]}_{\rm eff} \equiv C_{\mu i} \hat{v}^A_{\rm nuc} + 
 \sum_{k\in A}^{\rm Occ} \left[
 2C_{\mu i}^A \hat{v}^A_{kk} - C_{\mu k}^A \hat{v}^A_{ik}
 \right] \;.
\end{equation}
%
%The nuclear part can also be incorporated within the effective potential
%resulting in the following reduction
%
%
On the other hand, it immediately follows that
%
\begin{equation}
 \hat{v}^{A[\mu i]}_{\rm eff} \Ket{\mu} = 
  \sum_{k\in A}^{\rm Occ} \left\{
     2\hat{v}^A_{kk} \Ket{i} - \hat{v}^A_{ik} \Ket{k}
  \right\} \;.
\end{equation}
%
For practical calculations, the right hand side of the above equation can be expanded 
in the auxiliary basis,
%
\begin{equation} \label{e:v-oep.rep}
  \sum_{k\in A}^{\rm Occ} \left\{
     2\hat{v}^A_{kk} \Ket{i} - \hat{v}^A_{ik} \Ket{k}
  \right\} \cong
  \sum_{\xi\in A}^{\rm DF} 
  V_{\xi i}^A \Ket{\xi} \;,
\end{equation}
%
where the matrix ${\bf V}^A$ can be considered as effective fragment parameters.
Doing the same operations on the twin operators associated with the molecule $B$
original theory of Murrell et. al from Eq.~\eqref{e:rep.murrell-etal.S1}
reduces to
%
\begin{equation}
  \Delta^{AB}  E^{\rm Rep}(S^{-1}) \cong 
 -2\sum_{i\in A}^{\rm Occ} \sum_{j\in B}^{\rm Occ}
               S_{ij} \Big\{
           \sum_{\xi \in A}^{\rm DF} V_{\xi i}^A S_{\xi j}
         + \sum_{\eta\in B}^{\rm DF} V_{\eta j}^B S_{\eta i}
                \Big\}
\end{equation}
%
which is a fully OEP\hyp{}based result. Note, that now only 
overlap integrals, which are relatively computationally very inexpensive,
need to be evaluated.

As another benchmark model for accuracy, we compare our results with 
the exchange and repulsion energy term 
from the density\hyp{}based scheme of partitioning the interaction energy
(here abbreviated as DDS),
that was developed by Mandado and Hermida-Ramon. 
%In the above equations, the Pauli deformation density 
%%
%\begin{equation}
%   \Delta {\bf D}^{\rm Pauli} \equiv {\bf D}^{oo} - {\bf D}
%\end{equation}
%%
%where \f$ {\bf D}^{oo}\f$ and \f$ {\bf D}\f$ are the density matrix formed from
%mutually orthogonal sets of molecular orbitals within the entire aggregate (formed
%by symmetric orthogonalization of MO's) and the density matrix of the unperturbed
%system (that can be understood as a Hadamard sum \f$ {\bf D} \equiv {\bf D}^A \oplus {\bf D}^B\f$).
% 
%At HF level, the Pauli deformation density matrix is given by
%\f[
%  \Delta {\bf D}^{\rm Pauli} = {\bf C} \left[ {\bf S}^{-1} - {\bf 1} \right] {\bf C}^\dagger
%\f]
%whereas the density matrix constructed from mutually orthogonal orbitals is
%\f[
%   {\bf D}^{oo} = {\bf C} {\bf S}^{-1} {\bf C}^\dagger
%\f]
%In the above equations, \f$ {\bf S} \f$ is the overlap matrix between doubly occupied molecular orbitals
% of the entire aggregate.
It is also mentioned here that the DDS model of exchange-repulsion at HF level of theory
is equivalent to the perturbative method of Hayes and Stone and the two methods
yield the same results, although the exchange and repulsive interaction energies
are slightly different.

\subsubsection{\label{ssdff.sd}Defining OEP's for Repulsive Potential}

\subsection{\label{ss.5.2}Charge-Transfer Interaction Energy}

Charge transfer (CT) between molecules occurs when the net electronic populations 
of interacting molecules change which leads to an additional stabilization 
of a molecular aggregate. Since the effect of CT on the interaction energy 
is known to be often non\hyp{}negligible, it is quite important to model 
CT energies with reasonable accuracy and efficiency when performing energy 
and molecular dynamics calculations with \emph{ab initio} force fields, 
especially in donor\hyp{}acceptor systems such as H\hyp{}bonded species. 
However, estimating the CT stabilization energy is usually much more expensive 
than estimating the contributions due to other effects such as electrostatics, 
exchange\hyp{}repulsion and dispersion.\cite{Gordon.Fedorov.Pruitt.Slipchenko.ChemRev.2012}

As an illustration of application of the OEP technique for efficient calculation 
of CT energy we consider the second\hyp{}order perturbation theory of Murrell et al.,
in which the energy associated with the charge transfer from molecule $A$ to $B$ is given by
%
\begin{equation} \label{e:ct-murell-etal.e}
 E^{A\rightarrow B} = 2 \sum_{i\in A}^{\rm Occ} \sum_{n\in B}^{\rm Vir} 
  \frac{\vert V^{A\rightarrow B}_{in} \vert^2 }{\varepsilon_i - \varepsilon_n}
\end{equation}
%
%coupling constants between occupied and virtual molecular orbitals 
%are re-expressed by the effective one-electron potentials. The resulting formula provides new possibilities 
%of optimization and applications in condensed phase simulations. [3]
%
where $\varepsilon_i$ and $\varepsilon_n$ are the canonical energies
of the occupied (denoted by `Occ') and virtual (denoted by `Vir') 
HF molecular orbitals, respectively,
whereas the coupling constant is given by
%
\begin{multline} \label{e:ct-murell-etal.vin}
 V^{A\rightarrow B}_{in} = 
        \tBraKet{i}{\hat{v}^B_{\rm tot} }{n} 
      - \sum_{j\in B}^{\rm Occ} \BraKet{nj}{ij} 
      - \sum_{k\in A}^{\rm Occ} S_{nk} \tBraKet{k}{\hat{v}^B_{\rm tot} }{i} \\
      - \sum_{j\in B}^{\rm Occ} S_{ij} \tBraKet{j}{\hat{v}^A_{i} }{n}  
     + \sum_{k\in A}^{\rm Occ} \sum_{j\in B}^{\rm Occ}  
        S_{kj} \left( 1 - \delta_{ik} \right) 
        \BraKet{nj}{ik} \;.
\end{multline}
%
In the above expression, the following effective potentials,
%
\begin{subequations} 
\begin{align} \label{e:ct-murell-etal.vtot-vi}
 \hat{v}^B_{\rm tot} &= \hat{v}^B_{\rm nuc} + 2\sum_{j\in B}^{\rm Occ} \hat{v}^B_{jj} \quad\text{ and }\\ 
 \hat{v}^A_{i      } &= \hat{v}^A_{\rm tot} - 2\hat{v}^A_{ii} \;,
\end{align}
\end{subequations}
%
were introduced without making any approximation to the original equation
from Ref. One can immediately notice that the five summation terms
from Eq.~\eqref{e:ct-murell-etal.vin} can be classified into three groups
regarding the type of ERI's that are required:
(i) $\BraKet{AB}{BB}$ -- the first two terms;
(ii) $\BraKet{AA}{BB}$ -- the third term and
(iii) $\BraKet{BB}{AA}$ -- the two last terms. 
Note also that there are no exchange\hyp{}like terms needed in this case.
Therefore, all the contributions can be re\hyp{}cast in terms of the OEP's.
%Note that groups (ii) and (iii) correspond to the Coulomb interactions between
%The group (i) corresponds to the overlap\hyp{}like integrals, whereas the groups (ii) and (iii)
%correspond to the Coulomb\hyp{}like integrals

\paragraph{Group (i).}
Group (i) can be rewritten by noticing that
%
\begin{equation} \label{e:ct-murell-etal.group-i.notice}
 \sum_{j\in B}^{\rm Occ} 
 \BraKet{nj}{ij} =
 \Bra{i} \left[ \hat{v}^B_{nj} \Ket{j} \right]  \;,
\end{equation}
%
which, by combining with the first term from Eq.~\eqref{e:ct-murell-etal.vin}, 
allows to apply the density fitting in an
incomplete basis set as follows
%
\begin{equation} \label{e:v-oep.ct}
\Bra{i} \left[ \hat{v}^B_{\rm tot} \Ket{n} - \sum_{j\in B}^{\rm Occ} \hat{v}^B_{nj} \Ket{j} \right]
\cong \Bra{i} \sum_{\eta\in B}^{\rm DF} V^B_{n\eta} \Ket{\eta}
\end{equation}
%
with the error density defined by
%
\begin{multline}
 \Xi_n({\bf r}) = 
  \left\{
      v_{nuc}^B({\bf r}) + 2 \sum_{j\in B}^{\rm Occ}
   \int \frac{ \vert \phi_j({\bf r}') \vert^2 }{\vert {\bf r}' - {\bf r} \vert} d{\bf r}'
  \right\}
  \phi_n({{\bf r}})  \\
 - \sum_{j\in B}^{\rm Occ} \phi_j({{\bf r}})
   \int \frac{ \phi^*_n({\bf r}') \phi_j({{\bf r}'}) }{\vert {\bf r}' - {\bf r} \vert} d{\bf r}'
 - \sum_{\eta\in B}^{\rm DF} V^B_{n\eta} \varphi_\eta({\bf r}) \;.
\end{multline}
%
Substituting the above equation into Eq.~\eqref{e:z-incompl}
%and then
%using resolution of identity to simplify the 
leads to
%
\begin{equation}
 V^B_{n\xi} = \sum_{\eta\in B}^{\rm DF} \left[ {\bf R}^{-1} \right]_{\xi\eta} b_\eta^{(n)} \;,
\end{equation}
%
where
%
\begin{multline}
 b^{(n)}_\eta = \iint 
           d{\bf r}_1 d{\bf r}_2  
           \frac{ \varphi^*_\eta({\bf r}_1) }
            {\vert {\bf r}_1 - {\bf r}_2 \vert } \\ \times
          \left[ 
           \hat{v}^B_{\rm tot}  \phi_n({\bf r}_2) 
         - \sum_{j\in B}^{\rm Occ} \hat{v}^B_{nj} \phi_j({{\bf r}_2})
           \right]  \;.
\end{multline}
%
The above result is given by a sum of triple integrals from Eq.~\eqref{e:triple-integral}.
However, the following application of the resolution of identity,
%
\begin{equation}
 \hat{v}^B_{\rm tot} \Ket{n} - \sum_{j\in B}^{\rm Occ} \hat{v}^B_{nj} \Ket{j}
 \cong \sum_{\varepsilon\in B}^{\rm RI} H_{n\varepsilon}^B \Ket{\varepsilon} \;,
\end{equation}
%
leads to a much simpler formula
that requires only the one- and two\hyp{}electron integrals. 
Therefore, by combining
equations %which equations?
one arrives to
%
\begin{equation}
 V^B_{n\xi} = \sum_{\eta\in B}^{\rm DF} 
          \sum_{\varepsilon\in B}^{\rm RI}
         \left[ {\bf R}^{-1} \right]_{\xi\eta}
         R_{\eta \varepsilon} 
         H_{n\varepsilon}^B \;,
\end{equation}
%
where
%
\begin{equation}
 H_{n\varepsilon}^B = \sum_{\zeta\in B}^{\rm RI} \left[ {\bf S}^{-1} \right]_{\varepsilon\zeta}
   a_\zeta^{(n)}
\end{equation}
%
and
%%
\begin{align}
 a^{(n)}_\zeta &= \tBraKet{\zeta}{\hat{v}^B_{\rm tot}}{n}
      - \sum_{j\in B}^{\rm Occ} \tBraKet{\zeta}{\hat{v}^B_{nj}}{j} \nonumber \\
 &= \sum_{y\in B} W^{(y)}_{\zeta n} 
  + \sum_{j\in B}^{\rm Occ} 
  \left\{
   2\BraKet{\zeta n}{jj} - \BraKet{\zeta j}{nj} 
  \right\} \;.
\end{align}
%
Note that all the calculations that are required to obtain $V^B_{n\xi}$ are performed
solely on the densities and basis sets associated with the unperturbed molecule $B$.
Therefore, $V^B_{n\xi}$ can be considered as effective fragment parameters
used to compute the first two terms of Eq.~\eqref{e:ct-murell-etal.vin} by
%
\begin{equation} \label{e:ct.group-i.final}
        \tBraKet{i}{\hat{v}^B_{\rm tot} }{n} 
      - \sum_{j\in B}^{\rm Occ} \BraKet{nj}{ij} 
       = \sum_{\eta\in B}^{\rm RI} V^B_{n\eta} S_{\eta i} \;,
\end{equation}
%
which is a great simplification over the original form of group (i)
because only the overlap integrals between the $i$th MO on molecule $A$
and $\eta$th auxiliary orbital on molecule $B$ need to be evaluated on the fly.
Note that the only approximation made so far was the application of density fitting
and resolution of identity. If the auxiliary and intermediate
basis sets are sufficiently large, the errors
due to this approximation can be minimal and negligible.



\paragraph{Group (ii).}
The term belonging to this group can be considered as a sum of interaction
energies between the total charge density distribution of molecule $B$
and the partial density $\rho_{ik}({\bf r})$ of molecule $A$,
weighted by the overlap integrals $S_{nk}$. Using the distributed multipole 
expansion this group can be written as
%
\begin{equation} \label{e:ct.group-2}
      - \sum_{k\in A}^{\rm Occ} S_{nk} \tBraKet{k}{\hat{v}^B_{\rm tot} }{i} 
 \cong - \sum_{k\in A}^{\rm Occ} S_{nk} \rho_{ki}^A \odot \rho^B_{\rm tot} \;.
\end{equation}
%
Consider now a situation in which the localized occupied MO's are used.
If the multipole series expansions are truncated
at the distributed charges $q_i$ and dipole moments ${\BM{\upmu}_i}$, 
located at the centroids ${\bf r}_i = \tBraKet{i}{\hat{\bf r}}{i}$, 
then the interaction energy becomes
%
\begin{multline}
 \rho_{ki}^A \odot \rho^B_{\rm tot} 
 \approx 
 q_{ki} 
 \left[
  %\sum_{i\in A}^{\rm LC} 
 \sum_{y\in B}^{\rm At}
  \frac{Z_y}{\vert {\bf r}_y - {\bf r}_i \vert } 
 +
 2\sum_{j\in B}^{\rm LC}
  \frac{q_j}{\vert {\bf r}_j - {\bf r}_i \vert } 
 \right] \\
 + \text{ dipole terms } \;.
\end{multline}
%
Note however that $q_{ik} = -\delta_{ik}$ and $q_j = -1$. Moreover, 
if the localized orbitals are perfectly spherical distributed dipole moments vanish.
This means that Eq.~\eqref{e:ct.group-2} can be approximated as follows:
%
\begin{equation} \label{e:ct.group-ii.final}
      - \sum_{k\in A}^{\rm Occ} S_{nk} \tBraKet{k}{\hat{v}^B_{\rm tot} }{i} 
 \approx S_{ni}  \left[
  %\sum_{i\in A}^{\rm LC} 
 \sum_{y\in B}^{\rm At}
  \frac{Z_y}{ r_{yi} } 
 -
 \sum_{j\in B}^{\rm LC}
  \frac{2}{r_{ji}} 
 \right] \;.
\end{equation}
%
Therefore, only overlap integrals and relative distances between
atomic and LC's are needed, which leads to a great reduction of the
calculation cost,
as compared either to the original expression or to the multipole expansion (left\hyp{} and right\hyp{}hand sides
of Eq.~\eqref{e:ct.group-2}, respectively).
We shall refer to this approximation as to
the spherical overlap approximation (SOA). Note that in order for this approximation to be valid,
occupied molecular orbitals need to be localized.

\paragraph{Group (iii).}
The terms with the overlap integrals involving the occupied MO on $A$
can be combined into a single summation term, i.e.,
\begin{multline}
       - \sum_{j\in B}^{\rm Occ} S_{ij} \tBraKet{j}{\hat{v}^A_{i} }{n}  
     + \sum_{k\in A}^{\rm Occ} \sum_{j\in B}^{\rm Occ}  
        S_{kj} \left( 1 - \delta_{ik} \right) 
        \BraKet{nj}{ik} \\ = -
 \sum_{k\in A}^{\rm Occ} 
 \sum_{j\in B}^{\rm Occ}
 S_{kj} 
 \tBraKet{j}{
 \underbrace{
 \left\{ 
  \delta_{ik} \left( \hat{v}^A_{k} + \hat{v}^A_{ik} \right)
  - \hat{v}^A_{ik}
 \right\} 
  }_{\hat{v}^{A,{\rm eff}}_{ik}}
 }{n}  \\ \cong -
%
  \sum_{k\in A}^{\rm Occ} 
 \sum_{j\in B}^{\rm Occ}
 S_{kj} 
 \rho_{nj}^B \odot \rho^{A,{\rm eff}}_{ik} \;,
\end{multline}
%
where the effective potential ${v}^{A,{\rm eff}}_{ik}$ (with the associated 
effective density $\rho^{A,{\rm eff}}_{ik}$) 
is defined by
%
\begin{equation} \label{e:ct.group-iii.oep}
 {v}^{A,{\rm eff}}_{ik}({\bf r}) \equiv
 \delta_{ik} 
 \left[
  v^A_{\rm tot} ({\bf r}) - 2 v^A_{kk} ({\bf r}) + v^A_{ik} ({\bf r})
 \right] 
  - v^A_{ik} ({\bf r}) \;.
\end{equation}
%
Now, it is first assumed here that $\rho^{B}_{nj} ({\bf r}) \approx 0$ if occupied MO's on $B$ are localized.
This means that
%
\begin{multline} \label{e:ct.group-iii.final.a}
        - \sum_{j\in B}^{\rm Occ} S_{ij} \tBraKet{j}{\hat{v}^A_{i} }{n}  
     + \sum_{k\in A}^{\rm Occ} \sum_{j\in B}^{\rm Occ}  
        S_{kj} \left( 1 - \delta_{ik} \right) 
        \BraKet{nj}{ik}  \\
 \approx 0 \;.
\end{multline}
%
The above result is a crude approximation, since the channels $j\rightarrow n$
are quite important contributions in most of perturbation theory treatments.
%virtual MO's can penetrate the occupied MO's to some extent.
In order to include the $\rho^{B}_{nj}$ density, it is approximately represented here by a set of effective 
cumulative
atomic charges $\{ q^{B,(nj)}_{y} \}$ associated with the effective one\hyp{}particle density matrix
%
\begin{equation}
 D^{B,(nj)}_{\beta\delta} = C_{\beta n} C_{\delta j} \;.
\end{equation}
%
In this work, the effective charges were computed via the CAMM or ESP methods as
discussed in Section~XXX. Applying the SOA 
and the effective potential from Eq.~\eqref{e:ct.group-iii.oep} 
leads to 
%
\begin{multline} \label{e:ct.group-iii.final.b}
        - \sum_{j\in B}^{\rm Occ} S_{ij} \tBraKet{j}{\hat{v}^A_{i} }{n}  
     + \sum_{k\in A}^{\rm Occ} \sum_{j\in B}^{\rm Occ}  
        S_{kj} \left( 1 - \delta_{ik} \right) 
        \BraKet{nj}{ik}  \\
 \approx  -
 \sum_{j\in B}^{\rm Occ} S_{ij}
 \sum_{y\in B}^{\rm At} 
 q^{B,(nj)}_{y} 
 \left[ 
   \sum_{x\in A}^{\rm At}
   \frac{Z_x}{r_{xy}}
  + \frac{2}{r_{iy}}
  - \sum_{k\in A}^{\rm Occ}
    \frac{2}{r_{ky}} 
 \right]
 \;.
\end{multline}
%


\paragraph{Final OEP-based forms of the coupling constant.}
Gathering the results from Eqs. the coupling constant
can be given as follows
%
\begin{subequations}
\begin{align} \label{e:ct-murell-etal.vin.oep}
\Approx{ V_{in}^{A\rightarrow B} }{1}
       &= \sum_{\eta\in B}^{\rm RI} V^B_{n\eta} S_{\eta i} 
 +
S_{ni} u_i^{BA} \;, \\
%
\Approx{ V_{in}^{A\rightarrow B} }{2} &=
\Approx{ V_{in}^{A\rightarrow B} }{1} -
  \sum_{j   \in B}^{\rm Occ} S_{ij}
  \sum_{y   \in B}^{\rm At} q_y^{B,(nj)} w_{yi}^{BA}
\;,
\end{align}
\end{subequations}
%
where the auxiliary variables are
%
\begin{subequations}
\begin{align} \label{e:ct-murell-etal.vin.oep.aux-var}
 u_i^{BA} &\equiv    
 \sum_{y\in B}^{\rm At}
  \frac{Z_y}{ r_{yi} } 
 -
 \sum_{j\in B}^{\rm LC}
  \frac{2}{r_{ji}} 
                \;, \\
 w_{yi}^{BA} &\equiv 
   \sum_{x\in A}^{\rm At}
   \frac{Z_x}{r_{xy}}
  + \frac{2}{r_{iy}}
  - \sum_{k\in A}^{\rm Occ}
    \frac{2}{r_{ky}} 
                \;.
\end{align}
\end{subequations}
%
To compute the interaction energy due to CT from $A$ to $B$ by using the above SOA\hyp{}based
approximations, coupling elements need to be transformed from localized to canonical MO basis, i.e.,
%
\begin{equation} \label{e:ect-oep.prefinal}
 E^{A\rightarrow B} \approx
 2 
 \sum_{i\in A}^{\rm CMO}
 \sum_{n\in B}^{\rm Vir}
 \frac{
 \lvert
   \sum_{i'\in A}^{\rm LMO} L_{ii'}^A
   \Approx{V_{i'n}^{A\rightarrow B}}{} 
 \rvert^2 }{\varepsilon_i - \varepsilon_n} \;,
\end{equation}
%
where ${\bf L}^A$ is the CMO\hyp{}LMO transformation matrix for occupied orbitals of $A$. 
Note however, that the effective potential parameters from the density fitted term of group (i) 
do not involve any occupied orbitals.
Therefore, to save computational costs, only groups (ii) and (iii) need to be transformed
whereas for group (i) the overlap integrals from Eq.~\eqref{e:ct.group-i.final} 
can be already evaluated in CMO basis.
Thus, the final working formula for the interaction energy due to CT from $A$ to $B$ reads
%
\begin{multline} \label{e:ect-oep.final}
 E^{A\rightarrow B} \approx
 2 
 \sum_{i\in A}^{\rm CMO}
 \sum_{n\in B}^{\rm Vir}
 \frac{1}{\varepsilon_i - \varepsilon_n} \times 
 \Bigg(
   \Approx{V_{in}^{A\rightarrow B}}{\rm G1} \\
  +
   \sum_{i'\in A}^{\rm LMO} L_{ii'}^A
   \left\{
   \Approx{V_{i'n}^{A\rightarrow B}}{\rm G2} 
  +\Approx{V_{i'n}^{A\rightarrow B}}{\rm G3}
  \right\}
 \Bigg)^2 \;,
\end{multline}
%
where the subscripts G$n$ ($n$ = 1, 2, 3) denote a particular group of 
terms from Eqs.~\eqref{e:ct.group-i.final}, \eqref{e:ct.group-ii.final} and \eqref{e:ct.group-iii.final.a}
or \eqref{e:ct.group-iii.final.b}. 
The total CT energy is given by the sum of the above contribution and the twin contribution
due to CT from molecule $B$ to $A$.

%
%he above formula does not explicitly define an OEP due to the summation over different
%state vectors. 
%However, it can be achieved
%by transforming from MO to AO basis, i.e.,
%%
%\begin{equation}
% \hat{v}^B_{\rm tot} \Ket{n} - \sum_{j\in B}^{\rm Occ} \hat{v}^B_{nj} \Ket{j}
% = \sum_{\beta\in B}^{\rm AO} 
% \left\{ 
%  C_{\beta n} \hat{v}^B_{\rm tot} - \sum_{j\in B}^{\rm Occ} C_{\beta j} \hat{v}^B_{nj} 
% \right\} \Ket{\beta} \;,
%\end{equation}
%%
%which leads to the OEP in spatial representation
%%
%\begin{multline}
% v_{\beta n}({\bf r}) \equiv C_{\beta n} v_{nuc}^B({\bf r}) \\ + 
%  \sum_{j\in B}^{\rm Occ} \Bigg\{ 
%   2 C_{\beta n} \int \frac{ \vert \phi_j({\bf r}') \vert^2 }{\vert {\bf r}' - {\bf r} \vert} d{\bf r}'
% %\\
% - C_{\beta j} \int \frac{ \phi^*_n({\bf r}') \phi_j({\bf r}') }{\vert {\bf r}' - {\bf r} \vert} d{\bf r}'
%  \Bigg\}  
%%
%\end{multline}
%%

\subsection{\label{s:413s6}Reduction of computational costs}

Until this moment in this work, two models have been tested with application of the OEP technique.
In this section, computational cost is estimated from the formulas.
In
Table~\ref{t:oep-costs} 
%
{
\renewcommand{\arraystretch}{1.4}
\begin{table}[b]
\caption[Computational cost of the benchmark and OEP-based methods: calculation of coupling constant]
{{\bf Computational cost of the benchmark and OEP-based methods\footnotemark[1]}
}
\label{t:oep-costs}
\begin{ruledtabular}
\begin{tabular}{lcccccc}
Method             && Murrell et al. &&    EFP2        &&      OEP     \\ 
	\cline{1-7}
Group (i)         &&  &&        &&   $obQ$     \\
Group (ii)        &&  &&        &&   $ob$      \\
Group (iii)       &&  &&        &&   $ob(n+o)$     \\
Total Cost        &&  && $10ob^2 + 2b^3 + o^2b^2$       &&   $ob(Q+n+o)$     \\
\end{tabular}
\end{ruledtabular}
%
\footnotetext[1]{Numbers of: $o$ - occupied molecular orbitals; $b$ - atomic basis functions; $n$ - atoms; 
$Q$ - auxiliary basis functions. It was assumed that the number of virtual orbitals is equal to $b$.}
%
\end{table}
}
%
It is clear that the cost of the OEP\hyp{}based equations is much lesser than the cost of EFP2 method,
and many orders of magnitude lesser than the original theory of Murrell et al.


\section{\label{s:6}Summary and a few concluding remarks}

Bla.

\begin{acknowledgments}
This project is carried out under POLONEZ programme which has received funding from the European Union's
Horizon~2020 research and innovation programme under the Marie Skłodowska-Curie grant agreement 
No.~665778. This project is funded by National Science Centre, Poland 
(grant~no. 2016/23/P/ST4/01720) within the POLONEZ 3 fellowship.
\end{acknowledgments}

%%
\appendix

\section{Optimized Auxiliary Basis Sets for OEP Applications\label{a:auxiliary-basis}}

To fit the auxiliary DF basis for the treatment of the overlap\hyp{}like
OEP matrix elements with the operator $\hat{v}_{\rm eff}$, 
the following objective function is minimized
%
\begin{equation} \label{e:a1-obj}
 Z[\{\xi\}] = \sum_{\alpha i} \left[ 
     \tBraKet{\alpha}{\hat{v}_{\rm eff}}{i} - 
     \sum_{\xi}^{\rm DF} V_{\xi i} S_{\alpha \xi} 
    \right]^2 \;,
\end{equation}
%
where $\{\xi\}$ is the auxiliary basis set to optimize, whereas $\{\alpha\}$
is the `test' basis set used to probe the accuracy of the density fitting.
The forms of the Pauli and charge\hyp{}transfer DF matrices $\tBraKet{\alpha}{\hat{v}_{\rm eff}}{i}$
can be directly derived from Eqs.~\eqref{e:v-oep.rep} and \eqref{e:v-oep.ct}, respectively, 
by expanding the MO's in terms of atomic basis functions. The working formulae
are given below:
%
\begin{subequations} \label{e:a1-v.oep}
 \begin{align}
   \tBraKet{\alpha}{\hat{v}_{\rm eff}^{\rm Rep}}{i}
     &= \sum_{x}^{\rm At} W_{\alpha i}^{(x)} \nonumber   \\ 
        + \sum_{\beta\gamma\delta} &
           \left\{ 
             2 C_{\beta i} D_{\gamma\delta} - C_{\gamma i} D_{\beta \delta}
           \right\}
           \BraKet{\alpha\beta}{\gamma\delta} \;,\\
   \tBraKet{\alpha}{\hat{v}_{\rm eff}^{\rm CT}}{n} 
     &= \sum_{x}^{\rm At} W_{\alpha n}^{(x)} \nonumber   \\
       + \sum_{\beta\gamma\delta} &
           \left\{
             2 C_{\beta n} D_{\gamma\delta} - C_{\gamma n} D_{\beta \delta}
           \right\}
           \BraKet{\alpha\beta}{\gamma\delta}
    \;.
 \end{align}
\end{subequations}
%
In the above equations, ${\bf C}$ is the LCAO\hyp{}MO matrix whereas ${\bf D}$
is the one\hyp{}particle density matrix in AO basis.

%
%\section{\label{a:mcmurchie-davidson} McMurchie-Davidson Method for Asymmetric Electron Repulsion Integrals}
%
%Evaluation of integrals $(ijk\vert l)$,
%that are necessary for the generalized density fitting of OEP's,
%can be easily carried out by re\hyp{}expressing the
%unnormalized product of three primitive Gaussian\hyp{}type (GTO) functions,
%%
%\begin{equation} \label{e:1}
%[ijk] \equiv \phi_i({\bf r}) \phi_j({\bf r}) \phi_k({\bf r}) \;,
%\end{equation}
%%
%in terms of Hermite functions. It can be shown that
%%
%\begin{multline} \label{e:2}
%   [ijk] = E_{ijk} \sum_{N=0}^{n_1+n_2+n_3} \sum_{L=0}^{l_1+l_2+l+3} \sum_{M=0}^{m_1+m_2+m_3} 
%	\\
%          d_N^{n_1n_2n_3} d_L^{l_1l_2l_3} d_M^{m_1m_2m_3}
%          \Lambda_N(x_R)\Lambda_L(y_R)\Lambda_M(z_R)e^{-\alpha_Rr_R^2}
%\end{multline}
%%
%in which the McMurchie-Davidson $d3$ coefficients are given by the following recurrence relationships
%%
% \begin{align} \label{e:3}
%  d_N^{n_1+1,n_2,n_3} &= \frac{1}{2\alpha_R} d_{N-1}^{n_1n_2n_3} 
%                             + \vert {\bf R} - {\bf A}\vert_x d_N^{n_1n_2n_3} \nonumber \\ 
%			    & \qquad\qquad + (N+1) d_{N+1}^{n_1n_2n_3} \\
%  d_N^{n_1,n_2+1,n_3} &= \frac{1}{2\alpha_R} d_{N-1}^{n_1n_2n_3} 
%                             + \vert {\bf R} - {\bf B}\vert_x d_N^{n_1n_2n_3} \nonumber \\
%			    & \qquad\qquad + (N+1) d_{N+1}^{n_1n_2n_3} \\
%  d_N^{n_1,n_2,n_3+1} &= \frac{1}{2\alpha_R} d_{N-1}^{n_1n_2n_3} 
%                             + \vert {\bf R} - {\bf C}\vert_x d_N^{n_1n_2n_3} \nonumber \\
%			    & \qquad\qquad + (N+1) d_{N+1}^{n_1n_2n_3} 
%\end{align}
%%
%with $d_0^{000} = 1$.
%In Eq.~\eqref{e:1},
%$\phi_i({\bf r})$ is given by
%%
%\begin{equation}
%\phi_i({\bf r}) \equiv x_A^{n_1} y_A^{l_1} z_A^{m_1} e^{-\alpha_1r_A^2}
%\end{equation}
%%
%where 
%${\bf r}_A \equiv {\bf r} - {\bf A}$,
%${\bf A}$ is the centre of the GTO, $\alpha_1$ its exponent, whereas $n_1,l_1,m_1$
%the Cartesian angular momenta, with the total angular momentum $\theta_1 = n_1+l_1+m_1$.
%
%It can be easily shown that the multiplicative constant $E_{ijk}$ is given by
%%
%\begin{multline}
%  E_{ijk}(\alpha_1,\alpha_2,\alpha_3)  = \exp{\left[-\frac{\alpha_1\alpha_2}
%                                        {\alpha_1+\alpha_2}\vert {\bf A}-{\bf B}\vert^2\right]} \\ \times
%                                         \exp{\left[-\frac{(\alpha_1+\alpha_2)\alpha_3}
%                                        {\alpha_1+\alpha_2+\alpha_3}
%                                         \vert {\bf P}-{\bf C}\vert^2\right]} 
%\end{multline}
%%

% -----------------------
\bibliography{references}
% -----------------------

\end{document}
