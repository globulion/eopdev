%%
%%   Version 3.1 of 16 May 2019.
%%

\usepackage{graphicx}% Include figure files
\usepackage{dcolumn}% Align table columns on decimal point
\usepackage{bm}% bold math

% hyphenation
\usepackage{hyphenat}

% tables
\usepackage{multirow}
\usepackage{booktabs}

% mathematics
\usepackage{amsmath}
\usepackage{amsfonts}
\usepackage{amssymb}
\usepackage{amsbsy}
\usepackage{mathrsfs}
\usepackage{upgreek}
\usepackage[cbgreek]{textgreek}

%---------------------------------------------------
% Approximations
\newcommand{\Approx}[2]{\ensuremath{\text{Ap}_{#2} \left[ {#1} \right] }}
% happy integral
\newcommand{\rint}[1]{\mbox{\Large $ \int\limits_{\mbox{\tiny  $#1$}}$}}
% SHORTCUTS
%\newcolumntype{,}{D{.}{,}{2}}
\newcommand{\citee}[1]{\ensuremath{\scriptsize^{\citenum{#1}}}}
\newcommand{\HRule}{\rule{\linewidth}{0.2mm}}
% Quantum notation
\newcommand{\Bra}[1]{\ensuremath{\bigl\langle {#1} \bigl\lvert}}
\newcommand{\Ket}[1]{\ensuremath{\bigr\rvert {#1} \bigr\rangle}}
\newcommand{\BraKet}[2]{\ensuremath{\bigl\langle {#1} \bigl\lvert {#2} \bigr\rangle}}
\newcommand{\tBraKet}[3]{\ensuremath{\bigl\langle {#1} \bigl\lvert {#2} \bigl\lvert {#3} \bigr\rangle}}
%
\newcommand{\bra}[1]{\ensuremath{\bigl( {#1} \bigl\lvert}}
\newcommand{\ket}[1]{\ensuremath{\bigr\rvert {#1} \bigr)}}
\newcommand{\braket}[2]{\ensuremath{\bigl( {#1} \bigl\lvert {#2} \bigr)}}
\newcommand{\tbraket}[3]{\ensuremath{\bigl( {#1} \bigl\lvert {#2} \bigl\lvert {#3} \bigr)}}
% Math
\newcommand{\pd}{\ensuremath{\partial}}
\newcommand{\DR}{\ensuremath{{\rm d} {\bf r}}}
%\newcommand{\BM}[1]{\ensuremath{\mbox{\boldmath${#1}$}}}
\newcommand{\BM}[1]{\bm{#1}}
% Chemistry (formulas)
\newcommand{\ch}[2]{\ensuremath{\mathrm{#1}_{#2}}}
% Math 
\newcommand{\VEC}[1]{\ensuremath{\mathrm{\mathbf{#1}}}}
% vector nabla
\newcommand{\Nabla}{\ensuremath{ \BM{\nabla}}}
% derivative
\newcommand{\FDer}[3]{\ensuremath{
\bigg(
\frac{\partial #1}{\partial #2}
\bigg)_{#3}}}
% diagonal second derivative
\newcommand{\SDer}[3]{\ensuremath{
\biggl(
\frac{\partial^2 #1}{\partial #2^2}
\biggr)_{#3}}}
% off-diagonal second derivative
\newcommand{\SSDer}[4]{\ensuremath{
\biggl(
\frac{\partial^2 #1}{\partial #2 \partial #3}
\biggr)_{#4}}}
% derivatives without bound
% derivative
\newcommand{\fderiv}[2]{\ensuremath{
\frac{\partial #1}{\partial #2}}}
% diagonal second derivative
\newcommand{\sderiv}[2]{\ensuremath{
\frac{\partial^2 #1}{\partial #2^2}
}}
% off-diagonal second derivative
\newcommand{\sderivd}[3]{\ensuremath{
\frac{\partial^2 #1}{\partial #2 \partial #3}
}}
% derivatives for tables
\newcommand{\fderivm}[2]{\ensuremath{
{\partial #1}/{\partial #2}}}
% diagonal second derivative
\newcommand{\sderivm}[2]{\ensuremath{
{\partial^2 #1}/{\partial #2^2}
}}
% off-diagonal second derivative
\newcommand{\sderivdm}[3]{\ensuremath{
{\partial^2 #1}/{\partial #2 \partial #3}
}}
% ERIs and OEIs
\newcommand{\OEIc}[3]{\ensuremath{\left(#1 \lvert #2 \rvert #3 \right)}}
\newcommand{\ERIc}[4]{\ensuremath{\left(#1 #2 \vert #3 #4 \right)}}

% Partial density and potential
\newcommand{\PartPot}[4]{\ensuremath{\frac{#1 #2}{\lvert #3-#4 \rvert }}}

% trace operator
\DeclareMathOperator{\Tr}{Tr}

%\draft % marks overfull lines with a black rule on the right

% Define location of graphics
\graphicspath{{./figures/}}

\begin{document}
\preprint{AIP/123-OEP}

\title{Ab Initio Effective One-Electron Potential Operators. II.
Applications for Exchange-Repulsion Energy in Effective Fragment Potentials}

\author{Bartosz B{\l}asiak}
\email[]{blasiak.bartosz@gmail.com}
\homepage[]{https://www.polonez.pwr.edu.pl}

\author{Wojciech Bartkowiak}

\affiliation{Department of Physical and Quantum Chemistry, Faculty of Chemistry, 
Wroc{\l}aw University of Science and Technology, 
Wybrze{\.z}e Wyspia{\'n}skiego 27, Wroc{\l}aw 50-370, Poland}

\date{\today}

\begin{abstract}
Recently, it was shown that
the effective one\hyp{}electron potentials (OEP) method
of effective elimination of the electron repulsion integrals
can be used to lower the computational cost
of evaluation of the charge transfer interaction energy
within the effective fragment potential method (EFP2)
by the factor of 20, making it no longer a bottleneck
in EFP2\hyp{}based simulations of complex systems.
Here, OEP technique 
is used to enhance computational efficiency in evaluating the
exchange\hyp{}repulsion EFP2 interaction energy
by incorporating the kinetic energy operator in the OEP
via the extended density fitting in incomplete auxiliary basis.
It is shown that the computational cost can be reduced by a factor
of 1.5 and kinetic energy integrals are no longer necessary to evaluate.
\end{abstract}

\pacs{}

\maketitle

\tableofcontents

\section{\label{s:1.introduction}Introduction}

The exchange\hyp{}repulsion (EXR) interaction energy in the
second generation of the
effective fragment potentials (EFP2) method\cite{Gordon.Smith.Xu.Slipchenko.AnnuRevPhysChem.2013,
   Nguyen.Pachter.Day.JCP.2014,
   Day.Jensen.Gordon.Webb.Stevens.Krauss.Garmer.Basch.Cohen.JCP.1996}
describes the non\hyp{}electrostatic repulsion
between two Hartree\hyp{}Fock\cite{Roothaan.RevModPhys.1951} 
isolated and unperturbed wavefunctions in an interacting
bi\hyp{}molecular complex.
The EXR EFP2 term was derived from the intermolecular perturbation theory
of Murrell et al.\cite{Murrell.Randic.Williams.Longuet-Higgins.ProcRSocLondA.1965,Otto.Ladik.ChemPhys.1975} 
up to second order with respect to the wavefunction overlap,
and is given by~\cite{Jensen.JCP.1996,Jensen.Gordon.MolPhys.1996,Jensen.Gordon.JCP.1998}
%
\begin{equation} \label{e:exr-efp2}
 E^{\rm Ex-Rep} \approx
 E^{\rm Ex} +
 E^{\rm Rep}(S^{-1}) + 
 E^{\rm Rep}(S^{-2}) \;.
\end{equation}
%
In the above equation,
$E^{\rm Ex}$ is the exchange energy,\cite{Jensen.JCP.1996}
%
\begin{equation} \label{e:exc-efp2}
 E^{\rm Ex} \approx 2 \sum_{i\in A}^{\rm Locc} \sum_{j\in B}^{\rm Locc} 
 \sqrt{\frac{-2\ln{\vert S_{ij}\vert}}{\pi}} \frac{S^2_{ij}}{r_{ij}} \;,
\end{equation}
%
whereas the first\hyp{} and second\hyp{}order repulsion terms
are accordingly~\cite{Jensen.Gordon.MolPhys.1996,Jensen.Gordon.JCP.1998}
%
\begin{multline} \label{e:rep-efp2.s1}
 E^{\rm Rep}(S^{-1}) \approx - 2 \sum_{i\in A}^{\rm Locc} \sum_{j\in B}^{\rm Locc} S_{ij} \\
 \times
 \left\{
 \sum_{k\in A}^{\rm Locc} F_{ik}^A S_{kj} 
+\sum_{l\in B}^{\rm Locc} F_{jl}^A S_{li} 
 -2T_{ij}
 \right\}
\end{multline}
%
and 
%
\begin{multline} \label{e:rep-efp2.s2}
 E^{\rm Rep}(S^{-2}) \approx 2 \sum_{i\in A}^{\rm Locc} \sum_{j\in B}^{\rm Locc} S_{ij}^2 
 \\ \times 
 \Bigg\{
 \sum_{x\in A}^{\rm At}  \frac{-Z_x}{r_{xb}}
+\sum_{y\in B}^{\rm At}  \frac{-Z_y}{r_{ya}} 
+\sum_{k\in A}^{\rm Locc} \frac{   2}{r_{jk}}
+\sum_{l\in B}^{\rm Locc} \frac{   2}{r_{il}}
-r_{ij}^{-1}
 \Bigg\} \;.
\end{multline}
%
The indices $i$, $j$, $k$ and $l$ label the localized occupied orbitals (LMO's, denoted by `Locc')
located at their charge centroids ${\bf r}_{i(j)}$,
whereas $x$ and $y$ label atomic nuclei (denoted by `At') with atomic numbers 
$Z_{x(y)}$ located at ${\bf r}_{x(y)}$. The relative distances are
defined by $r_{uw}=\vert {\bf r}_u-{\bf r}_w\vert$.
It is known that evaluating Eq.~\eqref{e:exr-efp2}
is remarkably efficient and gives usually very accurate
estimates of the benchmark EXR energy according to the symmetry 
adapted perturbation theory\cite{Jeziorski.Moszynski.Szalewicz.ChemRev.1994} (SAPT).
The computational cost of
the exchange and second\hyp{}order terms
is much less than the first\hyp{}order term,
which requires evaluaton of not only the one\hyp{}electron overlap,
but also the kinetic
energy integrals in a space of occupied molecular orbital (MO) basis.

In this work, an alternative formulation for the first\hyp{}order
term is proposed based on the effective one\hyp{}electron potential (OEP)
operator technique developed in the preceeding contribution.~\cite{Blasiak.Bednarska.Choluj.Bartkowiak.JCP.2019}
It is shown that calculation of kinetic energy integrals can be avoided
by introducing OEP's.

\section{Results}

A theoretical foundation of the
first\hyp{}order repulsive term in the EFP2 model
is the perturbation theory of
Murrell et~al.~\cite{Murrell.Randic.Williams.Longuet-Higgins.ProcRSocLondA.1965} 
in which
%
\begin{multline} \label{e:rep.murrell-etal.S1}
    E^{\rm Rep}(S^{-1}) = -2\sum_{i\in A}^{\rm Locc} \sum_{j\in B}^{\rm Locc}
               S_{ij} \Big\{
           W^A_{ij} 
         + W^B_{ji} \\
 + \sum_{k\in A}^{\rm Occ} \left[ 2\BraKet{ij}{kk} - \BraKet{ik}{jk} \right] 
 + \sum_{l\in B}^{\rm Occ} \left[ 2\BraKet{ij}{ll} - \BraKet{il}{jl} \right]
                \Big\}
\end{multline}
%
where $W^A_{ij} \equiv \tBraKet{i}{\hat{v}^A_{\rm nuc}}{j}$,
$\hat{v}^A_{\rm nuc}$
is the nucleus\hyp{}electron potential operator
and
ERI is defined by
%
\begin{equation} \label{e:eri}
	\BraKet{\alpha\beta}{\gamma\delta} \equiv
	\iint 
	\frac{ \phi_\alpha^{*}({\bf r}_1) \phi_\beta({\bf r}_1) 
	       \phi_\gamma^{*}({\bf r}_2) \phi_\delta({\bf r}_2) }{ \vert {\bf r}_1 - {\bf r}_2 \vert}
	d{\bf r}_1 d{\bf r}_2  \;.
\end{equation}
%
Now, note that the Coulomb and exchange integrals can be re\hyp{}cast as follows:
%
\begin{subequations}
 \begin{align}
 \BraKet{ij}{kk} &\equiv \sum_{\mu\in A} 
     C_{\mu i}^A \tBraKet{\mu}{\hat{v}_{kk}}{j} \;, \\
 \BraKet{ik}{jk} &\equiv \sum_{\mu\in A} 
     C_{\mu k}^A \tBraKet{\mu}{\hat{v}_{ik}}{j} \;.
 \end{align}
\end{subequations}
%
%Similarly, the exchange integrals reduce to
%
%\begin{equation}
% \BraKet{ik}{jk} \equiv \sum_{\mu\in A} 
%     C_{\mu k}^A \tBraKet{\mu}{\hat{v}_{ik}}{j} \;.
%\end{equation}
%
In the above equations, the auxiliary potential operators are given by
%
\begin{equation}
  \hat{v}_{ik} \equiv \int d{\bf r} \Ket{{\bf r}} 
        \left[
        \int d{\bf r}' \frac{\phi_i^{*}({\bf r}') \phi_k({\bf r}')}{\vert {\bf r}' - {\bf r}\vert}
        \right] \Bra{{\bf r}} \;.
\end{equation}
%
ERI's can be effectively eliminated from Eq.~\eqref{e:rep.murrell-etal.S1} 
by using the following prescription~\cite{Blasiak.Bednarska.Choluj.Bartkowiak.JCP.2019}
%
\begin{equation} \label{e:ft-reduction}
	\sum_t \sum_{ij}\sum_{kl\in A} {\mathcal{F}}_t\left[ 
   \BraKet{ij}{k^Al^A}
 \right] = \sum_{ij} \tBraKet{i}{\hat{v}_{\text{eff}}^A}{j} \;.
\end{equation}
%
where ${\mathcal{F}}_t$ is a certain and well defined functional of ERI's.
Application of Eq.~\eqref{e:ft-reduction} enables to define
a joint OEP
operator constructed from nuclear, Coulomb and exchange parts 
as
%
\begin{equation}
 V^A_{ij} + 
 \sum_{k\in A}^{\rm Occ} 
  \left\{ 2\BraKet{ij}{kk} - \BraKet{ik}{jk} \right\}
\equiv \sum_{\mu\in A} \tBraKet{\mu}{ 
\hat{v}^{A[\mu i]}_{\rm eff}
 }{j}
\end{equation}
%
with
%
\begin{equation}
 \hat{v}^{A[\mu i]}_{\rm eff} \equiv C_{\mu i} \hat{v}^A_{\rm nuc} + 
 \sum_{k\in A}^{\rm Occ} \left[
 2C_{\mu i}^A \hat{v}^A_{kk} - C_{\mu k}^A \hat{v}^A_{ik}
 \right] \;.
\end{equation}
%
%The nuclear part can also be incorporated within the effective potential
%resulting in the following reduction
%
%
On the other hand, it immediately follows that
%
\begin{equation}
 \hat{v}^{A[\mu i]}_{\rm eff} \Ket{\mu} = 
  \sum_{k\in A}^{\rm Occ} \left\{
     2\hat{v}^A_{kk} \Ket{i} - \hat{v}^A_{ik} \Ket{k}
  \right\} \;.
\end{equation}
%
For practical calculations, the right hand side of the above equation can be expanded 
in the auxiliary basis,
%
\begin{equation} \label{e:v-oep.rep}
  \sum_{k\in A}^{\rm Occ} \left\{
     2\hat{v}^A_{kk} \Ket{i} - \hat{v}^A_{ik} \Ket{k}
  \right\} \cong
  \sum_{\xi\in A}^{\rm DF} 
  V_{\xi i}^A \Ket{\xi} \;,
\end{equation}
%
where the matrix ${\bf V}^A$ can be considered as effective fragment parameters.
Doing the same operations on the twin operators associated with the molecule $B$
original theory of Murrell et. al from Eq.~\eqref{e:rep.murrell-etal.S1}
reduces to
%
\begin{equation} \label{e:rep.murrell-etal.S1.oep}
    E^{\rm Rep}(S^{-1}) \cong 
 -2\sum_{i\in A}^{\rm Occ} \sum_{j\in B}^{\rm Occ}
               S_{ij} \Big\{
           \sum_{\xi \in A}^{\rm DF} V_{\xi i}^A S_{\xi j}
         + \sum_{\eta\in B}^{\rm DF} V_{\eta j}^B S_{\eta i}
                \Big\} \;,
\end{equation}
%
where 
effective fragment parameters
%the matrix elements $V_{\xi i}^A$ 
can be obtained 
from the extended density fitting\cite{Blasiak.Bednarska.Choluj.Bartkowiak.JCP.2019} (EDF)
by
%
\begin{equation} \label{e:rep.murrell-etal.S1.oep-edf}
            V_{\xi i}^X = \sum_{\eta\in X}^{\rm DF} 
                          \sum_{\varepsilon,\zeta\in X}^{\rm RI}
%                         \sum_{\zeta\in A}^{\rm RI}
                          \left[ {\bf R}^{-1} \right]_{\xi\eta} R_{\eta\varepsilon} 
                          \left[ {\bf S}^{-1} \right]_{\varepsilon\zeta} 
                          \tBraKet{\zeta}{\hat{v}_{\rm eff}^{X[\zeta i]}}{i}
                \;.
\end{equation}
%
In the procedure, the required OEP matrix elements can be calculated from
%
\begin{equation} \label{e:rep.murrell-etal.S1.oep-v}
   \tBraKet{\zeta}{\hat{v}_{\rm eff}^{X[\zeta i]}}{i}
     = \sum_{x\in X}^{\rm At} W_{\zeta i}^{(x)} 
        + \sum_{\beta\gamma\delta\in X}^{\rm AO}
           \left\{ 
             2 C_{\beta i}^X D_{\gamma\delta}^X - C_{\gamma i}^X D_{\beta \delta}^X
           \right\}
           \BraKet{\zeta\beta}{\gamma\delta} \;,
\end{equation}
%
where ${\bf D}^X$ and ${\bf C}^X$ is the one\hyp{}particle AO density and
the LCAO\hyp{}MO matrices of isolated molecule $X$, respectively.
%where $\{\phi_\alpha\}$ is the test basis set, assumed here to be identical
%to the intermediate basis set necessary to perform the extended density fitting (EDF).
%
Eq.~\eqref{e:rep.murrell-etal.S1.oep} and Eq.~\eqref{e:rep-efp2.s1} have almost
the same form, with two exceptions in the new OEP formulation: 
(i) only overlap integrals, which are relatively computationally inexpensive,
need to be evaluated, instead of additional kinetic energy integrals in the EFP2
formulation that are slightly more expensive;
(ii) overlap integrals need to be evaluated also between
auxiliary basis set. The smaller the size of the auxiliary basis, the less expensive
evaluation of the $E^{\rm Rep}(S^{-1})$ becomes. 
The alternative EFP2 formulation %with OEP\hyp{}based first\hyp{}order term
is therefore still given by Eq.~\eqref{e:exr-efp2}
%
%\begin{equation} \label{e:exrep-new}
%E^{\rm Ex-Rep} \approx E^{\rm Ex} + E^{\rm Rep}_{\rm OEP} + E^{\rm Rep}(S^{-2}) \;,
%\end{equation}
%
but with $E^{\rm Rep}(S^{-1})$ replaced by OEP\hyp{}based
first\hyp{}order term from Eq.~\eqref{e:rep.murrell-etal.S1.oep}.
%with $E^{\rm Ex}$ given by Eq.~\eqref{e:exc-efp2}, 
%$E^{\rm Rep}(S^{-1})$ given by Eq.~\eqref{e:rep.murrell-etal.S1.oep} and
%$E^{\rm Rep}(S^{-2})$ given by Eq.~\eqref{e:rep-efp2.s2}.
%\section{\label{s:5.results}Results and Discussion}

We implemented EXR energy models from Eq.~\eqref{e:exr-efp2}
(EFP2), Eq.~\eqref{e:rep.murrell-etal.S1.oep} (OEP), as well as the repulsion term of Murrell et~al.
from Eq.~\eqref{e:rep.murrell-etal.S1} (here abbreviated as S1-Ref) for validation of the EDF technique.
Benchmark EXR energy were estimated from the density decomposition scheme 
at the Hartree\hyp{}Fock level of theory (here referred as to the DDS/HF) 
developed by Mandado and Hermida\hyp{}Ram{\'o}n,\cite{Mandado.Hermida-Ramon.JCTC.2011} 
which is equivalent to the exact EXR energy 
in the intermolecular perturbation theory with exchange of Hayes and Stone
at this level of theory.\cite{Hayes.Stone.MolPhys.1984} 
To localize molecular orbitals, procedure that is necessary for the EFP2 calculations,
the Boys method\cite{Boys.RevModPhys.1960} was used. Note here that, since
Murrell et~al.'s theory is invariant with respect to unitary transformation
of molecular orbitals, Eq.~\eqref{e:rep.murrell-etal.S1.oep} is also invariant and does not require
orbital localization. 
%in which
%
%\begin{equation}\label{e:dds-exr}
% \Delta E^{\rm Ex-Rep}_{\rm DDS/HF} = 
%\end{equation}
%

Three complexes: 
(i) (H$_2$O)$_2$, 
(ii) H$_2$O--HOCH$_3$, 
(iii) (DMSO)$_2$,
were chosen as model systems to analyze the asymptotic dependence 
of EXR energy, which is shown in Figure~XXX. 
The reference (zero\hyp{}displacement) geometries
were obtained as described in Ref.\cite{Blasiak.Bednarska.Choluj.Bartkowiak.JCP.2019} 
and the structures  
along with the translation direction are depicted
in the insets in Figure~XXX.
Energy\hyp{}optimizations were performed at the HF/6-31+G(d,p) level,
as implemented in 
the {\sc Gaussian16} quantum chemistry program package.\cite{Gaussian16}
Small auxiliary basis sets for H$_2$O, CH$_3$OH and DMSO were developed
to minimize the computational cost of evaluating the summation terms
from Eq.~\eqref{e:rep.murrell-etal.S1.oep}. For this purpose, the basis set optimization 
as described in Appendix~A to Ref.~\cite{Blasiak.Bednarska.Choluj.Bartkowiak.JCP.2019} was implemented
in conjunction with the basin hopping global optimization algorithm,  %cite
available in the SciPy Python module. %cite

{
\renewcommand{\arraystretch}{1.4}
\begin{table}[h]
\caption[]
{{\bf CPU timings in miliseconds of exchange\hyp{}repulsion single point 
energy calculations via various methods\footnotemark[1]}
}
\label{t:5}
\begin{ruledtabular}
\begin{tabular}{llccccccccc}
                      && \multicolumn{2}{c}{(H$_2$O)$_2$\footnotemark[2]} 
                      && \multicolumn{2}{c}{H$_2$O--HOCH$_3$\footnotemark[2]} 
                      && \multicolumn{2}{c}{(DMSO)$_2$\footnotemark[3]} \\
\cline{3-4}
\cline{6-7}
\cline{9-10}
Murrell et al.\footnotemark[4] && 3$\times 10^4$  & (10.0)  
                               && 3$\times 10^4$  & (10.0)
                               && 3$\times 10^4$  & (10.0) \\
Hayes-Stone\footnotemark[5]    && 3$\times 10^4$  & (10.0)  
                               && 3$\times 10^4$  & (10.0)
                               && 3$\times 10^4$  & (10.0) \\
EFP2                           && 3$\times 10^4$  & (10.0)  
                               && 3$\times 10^4$  & (10.0)
                               && 3$\times 10^4$  & (10.0) \\
OEP/large\footnotemark[6]      && 3$\times 10^4$  & (10.0)  
                               && 3$\times 10^4$  & (10.0)
                               && 3$\times 10^4$  & (10.0) \\
OEP/mini\footnotemark[6]       && 3$\times 10^4$  & (10.0)  
                               && 3$\times 10^4$  & (10.0)
                               && 3$\times 10^4$  & (10.0) \\
\end{tabular}
\end{ruledtabular}
%
\footnotetext[1]{1.2 GHz AMD EPYC\texttrademark{} 7301 16-Core Processor, calculations performed on 1 core. 
Exchange\hyp{}repulsion energies are given in parentheses for reference (kcal/mol).
See also the implementation details in the main text.}
\footnotetext[2]{Primary basis set: 6-311++G(d,p).}
\footnotetext[3]{Primary basis set: 6-31++G(d,p).}
\footnotetext[4]{Reference~\cite{Murrell.Randic.Williams.Longuet-Higgins.ProcRSocLondA.1965}.}
\footnotetext[5]{Reference~\cite{Hayes.Stone.MolPhys.1984}.}
\footnotetext[6]{Eq.~\eqref{}. Auxiliary basis sets for the EDF in $E^{\rm Rep}(S^{-1})$ 
from Eq.~\eqref{e:rep.murrell-etal.S1.oep}
are: `large' - aug-cc-pVDZ-jkfit for (H$_2$O)$_2$ and H$_2$O--HOCH$_3$ systems as well as
aug-cc-pVDZ-ri for (DMSO)$_2$ system;
`mini' - optimized small basis sets (see Supporting Information).}
%
\end{table}
}
%




To investigate the accuracy of the OEP\hyp{}based repulsion
term across
%perform statistical error analysis for 
a variety of
interacting systems, databases
of bi\hyp{}molecular complexes in the non\hyp{}covalent
interactions database NCB31 developed by the Truhlar's 
group,\cite{Zhao.Schultz.Truhlar.JCTC.2006,
Zhao.Truhlar.JCTC.2005,Zhao.Schultz.Truhlar.JCTC.2006,Zhao.Schultz.Truhlar.JCP.2005}
as well as the BioFragment Database subset BBI for backbone\hyp{}backbone
interactions in proteins,\cite{Burns.Faver.Zheng.Marshall.Smith.Vanommeslaeghe.MacKerell.Merz.Sherrill.JCP.2017} 
as implemented in the {\sc Psi4}
program,\cite{Psi4.JCTC.2017}
were utilized as sources of equilibrium structures. 
In Figure~YYY we show statistical error analysis...

%In particular, the subsets from the NCB31 database were separately considered:
%the HB6/04 hydrogen bonding database,\cite{Zhao.Truhlar.JCTC.2005,Zhao.Schultz.Truhlar.JCTC.2006,Zhao.Schultz.Truhlar.JCP.2005}
%the DI6/04 dipoleinteraction database,\cite{Zhao.Truhlar.JCTC.2005,Zhao.Schultz.Truhlar.JCTC.2006,Zhao.Schultz.Truhlar.JCP.2005}
%the CT7/04 charge-transfer complex database,\cite{Zhao.Truhlar.JCTC.2005,Zhao.Schultz.Truhlar.JCTC.2006,Zhao.Schultz.Truhlar.JCP.2005}
%the WI7/05 weak interaction database,\cite{Zhao.Schultz.Truhlar.JCTC.2006,Zhao.Schultz.Truhlar.JCP.2005,Zhao.Truhlar.JPCA.2005}
%and PPS5/05 the $\pi$-$\pi$ stacking database.\cite{Zhao.Schultz.Truhlar.JCTC.2006,Zhao.Schultz.Truhlar.JCP.2005,Zhao.Truhlar.JPCA.2005}





\section{\label{s:6.conclusions}Summary and a few concluding remarks}

Therefore, the following model is proposed for the
total exchange\hyp{}repulsion energy in the EFP2 model: 
(i) $E^{\rm Ex}$ and $E^{\rm Rep}(S^{-2})$ are evaluated
from the original EFP2 formulae of Jensen and Gordon, i.e.,
from Eqs.~\eqref{e:exc-efp2} and \eqref{e:rep-efp2.s2}, respectively;
(ii) $E^{\rm Rep}(S^{-1})$ term is evaluated from Eq.~\eqref{e:rep.murrell-etal.S1.oep}
and assuming a small auxiliary basis sets, optimized for each
EFP2 fragment separately.




\begin{acknowledgments}
This project is carried out under POLONEZ programme which has received funding from the European Union's
Horizon~2020 research and innovation programme under the Marie Sk{\l}odowska-Curie grant agreement 
No.~665778. This project is funded by National Science Centre, Poland 
(grant~no. 2016/23/P/ST4/01720) within the POLONEZ 3 fellowship.
\end{acknowledgments}

%%


% -----------------------
\bibliography{references}
% -----------------------

\end{document}
