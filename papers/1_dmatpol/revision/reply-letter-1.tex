\documentclass[12pt]{article}
%\usepackage{a4wide}
\usepackage[utf8]{inputenc}
\usepackage{lipsum} % to generate some filler text
\usepackage{fullpage}
\usepackage{xcolor}
%\usepackage[a4paper, margin=2.7cm]{geometry}
\usepackage{amsmath, amsthm, amsfonts, amssymb}
\usepackage{mathrsfs}           % \mathscr font.

\title{Review of Manuscript ID: A18.08.0126}
\author{Bartosz B{\l}asiak}
% import Eq and Section references from the main manuscript where needed
% \usepackage{xr}
% \externaldocument{manuscript}

% package needed for optional arguments
\usepackage{xifthen}
% define counters for reviewers and their points
\newcounter{reviewer}
\setcounter{reviewer}{0}
\newcounter{point}[reviewer]
\setcounter{point}{0}

% This refines the format of how the reviewer/point reference will appear.
\renewcommand{\thepoint}{P\,\thereviewer.\arabic{point}}

% command declarations for reviewer points and our responses
\newcommand{\reviewersection}{\stepcounter{reviewer} \bigskip \hrule
 \section*{Reviewer \thereviewer}}

\newenvironment{point}
 {\refstepcounter{point} \bigskip \noindent {\textbf{Reviewer~Point~\thepoint} } ---\ }
 {\par }

\newcommand{\shortpoint}[1]{\refstepcounter{point} \bigskip \noindent 
 {\textbf{Reviewer~Point~\thepoint} } ---~#1\par }

\newenvironment{reply}
 {\medskip \noindent \begin{sf} \color{blue} \textbf{Reply}:\ }
 {\medskip \end{sf}}

\newcommand{\shortreply}[2][]{\medskip \noindent \begin{sf}\textbf{Reply}:\ #2
 \ifthenelse{\equal{#1}{}}{}{ \hfill \footnotesize (#1)}%
 \medskip \end{sf}}


\begin{document}
\maketitle
\section*{Response to the reviewers}

I would like to thank the Editor and two Reviewers for the time and insights given to my manuscript 
and for supporting the work to be published in \emph{J. Chem. Phys.}
I appreciate the positive and constructive feedback given, and agree with the Reviewers’ 
suggestions for revisions. The Reviewers’ comments have been fully addressed as outlined below. 
%In addition, I have also introduced some minor adjustments in the notation.
Now, I hope that the revised manuscript is suitable for publication in \emph{J. Chem. Phys.}

\clearpage
\reviewersection

Recommendation: Publish as is \\
New Potential Energy Surface: No \\
Reviewer  (Comments to the Author): \\

I think this manuscript is very interesting. The model developed provides a very good reference 
to be used in connection with testing of less sophisticated schemes, 
e.g. models where a quantum chemical description is coupled to a classical polarizable model 
for the environment. Also the suggested use of the model within the field of aggregation 
is very interesting. As such I recommend publication of this manuscript. 

\begin{point}
One very minor issue relates to the statement in the introduction saying 
that the EFP model is limited to either HF or DFT. Indeed the EFP model has been 
formulated also within correlated wave function approaches like CC. 
 \label{pt:rev-1}
\end{point}

\begin{reply}
Thanks for pointing this out. 
On the one hand, the \emph{ab initio} parameters derived from the EFP methods
can be obtained by using only HF or DFT wavefunctions, 
because no correlated wavefunction formulation of EFP method
was developed so far. %However, I agree with the Reviewer that the EFP method is not limited to only HF or DFT
On the other hand, I agree with the Reviewer that the EFP method is not limited to only HF or DFT
in a sense that it is possible to use it together with correlated wavefunction methods like CC
in a hybrid fashion (such as CC-EOM/EFP method). 
%In such circumstances, the QM method
%describes electronically excited states, while the EFP method is used for ground-state solvent molecules.
Therefore, to clarify this issue, the following sentence (shown in bold font) was added on Page 3: 
\begin{quote}
While they [the EFP methods] can approximately describe the environment composed of molecules in their ground states,
they cannot be used for highly correlated wavefunctions such as electronically excited states. 
\textbf{\noindent
In such cases, the full QM method needs to be utilized on the excited state wavefunction, that is 
embedded within the EFP environment through the electrostatic 
or polarizable embedding. [D. Ghosh, \emph{J. Chem. Phys. A} {\bf 121}, 741 (2017).] }
\end{quote}
\end{reply}

\clearpage
\reviewersection
Recommendation: Revision \\
New Potential Energy Surface: No \\
Comments to the Author: \\

The paper entitled `One-Particle Density Matrix Polarization Susceptibility Tensors' 
presents a new way to describe electric field induced changes in a molecule by the mean of 
one-particle density matrix susceptibilities (DMS) interacting with a spacial distribution 
of electric field. The author applied this method to the water molecule considering uniform 
and non-uniform electric-field distributions and showed that the generalized DMS model 
is able to reproduce electric-field induced changes in the polarization energy, dipole moment 
and quadrupole moment characterized at the Hartree-Fock level of theory. This study 
is the first step prior to provide an efficient effective one-electron fragment potential 
to reduce the cost of quantum chemistry calculations in condensed phase. This is 
an interesting article that is suitable to be published in the Journal of Chemical Physics. 
Though, I have some minor comments/questions that should be answered prior to the publication. 

\begin{point}
When I finished reading this MS, I was left hungering for a bit more. Since this DMS scheme 
could be a more precise alternative to multipole-based approaches to describe aggregates, 
I think an application on some small water aggregates (or dimer of water) and comparing 
to full HF calculations could be nice addition. But I understand that this method is still 
in its early stage and that this is probably not yet possible. In any case is a bit more 
testing on other systems may be also of mixed polarity would improve the quality of the MS. 
 \label{pt:rev-2.1}
\end{point}

\begin{reply}
I definitely agree with the Reviewer that an additional illustration on some molecular
dimer would much increase the quality of the present manuscript. On the other hand, as the Reviewer also
understands, this requires a more elaborate theoretical section since other effects such as Pauli exchange-repulsion
need to be included and combined with polarization in order to compare to the full HF result. 
After an attempt to extend the scope of this manucript by the above illustration,
I decided to write a separate manuscript on that issue in the future, because the additional
material would distract the reader from the key focus of this article and introduce more aspects
to be separately discussed (effects of exchange-repulsion and intermolecular charge transfer, 
the choice of definition of induction energy
within a dimer and so on). 
\end{reply}

\begin{point}
The author should discuss what is the computational cost and applicability (automatization) 
of this method since quite a lot of calculations for all the spatial distributions are needed for fitting the DMS. 
 \label{pt:rev-2.2}
\end{point}

\begin{reply}
To discuss these aspects in more detail, I have added one subsection entitled
\emph{Computational cost and automatization of the generalized DMS model}
within the Results and Discussion section (IV.C) on Page 20.
\end{reply}

\begin{point}
On page 14 it is written in section IV.A.: 
`Polarization under the uniform electric field is the simplest type 
of electrostatic perturbation whereby it is possible to compute 
the exact DMS according to Eq. (5) directly by numerical differentiation.' 
Note that this is also possible analytically with the CPHF method. 
 \label{pt:rev-2.3}
\end{point}

\begin{reply}
Thanks. I changed the sentence on Page 14 as follows:
\begin{quote}
Polarization under the uniform electric field is the simplest type of electrostatic perturbation
whereby it is possible to compute the exact DMS according to Eq.~(5)
\textbf{\noindent
by using either the coupled-perturbed Hartree-Fock method [P. R. Horn and M. Head-Gordon, \emph{J. Chem. Phys.} {\bf 143}, 114111 (2015).]
or numerical differentiation.}
\end{quote}
\end{reply}


\end{document}



