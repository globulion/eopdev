%% ****** Start of file aiptemplate.tex ****** %
%%
%%   This file is part of the files in the distribution of AIP substyles for REVTeX4.
%%   Version 4.1 of 9 October 2009.
%%
%
% This is a template for producing documents for use with 
% the REVTEX 4.1 document class and the AIP substyles.
% 
% Copy this file to another name and then work on that file.
% That way, you always have this original template file to use.

\documentclass[aip,graphicx]{revtex4-1}
%\documentclass[aip,reprint]{revtex4-1}

% hyphenation
\usepackage{hyphenat}
%\usepackage{lipsum}

% tables
%\usepackage{dcolumn}
%\usepackage{multirow}
%\usepackage{hhline}
%\usepackage{booktabs}
%\usepackage{longtable}

% mathematics
\usepackage{amsmath}
\usepackage{amsfonts}
\usepackage{amssymb}
\usepackage{amsbsy}
\usepackage{bm}
\usepackage{mathrsfs}
\usepackage{upgreek}
%\allowdisplaybreaks

% symbols
%\usepackage{gensymb}
%\usepackage{textcomp}

%---------------------------------------------------
% happy integral
\newcommand{\rint}[1]{\mbox{\Large $ \int\limits_{\mbox{\tiny  $#1$}}$}}
% SHORTCUTS
%\newcolumntype{,}{D{.}{,}{2}}
\newcommand{\citee}[1]{\ensuremath{\scriptsize^{\citenum{#1}}}}
\newcommand{\HRule}{\rule{\linewidth}{0.2mm}}
% Quantum notation
\newcommand{\bra}[1]{\ensuremath{\bigl\langle {#1} \bigl\lvert}}
\newcommand{\ket}[1]{\ensuremath{\bigr\rvert {#1} \bigr\rangle}}
\newcommand{\braket}[2]{\ensuremath{\bigl\langle {#1} \bigl\lvert {#2} \bigr\rangle}}
\newcommand{\tbraket}[3]{\ensuremath{\bigl\langle {#1} \bigl\lvert {#2} \bigl\lvert {#3} \bigr\rangle}}
% Math
\newcommand{\pd}{\ensuremath{\partial}}
\newcommand{\DR}{\ensuremath{{\rm d} {\bf r}}}
%\newcommand{\BM}[1]{\ensuremath{\mbox{\boldmath${#1}$}}}
\newcommand{\BM}[1]{\bm{#1}}
% Chemistry (formulas)
\newcommand{\ch}[2]{\ensuremath{\mathrm{#1}_{#2}}}
% Math 
\newcommand{\VEC}[1]{\ensuremath{\mathrm{\mathbf{#1}}}}
% vector nabla
\newcommand{\Nabla}{\ensuremath{ \BM{\nabla}}}
% derivative
\newcommand{\FDer}[3]{\ensuremath{
\bigg(
\frac{\partial #1}{\partial #2}
\bigg)_{#3}}}
% diagonal second derivative
\newcommand{\SDer}[3]{\ensuremath{
\biggl(
\frac{\partial^2 #1}{\partial #2^2}
\biggr)_{#3}}}
% off-diagonal second derivative
\newcommand{\SSDer}[4]{\ensuremath{
\biggl(
\frac{\partial^2 #1}{\partial #2 \partial #3}
\biggr)_{#4}}}
% derivatives without bound
% derivative
\newcommand{\fderiv}[2]{\ensuremath{
\frac{\partial #1}{\partial #2}}}
% diagonal second derivative
\newcommand{\sderiv}[2]{\ensuremath{
\frac{\partial^2 #1}{\partial #2^2}
}}
% off-diagonal second derivative
\newcommand{\sderivd}[3]{\ensuremath{
\frac{\partial^2 #1}{\partial #2 \partial #3}
}}
% derivatives for tables
\newcommand{\fderivm}[2]{\ensuremath{
{\partial #1}/{\partial #2}}}
% diagonal second derivative
\newcommand{\sderivm}[2]{\ensuremath{
{\partial^2 #1}/{\partial #2^2}
}}
% off-diagonal second derivative
\newcommand{\sderivdm}[3]{\ensuremath{
{\partial^2 #1}/{\partial #2 \partial #3}
}}
% ERIs and OEIs
\newcommand{\OEIc}[3]{\ensuremath{\left(#1 \lvert #2 \rvert #3 \right)}}
\newcommand{\ERIc}[4]{\ensuremath{\left(#1 #2 \vert #3 #4 \right)}}

% Partial density and potential
\newcommand{\PartPot}[4]{\ensuremath{\frac{#1 #2}{\lvert #3-#4 \rvert }}}

% trace operator
\DeclareMathOperator{\Tr}{Tr}

\draft % marks overfull lines with a black rule on the right

\begin{document}

\title{Density Matrix Susceptibility} %Title of paper

\author{Bartosz B{\l}asiak}
\email[]{blasiak.bartosz@gmail.com}
\homepage[]{https://www.polonez.pwr.edu.pl}
\thanks{Bulak}
\affiliation{Department of Physical and Quantum Chemistry, Faculty of Chemistry, Wroc{\l}aw University of Technology, Wybrze{\.z}e Wyspia{\'n}kiego 27, Wroc{\l}aw 50-370, Poland}

\date{\today}

\begin{abstract}
To be filled ...
\end{abstract}

\pacs{}% insert suggested PACS numbers in braces on next line

\maketitle %\maketitle must follow title, authors, abstract and \pacs

\section{\label{s:1}Introduction}

\section{\label{s:2}Theory}

The electronic energy of a closed\hyp{}shell molecule $A$ is given by the time\hyp{}independent Schr{\"o}dinger equation,
%
\begin{equation}
 E^{(0)} = \tbraket{\Psi^{(0)}}{\mathscr{H}^{(0)}}{\Psi^{(0)}} \;,
\end{equation}
%
where $E$ stands for the total electronic energy, $\Psi$ the electronic wavefunction and $\mathscr{H}$
the Hamiltonian operator. In the above, the superscript $(0)$
denotes the reference, or unperturbed, state that is assumed to be known.

Under the influence of an external electrostatic perturbation, the total electronic energy
becomes
%
\begin{equation}
 E[{\bf F}({\bf r})] = \tbraket{\Psi}{\mathscr{H}^{(0)} + \mathscr{V}({\bf F}({\bf r}))}{\Psi}
\end{equation}
%
in which ${\bf F}({\bf r})$ is a non\hyp{}uniform electric field distribution.
The electric field\hyp{}induced energy change is an unknown functional 
of the associated change in the one\hyp{}electron density,
%
\begin{equation}\label{e:rho}
 \rho({\bf r}) = 2\sum_{\alpha\beta} D_{\alpha\beta} \phi_\alpha({\bf r}) \phi^*_\beta({\bf r}) \;.
\end{equation}
%
In Eq.~\eqref{e:rho} $\phi_\alpha({\bf r})$ is a basis funtion, which can be for instance 
an atomic orbital (AO) or a molecular orbital (MO), and $D_{\alpha\beta}$ is the corresponding
one\hyp{}particle density matrix.
If perturbation\hyp{}independent basis is chosen, the electric field\hyp{}induced change in the one\hyp{}particle density
is given by
%
\begin{equation}\label{e:rho}
 \delta \rho({\bf r}) = 2\sum_{\alpha\beta} \delta D_{\alpha\beta} \phi_\alpha({\bf r}) \phi^*_\beta({\bf r}) \;.
\end{equation}
%
Ideally, the electric field\hyp{}induced change in the ground state's one\hyp{}particle density 
follows its energy minimum of the perturbed state. In addition, all the perturbed 
excited state one\hyp{}electron densities must be excited state solutions of the Schr{\"o}dinger equation
with the exact energy\hyp{}minimized ground state. 

Development of a first\hyp{}principles theory
that is capable of accurately capturing the electric field\hyp{}dependence of the one\hyp{}particle density
is extremely difficult. This is not only because of the very complicated (and generally unknown) solution
to the electronic correlation problem, but also due to the dimensionality of the problem: the quantitative
model has to accurately predict all the $n^2$ elements of the density matrix. Even relatively small errors
in a particular matrix element could accumulate tremendously when computing traces with other matrices,
and in turn, obstruct the evaluated properties with non\hyp{}negligible errors.
Therefore, in this Work we adopt the simplistic approach: we first develop a qualitative first\hyp{}principles model 
at the Hartree\hyp{}Fock (HF)
level of theory, and then generalize it to the arbitrary level using the functional form that was found 
by considering HF level,
but treating the model as a set of adjustable parameters further on.

\subsection{Ab Initio Density Matrix Susceptibilities: Hartree-Fock Level}

\subsubsection{Induced Dipole Moment}

It seems intuitive that $\delta {\bf D}$ should 
be directly related to the
polarization\hyp{}induced multipole moments because they are
a measure of the distortion of the electron density distribution due to the external electric
field. 
%Unfortunately, such a mathematical problem is very complicated due to its high\hyp{}order tensor nature
%with respect to the unknown $\delta {\bf D}$ and no approximate analytical solutions have been proposed
%up to date. Therefore, in this paper we adopt a somewhat different
%approach that will require \emph{a priori} knowledge from training set of known cases.
Therefore, 
we start our analysis by considering the induced electronic dipole moment 
of a closed-shell molecule, %which is the first non\hyp{}vanishing moment,
%
\begin{equation} \label{e:dmu-exact}
 \delta {\BM{\upmu}}({\bf r}_Q) = 
     2 \Tr{ 
         \left[ 
              \delta {\bf D} {\mathbb{M}({\bf r}_Q)}
         \right] } \;.
\end{equation}
%
where ${\mathbb{M}({\bf r}_Q)}$ is AO\hyp{}representation
of the dipole moment operator defined with respect to origin at ${\bf r}_Q$,
%
\begin{equation}\label{e:m}
 \left[ {\mathbb{M}({\bf r}_Q)} \right]_{\alpha\beta} = \tbraket{\alpha}{{\bf r} - {\bf r}_Q}{\beta} \;.
\end{equation}
%
Note that if the total charge is conserved, the induced moments are independent on the origin.

In the Hartree\hyp{}Fock theory, ${\bf D}$ is approximated as
%
\begin{equation}
 {\bf D} = {\bf C}{\bf C}^\dagger \;,
\end{equation}
%
where ${\bf C}$ describes the occupied molecular orbitals as linear combinations
of atomic orbitals (AO's), referred here to as the LCAO-MO matrix.
When ${\bf D}$ is expressed in orthogonal basis, the density matrix is idempotent, i.e.,
%
\begin{equation}
 {\bf D}^2 = {\bf D} \;.
\end{equation}
%
The change in the molecular orbitals can be parameterized as suggested by McWeeny\cite{McWeeny.RevModPhys.1960}
by defining the idempotency\hyp{}preserving variation
%
\begin{equation}
 \delta {\bf C} = {\BM\Delta} \cdot {\bf C}^{(0)} \;,
\end{equation}
%
where $\BM\Delta$ is a square non\hyp{}singular and non\hyp{}symmetric matrix.
McWeeny has shown that
%
\begin{equation} \label{e:dmat-change.exact}
 \delta {\bf D} = -{\bf D}^{(0)} + \left[ {\bf D}^{(0)} + {\bf v} \right]
                                   \left[ {\bf 1} + {\bf v}^\dagger{\bf v} \right]^{-1}
                                   \left[ {\bf D}^{(0)} + {\bf v}^\dagger \right] \;,
\end{equation}
%
where the auxiliary vector ${\bf v}$ is defined as
%
\begin{equation}
 {\bf v} \equiv \left[ {\bf 1} - {\bf D}^{(0)} \right] {\BM\Delta} {\bf D}^{(0)}  \;.
\end{equation}
%
Expanding Eq.~\eqref{e:dmat-change.exact} in a Taylor series around ${\bf v}={\bf 0}$ and
truncating it at linear terms with respect to ${\BM\Delta}$ gives
%
\begin{subequations} 
 \begin{align}
 \delta {\bf D} &\cong \left[ {\bf 1} - {\bf D}^{(0)} \right] {\BM\Delta} {\bf D}^{(0)} + 
                        {\bf D}^{(0)} {\BM\Delta}^\dagger \left[ {\bf 1} - {\bf D}^{(0)} \right]  
 \label{e:dD-1} \\  &= 
  \delta {\bf C}  {\bf C}^{(0)\dagger} + {\bf C}^{(0)}\delta {\bf C}^\dagger
           - {\bf D}^{(0)} \delta {\bf C}  {\bf C}^{{(0)}\dagger} - {\bf C}^{(0)}\delta {\bf C}^\dagger {\bf D}^{(0)} 
 \label{e:dD-2}
 \end{align}
\end{subequations}
%
The latter equality is true only if the basis set functions are orthogonal, i.e., 
${\bf C}^{(0)\dagger}{\bf C}^{(0)}={\bf 1}$,
which will be the case throughout the entire work, unless stated differently.
Substituting the result from Eq.~\eqref{e:dD-1}
into Eq.~\eqref{e:dmu-exact} leads to
%%
%\begin{equation} \label{e:delta-Taylor}
% \delta {\bf D} = \left[ {\bf v} + {\bf v}^\dagger \right]
%                + \left[ {\bf v}{\bf v}^\dagger - {\bf v}^\dagger{\bf v}\right] + \ldots 
%\end{equation}
%
\begin{equation} \label{e:dmu-4-exact.linear-approximation}
 \frac{1}{2} 
 \delta {\BM{\upmu}}
  \cong
   \Tr{ 
    \left[ 
         {\mathbb{M}} {\BM\Delta} {\bf D}^{(0)}  
    \right] }
%
  +\Tr{ 
    \left[ 
         {\mathbb{M}} {\bf D}^{(0)} {\BM\Delta}^{\dagger}
    \right] }
%
  -\Tr{ 
    \left[ 
         {\mathbb{M}} {\bf D}^{(0)} {\BM\Delta} {\bf D}^{(0)}
    \right] }
%
  -\Tr{ 
    \left[ 
         {\mathbb{M}} {\bf D}^{(0)} {\BM\Delta}^{\dagger} {\bf D}^{(0)}
    \right] } \;,
\end{equation}
%
It can be proven (see Appendix~\ref{a:orig-dep}) that such obtained
induced dipole moment is independent on the origin. Therefore,
in the above result and also in later course of the Paper the dipole origin has been omitted. 
%where the dipole origin was omitted for notational clarity.
If the AO basis is orthogonal, Eq.~\eqref{e:dmu-4-exact.linear-approximation} can be simplified by realising that
%
\begin{subequations}
 \begin{align}
  \Tr{ 
    \left[ 
         {\mathbb{M}} {\BM\Delta} {\bf D}^{(0)}  
    \right] }
%
  +\Tr{ 
    \left[ 
         {\mathbb{M}} {\bf D}^{(0)} {\BM\Delta}^{\dagger}
    \right] }
  &=
2 \Tr{ 
    \left[ 
         \widetilde{\mathbb{M}} \cdot \delta {\bf C}
   \right] }  \;,
%
\\
%
  \Tr{ 
    \left[ 
         {\mathbb{M}} {\bf D}^{(0)} {\BM\Delta} {\bf D}^{(0)}
    \right] }
%
 +\Tr{ 
    \left[ 
         {\mathbb{M}} {\bf D}^{(0)} {\BM\Delta}^{\dagger} {\bf D}^{(0)}
    \right] }
  &=
2 \Tr{ 
    \left[ 
         \widetilde{\mathbb{K}} \cdot \delta {\bf C}
   \right] } \;.
%
 \end{align}
\end{subequations}
%
Here we introduced the auxiliary tensors
%
\begin{subequations}
 \begin{align}
   \widetilde{\mathbb{M}}  &\equiv {\bf C}^\dagger {\mathbb{M}}     \;,           \\
   \widetilde{\mathbb{K}}  &\equiv {\bf C}^\dagger {\mathbb{M}} {\bf D}^{(0)} \;, \\
   \widetilde{\mathbb{L}}  &\equiv \widetilde{\mathbb{M}} - \widetilde{\mathbb{K}} \;.
 \end{align}
\end{subequations}
%
to simplify the notation. Hence, we have
%
\begin{equation} \label{e:dmu-l-vector}
  \frac{1}{4} 
 \delta {\BM{\upmu}} %({\bf r}_0) 
   =
   \Tr{ 
    \left[ 
         \widetilde{\mathbb{L}} \cdot \delta {\bf C}
    \right] }
   %
   \equiv \sum_{i}^{\rm occ} \sum_\alpha {\bf l}_{i\alpha} \delta C_{\alpha i} \;,
\end{equation}
%
where the Cartesian vector
%
\begin{equation}\label{e:l-vector}
 {\bf l}_{i\alpha} \equiv \left[ \widetilde{\mathbb{M}} - \widetilde{\mathbb{K}} \right]_{i\alpha} 
      = \left[  {\bf C}^\dagger \mathbb{M} \left[ \bf{1} - \bf{D}^{(0)} \right] \right]_{i\alpha}
\end{equation}
%
and $i$ runs over occupied unperturbed MO's.
Note that the intermediate quantity $\widetilde{\mathbb{L}}$ contains the projector onto
the unoccupied MO space at the absence of the external electric field. It means that the
perturbed solution is constructed from the unperturbed virtual orbitals as well.
Although Eq.~\eqref{e:dmu-l-vector}
is exact up to first order in ${\BM\Delta}$, it cannot be inverted at this moment to obtain $\delta {\bf C}$
due to the presence of trace operation, unless the electric field is uniform.
%in which ${\bf D}^{(0)}$ is the unperturbed density matrix of the system whereas
%$\delta {\bf D}$ is the associated change due to the external electric field distribution. 
%In fact, $\delta {\bf D}$ is a complicated
%but unknown function of ${\bf F}({\bf r})$ and basis set. 
%Moreover, it certainly depends on the choice of
%molecular orbitals because the unitary transformation . 
Let us partition the induced dipole moment into separate contributions
from the occupied MO's, i.e.,
%
\begin{equation} \label{e:dmu-part}
 \delta {\BM{\upmu}} = \sum_i^{\rm occ} \delta {\BM{\upmu}}_i
\end{equation}
%
where
%
\begin{equation} \label{e:dmu-dpol}
 \delta {\BM{\upmu}}_i = {\BM{\alpha}}_i \cdot {\bf F}({\bf r}_i) \;.
\end{equation}
%
In Eq.~\eqref{e:dmu-dpol}, ${\BM{\alpha}}_i$ is the distributed polarizability tensor
associated with the $i$th MO whereas ${\bf F}({\bf r}_i)$ is the electric field evaluated at the $i$th MO centroid of charge
${\bf r}_i = \tbraket{i}{\hat{\bf r}}{i}$.
We can now insert the above result into Eq.~\eqref{e:dmu-l-vector} 
and after recasting the right hand side in a matrix form
we get
%
\begin{equation} \label{e:dmu-l-vector-mo-transform-X}
 \frac{1}{4} {\BM{\alpha}}_i \cdot {\bf F}({\bf r}_i) 
   =
   \delta {\bf c}_i^T \cdot {\bf L}_i \;.
\end{equation}
%
Note that now ${\bf L}_i$ is a matrix of size $(n \times 3)$ and $\delta{\bf c}_i$ is just the $i$th column of the
change in the LCAO-MO matrix due to the polarization process.
%
%The above equation can be re-written in a matrix form as
%
%\begin{equation} 
% \left[ {\mathbb L} \right]_{\alpha \zeta \beta} = L_{\beta\alpha}^{(\zeta)}
%\end{equation}
%%
%of size $m \times 3 \times m$.
%Note that, in this notation, ${\BM{\upmu}}_i^T$ and $\overline{\BM\Delta}_i^T$ are row vectors
%whereas $\overline{\BM\Delta}_i^p$ is a column vector. 
The solution for $\delta {\bf C}$ is
%
\begin{equation} \label{e:mu-ind.matrix.linear.solution}
  \delta {\bf c}_i^T = \frac{1}{4}
            {\bf F}({\bf r}_i)  \cdot {\BM{\alpha}}_i \cdot 
                    \left[ {\bf L}_i  \right]^{-1}_{\rm Left} \;,
\end{equation}
%
where $\left[ {\bf L}_i  \right]^{-1}_{\rm Left}$ is a left inverse
of ${\bf L}_i$ matrix 
%
\begin{equation} 
      \left[ {\bf L}_i  \right]^{-1}_{\rm Left}   \equiv
       \left[ {\bf L}_i^T {\bf L}_i \right]^{-1} {\bf L}_i^T 
\end{equation}
%
and has size $(3\times n)$. Note that the \emph{right inverse} of ${\bf L}_i$
does not exist because ${\bf L}_i {\bf L}_i^T$ is singular.
%Clearly, in the above approximation, ${\bf X}$ depends on the choice
%of unperturbed orbitals.

\subsubsection{Density Matrix Change}

Now we can obtain the final expression for the density matrix change 
due to external electric field that reproduces the 
induced dipole moment of the molecule.
The change in the LCAO\hyp{}MO coefficient is parametereized as
%
\begin{equation} \label{e:dC-dmatpol}
 \delta C_{\alpha i} = {\bf b}_{\alpha i} \cdot {\bf F}({\bf r}_i)  \;,
\end{equation}
%
where the susceptibility tensor is defined as
%
\begin{equation} \label{e:susceptibility-b}
  b_{\alpha i;w} = \frac{1}{4} \sum_u^{x, y, z} \left[ {\BM{\alpha}}_i \right]_{uw}
   \left[ \left[ {\bf L}_i  \right]^{-1}_{\rm Left} \right]_{u;\alpha}  
\end{equation}
%
for $w=x,y,z$. The result from Eq.~\eqref{e:dC-dmatpol} can be now inserted into 
the expression for $\delta {\bf D}$ from Eq.~\eqref{e:dD-2} to finally give
%
\begin{equation}\label{e:final-model.HF}
 \delta D_{\alpha\beta} \approx \sum_i^{\rm occ} {\bf B}^{(i)}_{\alpha\beta} \cdot {\bf F}({\bf r}_i)  \;,
\end{equation}
%
where the \emph{density matrix polarization susceptibility tensor} is
%
\begin{equation}  \label{e:susceptibility-B}
 {\bf B}^{(i)}_{\alpha\beta} = \left\{
%
                               C_{\alpha i} {\bf b}_{\beta i} + C_{\beta i} {\bf b}_{\alpha i}
                               - \sum_\gamma 
                                 \left( 
               D_{\alpha\gamma} C_{\beta i} + D_{\beta\gamma} C_{\alpha i}
                                 \right)
                                           {\bf b}_{\gamma i}
%
\right\}  \;.
\end{equation}
%
Hence, the 3D vector ${\bf B}^{(i)}_{\alpha\beta}$ describes the polarization affinity
of the density matrix element associated with $\alpha$th and $\beta$th orthogonal AO's.
Transforming these vectors to a non\hyp{}orthogonal AO basis is straightforward.
It is emphasized here that all elements of ${\bf B}_{\alpha\beta}^{(i)}$
are unique properties of electronically unperturbed molecule. 

One very important feature of the susceptibility from Eq.~\eqref{e:final-model.HF} is that
the density matrix increment dependent {\emph only} by susceptibility element associated with the same
pair of AO's. Therefore, there is no mixing in the AO's, rather the increment is proportional to the 
vector susceptibility, different for each density matrix element.

\subsection{Generalized Density Matrix Susceptibility Model}

\subsubsection{Functional form of the model}

Eq.~\eqref{e:final-model.HF}, which is the final result of the previous section, 
can be used as a guideline to construct a general model of density matrix polarization.
Quadratic terms with respect to the electric field as well as linear terms with respect to
electric field gradient can be added to the formulation, which leads to
%
\begin{equation}\label{e:final-model.General}
 \delta D_{\alpha\beta} = \sum_{i }^M \left\{
                                      {\bf B}^{(i ; 10)}_{\alpha\beta} \cdot {\bf F}({\bf r}_i)  
%                       + \sum_{ij}^M {\bf B}^{(ij; 20)}_{\alpha\beta} : {\bf F}({\bf r}_i) \otimes {\bf F}({\bf r}_j)
                        +             {\bf B}^{(i ; 20)}_{\alpha\beta} : {\bf F}({\bf r}_i) \otimes {\bf F}({\bf r}_i)
                        +             {\bf B}^{(i ; 01)}_{\alpha\beta} : {\BM\nabla} \otimes {\bf F}({\bf r}_i) 
                        + \ldots \right\}
\end{equation}
%
%However, Eq.~\eqref{e:final-model} is an approximation because (i) it is based on the approximate
%relation in Eq.~\eqref{e:dmu-l-vector-mo-transform} and (ii) only total dipole moment (but not quadrupole
%and higher moments) is exactly
%reproduced by the current model. Due to the above reasons, the existence of ${\bf X}$ that
%reproduces the exact perturbed density matrix with vanishing error is not guaranteed. 
%Moreover, as it was pointed out before, ${\bf X}$ almost certainly depends on the external electric field. 
%Next section adresses the problem of finding the optimal ${\bf X}$ that is treated as an adjustable
%matrix.
In the above equation, the density matrix susceptibilities
${\bf B}^{(i; RP)}_{\alpha\beta}$, distributed over $M$ sites,
denote the $R$-th order response with respect to the electric field
and $P$-th order response with respect to the electric field gradient. Note that
${\bf B}^{(i; 10)}_{\alpha\beta}$ is a vector of length $3$ whereas ${\bf B}^{(i; 20)}_{\alpha\beta}$
and ${\bf B}^{(i; 01)}_{\alpha\beta}$ are $(3\times 3)$ matrices. Also, the distributed expansion centres
(labeled as $i$) are generalized to any point in the molecule.

\subsubsection{Determining the generalized susceptibility tensors}

The exact mathematical form of the generalized density matrix susceptibilities is unknown. 
The first suscebility from Eq.~\eqref{e:final-model.HF} is valid only under the Hartree\hyp{}Fock
approximation, with a guarantee that only the overall charge and the dipole moment, 
but not higher multipole moments, are correctly described.
Our strategy is to treat the susceptibilities from Eq.~\eqref{e:final-model.General}
as adjustable parameters, rather than try to derive them from first principles theory.

Let  
$\left\{ {\bf F}^{(1)}({\bf r}), {\bf F}^{(2)}({\bf r}), \ldots, {\bf F}^{(N)}({\bf r}), \ldots \right\}$ 
be a set of $N_{\rm max}$ distinct and randomly sampled 
spatially non\hyp{}uniform distributions of the electric field. It is assumed that
the exact difference one\hyp{}particle density matrices (with respect to the unperturbed state)
defined as
%
\begin{equation}
 \delta \overline{\bf D}^{(N)} \equiv \overline{\bf D}^{(N)} - \overline{\bf D}^{(0)}
\end{equation}
%
are known for each sample (overline symbolizes the exact estimate).
%Let us assume that we a $\delta \overline{\bf D}^{(N)}$ 
%is the $N$th known (exact) difference one\hyp{}particle density matrix
%that was computed for the electric field distribution ${\bf F}^{(N)}({\bf r})$ chosen randomly.
Now, for each pair of the AO indices we construct the following 
parameterization of Eq.~\eqref{e:final-model.General}:
%
\begin{equation}\label{e:final-model.General.Parameters}
 \delta D^{(N)} = \sum_{i }^M \left\{ 
                              \sum_u^{x,y,z} S^{[1]}_{iu} F_{iu}^{(N)} 
%               + \sum_{ij}^M \sum_{uw}^{x,y,z} \left( S^{[2]}_{iu} + S^{[2]}_{jw} \right) F_{iu}^{(N)} F_{ju}^{(N)}
                +             \sum_{uw}^{x,y,z} \left( S^{[2]}_{iu} + S^{[2]}_{iw} \right) F_{iu}^{(N)} F_{iu}^{(N)}
                +             \sum_{uw}^{x,y,z} \left( S^{[3]}_{iu} + S^{[3]}_{iw}\right) F_{iu,iw}^{(N)}
                        + \ldots \right\}
\end{equation}
%
(the Greek subscripts were omitted here for notational simplicity).
The multiple parameter blocks 
$S^{[1]}$, $S^{[2]}$ and $S^{[3]}$ that are associated with electric field and its gradients
appear in the first power, allowing for linear least\hyp{}squares regression.
Note that $B_u^{(i,(10))} = S^{[1]}_{iu}$, $B_{uw}^{(i,(20))} = S^{[2]}_{iu} + S^{[2]}_{iw}$
and $B_{uw}^{(i,(01))} = S^{[3]}_{iu} + S^{[3]}_{iw}$. The square bracket superscripts 
denote the block of the parameter space. We chose symmetric forms of ${\bf B}^{(i;20)}$
and ${\bf B}^{(i;01)}$ (antisymmetric form could be constructed as well by taking the difference
of the parameters from block 2 and 3, respectively). $F_{iu}^{(N)}$ and $F_{iu,iw}^{(N)}$ 
stand for the electric field $\left[ {\bf F}({\bf r}_i)\right]_u$ and the electric field
gradient $\left[ {\BM \nabla} \otimes {\bf F}({\bf r}_i) \right]_{uw}$, respectively.

To determine the optimim set 
$
 {\bf S} = 
\begin{pmatrix}
{\bf S}^{[1]} &
{\bf S}^{[2]} &
{\bf S}^{[3]}
\end{pmatrix}^T
$, we define the loss function $Z$ 
of the three blocks of adjustable parameters
that is subject to the least\hyp{}squares minimization:
%
\begin{equation}\label{e:Z}
 Z[{\bf S}] = \sum_N^{N_{\rm max}} \left( \delta D^{(N)} - \delta \overline{D}^{(N)} \right)^2 \;.
\end{equation}
%
%where $N_{\rm max}$ is the total number of samples. 
%The linear form of Eq.~\eqref{e:final-model.General.Parameters} ensures that the Hessian 
%of $Z$
%is non\hyp{}singular and independent on the parameters. 
The Hessian of $Z$ computed with respect to the parameters is non\hyp{}singular 
and constant.
Therefore, the exact solution for the optimal parameters is given by a Newton equation
%
\begin{equation}\label{e:Newton}
 {\bf S} = -{\bf H}^{-1} \cdot {\bf g} \;,
\end{equation}
%
where ${\bf g}$ and ${\bf H}$ are the gradient vector and Hessian matrix, respectively.
%
%\begin{equation}\label{e:Newton.S}
% {\bf S} = 
%\begin{pmatrix}
%{\bf S}^{[1]} &
%{\bf S}^{[2]} &
%{\bf S}^{[3]}
%\end{pmatrix}^T
%\end{equation}
%
The explicit forms of the gradient and Hessian are given in the Appendix~\ref{a:blocks}.
Note that the dimensions of parameter space for the block 1, 2 and 3 are all
equal to $3M$.
%unperturbed molecule. This makes them perfect candidates for general effective fragment parameters.

\section{\label{s:3}Calculation Details}

The theory that is necessary to validate the density matrix susceptibility model from Section~\ref{s:2}
was implemented in our in\hyp{}house plugin to Psi4 quantum chemistry program.
A few simple molecules were selected as test systems, that were subject to 
perturbations due to uniform electric fields as well as collections of point charges
generating non\hyp{}uniform electric fields. 
%For finding a general unitary transformations,
%100 random point charge distributions were selected for which an exact density matrices
%were calculated.

In order to validate all the models discussed in this Contribution,
the electric field\hyp{}induced interaction energy, 
the dipole moment and the quadrupole moment were analysed in detail.
The electric field\hyp{}induced interaction energy at the HF level, $\delta E$,
was computed from the following exact formula:
%
\begin{multline}\label{e:dE}
 \delta E[{\bf F}] = 2\Tr{\left\{ {\bf D}^{(0)} \cdot {\bf V}[{\bf F}] \right\}}
                   + \Tr{\left\{ \left( {\bf D}^{(0)} + \delta {\bf D} \right) \cdot
                                \left( \delta {\bf J} + \delta {\bf K}\right) \right\}} \\
                   + \Tr{\left\{ \delta {\bf D} \cdot
                                \left( {\bf H}^{(0)} + 2 {\bf V}[{\bf F}] + {\bf f}^{(0)} \right) \right\}}
                   + E_{\rm nuc}({\bf F}) \;.
\end{multline}
%
In Eq.~\eqref{e:dE}, ${\bf f}^{(0)}$ and ${\bf H}^{(0)}$ are the Fock matrix and 
the one\hyp{}electron Hamiltonian matrix of an unperturbed HF ground state,
${\bf V}[{\bf F}]$ is the one\hyp{}electron external electrostatic potential matrix whereas $\delta {\bf J}$
and $\delta {\bf K}$ are electric field\hyp{}induced changes in the Coulomb and the exchange HF matrices due to 
the polarization of the electron density distribution, respectively. $E_{\rm nuc}({\bf F})$ denotes the interaction energy
due to the nuclei.
The electric field\hyp{}induced dipole and quadrupole moments were evaluated from the following expressions:
%
\begin{subequations}\label{e:dmult}
  \begin{align}
   \delta \upmu_{u}   &= 2\sum_{\alpha\beta} \delta D_{\alpha\beta} \tbraket{\alpha}{u}{\beta}  \;,\\
   \delta \Theta_{uw} &= 2\sum_{\alpha\beta} \delta D_{\alpha\beta} \tbraket{\alpha}{uw}{\beta} \;. 
  \end{align}
\end{subequations}
%
To compute the interaction energy and the induced multipole moments, either exact or approximated forms
of the difference density matrix from Eq.~\eqref{e:final-model.General}
were used for a particular electric field distribution. The statistical evaluation of
errors with respect to the benchmark (full HF calculations) were based on 
the following statistical quality descriptors:
%
\begin{enumerate}
 \item Root Mean Square Energy Error (RMSE)
   \begin{equation}
     {\rm RMSE} = \sum_N \sqrt{ \left( \delta E^{(N)} - \overline{\delta E^{(N)}}\right)^2 }
   \end{equation}
 \item Root Mean Square Dipole Error (RMSD)
   \begin{equation}
     {\rm RMSD} = \sum_N \sqrt{ \lvert \delta {\BM\upmu}^{(N)} - \overline{\delta {\BM\upmu}^{(N)}}\rvert^2 }
   \end{equation}
 \item Root Mean Square Quadrupole Error (RMSQ)
   \begin{equation}
     {\rm RMSQ} = \sum_N \sqrt{ \lvert \delta {\BM\Theta}^{(N)} - \overline{\delta {\BM\Theta}^{(N)}}\rvert^2 }
   \end{equation}
\end{enumerate}
%
The first describes the ability of the model to reproduce the exact energetics of the system, 
wherease the latter two contain crucial information on how well the deformation of the one\hyp{}electron
density is described.

Throughout the Work, the Hartree\hyp{}Fock level was assumed and the aug-cc-pVTZ basis set
was used. For all the calculations of the density matrix polarization susceptibility tensors
from Eqs.~\eqref{e:final-model.HF} and \eqref{e:susceptibility-B}, 
atomic basis was orthogonalized by using the L{\"o}wdin symmetric orthogonalization scheme.


\section{\label{s:4}Results and Discussion}

\subsection{Ab Initio Model of Densty Matrix Polarization}

We first test the accuracy of the density matrix susceptibility tensors that were derived
in Eq.~\eqref{e:susceptibility-B} from first principles. 
For this purpose we have considered a water molecule 
under weak uniform electric fields (Figure~XXX). 
3D difference one\hyp{}electron density isosurface plots for sample electric fields
with respect to the unperturbed density (${\bf F} = {\bf 0}$)
are shown in Figure~XXX. The statistical evaluation of errors for a set of
100 random electric fields is shown in Figure~XXX. The shape of the basins 
of decreased/increased density are comparable with the exact result
obtained from HF/6-311++G** calculations. 
%It can be seen that the theory
%describes the polarization of the electron density distribution qualitatively well:
However, the accuracy in the density matrix elements is on overall very low which 
manifests with drammatic errors in the electronic total energy of the system.
It is due to the approximations used to derive Eq.~\eqref{e:susceptibility-B},
which neglects quadrupole and higher\hyp{}order induced multipole moments.

\subsection{General Model of Densty Matrix Polarization}

%However, Eq.~\eqref{e:final-model} is an approximation because (i) it is based on the approximate
%relation in Eq.~\eqref{e:dmu-l-vector-mo-transform} and (ii) only total dipole moment (but not quadrupole
%and higher moments) is exactly
%reproduced by the current model. Due to the above reasons, the existence of ${\bf X}$ that




%The critical assessment of the proposed model is discussed in two stages: (i) first
%the capability of the model in capturing the polarization 
%It is instructive to analyze first the capability of the model in capturing the polarization
%process by studying the 3D electronic density distortions due to the electric field perturbation.
%For this purpose we consider a , i.e., without
%optimization of the particular density matrix elements with respect to benchmark. 

%For this purpose we studied few molecules
%in a uniform external electric field. Although the density matrix elements in a chosen AO basis
%can be inconsistent with other quantities (such as Fock matrix and so on) the polarization 


\section{\label{s:5}Summary and a few concluding remarks}
Test summary bla bla blaaaaa dfn fdj gf ads;jha g;ja gf; p.

\begin{acknowledgments}
This project is carried out under POLONEZ programme which has received funding from the European Union's
Horizon~2020 research and innovation programme under the Marie Skłodowska-Curie grant agreement 
No.~665778. This project is funded by National Science Centre, Poland 
(grant~no. 2016/23/P/ST4/01720) within the POLONEZ 3 fellowship.
\end{acknowledgments}

%
\appendix

\section{\label{a:orig-dep} Origin-Dependence of Induced Dipole Moment}

We need to be aware that multipole integrals depend on the origin with respect to which they are evaluated.
However, the induced dipole moments defined in Eq.~\eqref{e:mu-ind-distributed-general} 
have to be origin independent. 
%Let us check if this is the case.

Let us start with noticing that
%
\begin{equation}
 \left[ {\bf M} \right]_{i\alpha} ({\bf r}_0) 
 = \left[ {\bf M} \right]_{i\alpha} ({\bf 0}) - {\bf r}_0 \left[ {\bf C}^\dagger \right]_{i\alpha}  \;.
\end{equation}
%
Further we have
%
\begin{equation}
 \left[ {\bf K} \right]_{i\alpha} ({\bf r}_0) 
 = \left[ {\bf K} \right]_{i\alpha} ({\bf 0}) - {\bf r}_0 \left[ {\bf C}^\dagger {\bf D} \right]_{i\alpha} \;.
\end{equation}
%
But ${\bf C}^\dagger {\bf D}={\bf C}^\dagger {\bf C} {\bf C}^\dagger={\bf C}^\dagger$ in orthogonal
AO basis. This implies that
%
\begin{equation}
   \left[ {\bf L} \right]_{i\alpha} ({\bf r}_0) 
 = \left[ {\bf M} \right]_{i\alpha} ({\bf r}_0) - \left[ {\bf K} \right]_{i\alpha} ({\bf r}_0)
 = \left[ {\bf M} \right]_{i\alpha} ({\bf 0})   - \left[ {\bf K} \right]_{i\alpha} ({\bf 0})
 = \left[ {\bf L} \right]_{i\alpha} ({\bf 0}) \;.
\end{equation}
%
Therefore, it is proved that the polarization\hyp{}induced distributed dipole moments 
defined in Eq.~\eqref{e:mu-ind-distributed-general} 
are origin independent.
Thus, one can compute dipole integrals with respect to any origin and resulting
susceptibilities ${\bf B}_{\alpha\beta}^{(i)}$ from Eq.~\eqref{e:susceptibility-B} will be uniquely defined.

\section{\label{a:blocks} Explicit Formulae for Gradient and Hessian Blocks}

The gradient vector ${\bf g}$ and Hessian matrix ${\bf H}$ 
are build from blocks associated with the type of parameters, i.e.,
%
\begin{equation}\label{e:Newton.gH}
 {\bf g} = 
\begin{pmatrix}
{\bf g}^{[1]} \\ 
{\bf g}^{[2]} \\ 
{\bf g}^{[3]}
\end{pmatrix} ,\quad
 {\bf H} = 
\begin{pmatrix}
{\bf H}^{[11]} & {\bf H}^{[12]} & {\bf H}^{[13]} \\ 
{\bf H}^{[21]} & {\bf H}^{[22]} & {\bf H}^{[23]} \\ 
{\bf H}^{[31]} & {\bf H}^{[32]} & {\bf H}^{[33]} 
\end{pmatrix} \;.
\end{equation}
%
The gradient element of the $r$th block and Hessian element of the $rs$th block read
%
\begin{subequations}
 \begin{align}
  g^{[r ]}_{kv}    &\equiv \frac{\partial   Z}{\partial S_{kv}^{[r]}} 
     =-2\sum_N \overline{\delta D}^{(N)}
               \frac{\partial   \left[ \delta D^{(N)} \right]}{\partial S_{kv}^{[r]}} \;,\\
  H^{[rs]}_{kv,lw} &\equiv \frac{\partial^2 Z}{\partial S_{kv}^{[r]} \partial S_{lw}^{[s]}}  
     = 2\sum_N 
        \frac{\partial   \left[ \delta D^{(N)} \right]}{\partial S_{kv}^{[r]}}
        \frac{\partial   \left[ \delta D^{(N)} \right]}{\partial S_{lw}^{[s]}} \;.
 \end{align}
\end{subequations}
%
Note that the second derivatives of $\delta D^{(N)}$ vanish
due to the linearity with respect to the adjustable parameters.
Thus, the explicit formulae for the gradient are
%
\begin{subequations}
 \begin{align}
  g^{[1]}_{kv} &=-2\sum_N \overline{\delta D}^{(N)} F^{(N)}_{kv} \;,\\
  g^{[2]}_{kv} &=-4\sum_N \overline{\delta D}^{(N)} F^{(N)}_{kv} \widetilde{F}^{(N)}_k \;,\\
  g^{[3]}_{kv} &=-4\sum_N \overline{\delta D}^{(N)} \widetilde{F}^{(N)}_{kv} \;.
 \end{align}
\end{subequations}
%
The Hessian subsequently follows to be
%
\begin{subequations}
 \begin{align}
  H^{[11]}_{kv,lw} &= 2\sum_N F^{(N)}_{kv} F^{(N)}_{lw} \;,\\
  H^{[22]}_{kv,lw} &= 8\sum_N F^{(N)}_{kv} F^{(N)}_{lw} \widetilde{F}^{(N)}_k \widetilde{F}^{(N)}_l \;,\\
  H^{[33]}_{kv,lw} &= 8\sum_N \widetilde{F}^{(N)}_{kv} \widetilde{F}^{(N)}_{lw} \;,\\
  H^{[12]}_{kv,lw} &= 4\sum_N F^{(N)}_{kv} F^{(N)}_{lw} \widetilde{F}^{(N)}_l \;,\\
  H^{[13]}_{kv,lw} &= 4\sum_N F^{(N)}_{kv} \widetilde{F}^{(N)}_{lw} \;,\\
  H^{[23]}_{kv,lw} &= 8\sum_N F^{(N)}_{kv} \widetilde{F}^{(N)}_k \widetilde{F}^{(N)}_{lw} \;.
 \end{align}
\end{subequations}
%
In the above equations, the auxiliary quantities $\widetilde{F}^{(N)}_k$ 
and $\widetilde{F}^{(N)}_{kv}$ are defined as
%
\begin{subequations}
 \begin{align}
  \widetilde{F}^{(N)}_k    &\equiv \sum_w F^{(N)}_{kw} \;,\\
  \widetilde{F}^{(N)}_{kv} &\equiv \sum_w F^{(N)}_{kw,kv} \;.
 \end{align}
\end{subequations}
%
Note that due to the symmetry of the Hessian matrix, the blocks $21$, $31$ and $32$
are transposes of the blocks $12$, $13$ and $23$, respectively. The composite index
can be defined as $kv \equiv 3k+v$, for example.
\bibliography{references}

\end{document}
