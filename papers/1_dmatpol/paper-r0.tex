%% ****** Start of file aiptemplate.tex ****** %
%%
%%   This file is part of the files in the distribution of AIP substyles for REVTeX4.
%%   Version 4.1 of 9 October 2009.
%%
%
% This is a template for producing documents for use with 
% the REVTEX 4.1 document class and the AIP substyles.
% 
% Copy this file to another name and then work on that file.
% That way, you always have this original template file to use.

\documentclass[aip,graphicx]{revtex4-1}
%\documentclass[aip,reprint]{revtex4-1}

% hyphenation
\usepackage{hyphenat}
%\usepackage{lipsum}

% tables
%\usepackage{dcolumn}
%\usepackage{multirow}
%\usepackage{hhline}
%\usepackage{booktabs}
%\usepackage{longtable}

% mathematics
\usepackage{amsmath}
\usepackage{amsfonts}
\usepackage{amssymb}
\usepackage{amsbsy}
\usepackage{bm}
\usepackage{mathrsfs}
\usepackage{upgreek}
%\allowdisplaybreaks

% symbols
%\usepackage{gensymb}
%\usepackage{textcomp}

%---------------------------------------------------
% happy integral
\newcommand{\rint}[1]{\mbox{\Large $ \int\limits_{\mbox{\tiny  $#1$}}$}}
% SHORTCUTS
%\newcolumntype{,}{D{.}{,}{2}}
\newcommand{\citee}[1]{\ensuremath{\scriptsize^{\citenum{#1}}}}
\newcommand{\HRule}{\rule{\linewidth}{0.2mm}}
% Quantum notation
\newcommand{\bra}[1]{\ensuremath{\bigl\langle {#1} \bigl\lvert}}
\newcommand{\ket}[1]{\ensuremath{\bigr\rvert {#1} \bigr\rangle}}
\newcommand{\braket}[2]{\ensuremath{\bigl\langle {#1} \bigl\lvert {#2} \bigr\rangle}}
\newcommand{\tbraket}[3]{\ensuremath{\bigl\langle {#1} \bigl\lvert {#2} \bigl\lvert {#3} \bigr\rangle}}
% Math
\newcommand{\pd}{\ensuremath{\partial}}
\newcommand{\DR}{\ensuremath{{\rm d} {\bf r}}}
%\newcommand{\BM}[1]{\ensuremath{\mbox{\boldmath${#1}$}}}
\newcommand{\BM}[1]{\bm{#1}}
% Chemistry (formulas)
\newcommand{\ch}[2]{\ensuremath{\mathrm{#1}_{#2}}}
% Math 
\newcommand{\VEC}[1]{\ensuremath{\mathrm{\mathbf{#1}}}}
% vector nabla
\newcommand{\Nabla}{\ensuremath{ \BM{\nabla}}}
% derivative
\newcommand{\FDer}[3]{\ensuremath{
\bigg(
\frac{\partial #1}{\partial #2}
\bigg)_{#3}}}
% diagonal second derivative
\newcommand{\SDer}[3]{\ensuremath{
\biggl(
\frac{\partial^2 #1}{\partial #2^2}
\biggr)_{#3}}}
% off-diagonal second derivative
\newcommand{\SSDer}[4]{\ensuremath{
\biggl(
\frac{\partial^2 #1}{\partial #2 \partial #3}
\biggr)_{#4}}}
% derivatives without bound
% derivative
\newcommand{\fderiv}[2]{\ensuremath{
\frac{\partial #1}{\partial #2}}}
% diagonal second derivative
\newcommand{\sderiv}[2]{\ensuremath{
\frac{\partial^2 #1}{\partial #2^2}
}}
% off-diagonal second derivative
\newcommand{\sderivd}[3]{\ensuremath{
\frac{\partial^2 #1}{\partial #2 \partial #3}
}}
% derivatives for tables
\newcommand{\fderivm}[2]{\ensuremath{
{\partial #1}/{\partial #2}}}
% diagonal second derivative
\newcommand{\sderivm}[2]{\ensuremath{
{\partial^2 #1}/{\partial #2^2}
}}
% off-diagonal second derivative
\newcommand{\sderivdm}[3]{\ensuremath{
{\partial^2 #1}/{\partial #2 \partial #3}
}}
% ERIs and OEIs
\newcommand{\OEIc}[3]{\ensuremath{\left(#1 \lvert #2 \rvert #3 \right)}}
\newcommand{\ERIc}[4]{\ensuremath{\left(#1 #2 \vert #3 #4 \right)}}

% Partial density and potential
\newcommand{\PartPot}[4]{\ensuremath{\frac{#1 #2}{\lvert #3-#4 \rvert }}}

% trace operator
\DeclareMathOperator{\Tr}{Tr}

\draft % marks overfull lines with a black rule on the right

\begin{document}

\title{Density Matrix Susceptibility} %Title of paper

\author{Bartosz B{\l}asiak}
\email[]{blasiak.bartosz@gmail.com}
\homepage[]{https://www.polonez.pwr.edu.pl}
\thanks{Bulak}
\affiliation{Department of Physical and Quantum Chemistry, Faculty of Chemistry, Wroc{\l}aw University of Technology, Wybrze{\.z}e Wyspia{\'n}kiego 27, Wroc{\l}aw 50-370, Poland}

\date{\today}

\begin{abstract}
To be filled ...
\end{abstract}

\pacs{}% insert suggested PACS numbers in braces on next line

\maketitle %\maketitle must follow title, authors, abstract and \pacs

\section{\label{s:1}Introduction}

\section{\label{s:2}Theory}

The total electronic dipole moment of a closed-shell molecule can be expressed as
%
\begin{equation}\label{e:mu-total}
 {\BM{\upmu}}({\bf r}_0) = 
    2 \Tr{
        \left[
             {\bf D} {\mathbb{M}({\bf r}_0)}
        \right]} \;,
\end{equation}
%
where ${\bf D}$ is a one-particle density matrix and ${\mathbb{M}({\bf r}_0)}$ is certain representation
of the dipole moment operator defined with respect to origin at ${\bf r}_0$,
%
\begin{equation}\label{e:m}
 \left[ {\mathbb{M}({\bf r}_0)} \right]_{\alpha\beta} = \tbraket{\alpha}{{\bf r} - {\bf r}_0}{\beta} \;.
\end{equation}
%
In the above equations and throughout this Work, the basis set functions are labeled with Greek symbols.
%Note that ${\mathbb{M}}$ can be considered as a vector of $n\times n$ matrices with $n$ denoting the
%number of basis set functions.

In the Hartree\hyp{}Fock theory, ${\bf D}$ is approximated as
%
\begin{equation}
 {\bf D} = {\bf C}{\bf C}^\dagger \;,
\end{equation}
%
where ${\bf C}$ describes the occupied molecular orbitals as linear combinations
of atomic orbitals (AO's), referred here to as the LCAO-MO matrix.
When ${\bf D}$ is expressed in orthogonal basis, the density matrix is idempotent, i.e.,
%
\begin{equation}
 {\bf D}^2 = {\bf D} \;.
\end{equation}
%

%
\subsection{Molecule in External Electric Field}

Let us consider a case in which a molecule experiences external and generally non\hyp{}uniform electric field,
${\bf F}({\bf r})$. 
Then, if unchanging basis such as AO's is used, Eq.~\eqref{e:mu-total} can be recast as
%
\begin{equation}
 {\BM{\upmu}}({\bf r}_0) = 
     2 \Tr{ 
         \left[ 
              \left\{ {\bf D}^{(0)} + \delta {\bf D}[{\bf F}({\bf r})] \right\}  {\mathbb{M}({\bf r}_0)}
         \right] } \;,
\end{equation}
%
in which ${\bf D}^{(0)}$ is the unperturbed density matrix of the system whereas
$\delta {\bf D}$ is the associated change due to the external electric field distribution. 
In fact, $\delta {\bf D}$ is a complicated
but unknown function of ${\bf F}({\bf r})$ and basis set. 
%Moreover, it certainly depends on the choice of
%molecular orbitals because the unitary transformation . 
On the other hand it seems intuitive that $\delta {\bf D}$ should 
be directly related to the distortion of the electron density distribution, which is also described
by the polarization\hyp{}induced multipole moments. Then, the optimal $\delta {\bf D}$ minimizes
the total energy of the molecule under a constraint of reproducing the polarization\hyp{}induced 
distortion of the electron density distribution (or equivalently, all the multipole moments). 
Unfortunately, such a mathematical problem is very complicated due to its high\hyp{}order tensor nature
with respect to the unknown $\delta {\bf D}$ and no approximate analytical solutions have been proposed
up to date. Therefore, in this paper we adopt a somewhat different
approach that will require \emph{a priori} knowledge from training set of known cases.

\subsection{Polarization\hyp{}Induced Dipole Moment}

The first non\hyp{}vanishing induced moment is given by
%
\begin{equation} \label{e:dmu-exact}
 \delta {\BM{\upmu}}({\bf r}_0) = 
     2 \Tr{ 
         \left[ 
              \delta {\bf D} {\mathbb{M}({\bf r}_0)}
         \right] } \;.
\end{equation}
%
The change in the molecular orbitals can be parameterized as suggested by McWeeny\cite{McWeeny.RevModPhys.1960}
by defining the idempotency\hyp{}preserving variation
%
\begin{equation}
 \delta {\bf C} = {\BM\Delta} {\bf C} \;,
\end{equation}
%
where $\BM\Delta$ is a square non\hyp{}singular and non\hyp{}symmetric matrix.
McWeeny has shown that
%
\begin{equation} \label{e:dmat-change.exact}
 \delta {\bf D} = -{\bf D}^{(0)} + \left[ {\bf D}^{(0)} + {\bf v} \right]
                                   \left[ {\bf 1} + {\bf v}^\dagger{\bf v} \right]^{-1}
                                   \left[ {\bf D}^{(0)} + {\bf v}^\dagger \right] \;,
\end{equation}
%
where the auxiliary vector ${\bf v}$ is defined as
%
\begin{equation}
 {\bf v} \equiv \left[ {\bf 1} - {\bf D}^{(0)} \right] {\BM\Delta} {\bf D}^{(0)}  \;.
\end{equation}
%
Expanding Eq.~\eqref{e:dmat-change.exact} in a Taylor series around ${\bf v}={\bf 0}$ and
truncating it at linear terms with respect to ${\BM\Delta}$ gives
%
\begin{subequations} 
 \begin{align}
 \delta {\bf D} &\cong \left[ \bf{1} - {\bf D}^{(0)} \right] {\BM\Delta} {\bf D}^{(0)} + 
                        {\bf D}^{(0)} {\BM\Delta}^\dagger \left[ \bf{1} - {\bf D}^{(0)} \right]  
 \label{e:dD-1} \\  &= 
  \delta {\bf C}  {\bf C}^\dagger + {\bf C}\delta {\bf C}^\dagger
           - {\bf D}^{(0)} \delta {\bf C}  {\bf C}^\dagger - {\bf C}\delta {\bf C}^\dagger {\bf D}^{(0)} 
 \label{e:dD-2}
 \end{align}
\end{subequations}
%
The latter equality is true only if the basis set functions are orthogonal, i.e., ${{\bf C}^\dagger}{\bf C}={\bf 1}$,
which will be the case throughout the entire work, unless stated differently.
Substituting the result from Eq.~\eqref{e:dD-1}
into Eq.~\eqref{e:dmu-exact} leads to
%%
%\begin{equation} \label{e:delta-Taylor}
% \delta {\bf D} = \left[ {\bf v} + {\bf v}^\dagger \right]
%                + \left[ {\bf v}{\bf v}^\dagger - {\bf v}^\dagger{\bf v}\right] + \ldots 
%\end{equation}
%
\begin{equation} \label{e:dmu-4-exact.linear-approximation}
 \frac{1}{2} 
 \delta {\BM{\upmu}}
  \cong
   \Tr{ 
    \left[ 
         {\mathbb{M}} {\BM\Delta} {\bf D}^{(0)}  
    \right] }
%
  +\Tr{ 
    \left[ 
         {\mathbb{M}} {\bf D}^{(0)} {\BM\Delta}^{\dagger}
    \right] }
%
  -\Tr{ 
    \left[ 
         {\mathbb{M}} {\bf D}^{(0)} {\BM\Delta} {\bf D}^{(0)}
    \right] }
%
  -\Tr{ 
    \left[ 
         {\mathbb{M}} {\bf D}^{(0)} {\BM\Delta}^{\dagger} {\bf D}^{(0)}
    \right] } \;,
\end{equation}
%
It can be proven (see Appendix~\ref{a:orig-dep}) that such obtained
induced dipole moment is independent on the origin. Therefore,
in the above result and also in later course of the Paper the dipole origin has been omitted. 
%where the dipole origin was omitted for notational clarity.
If the AO basis is orthogonal, Eq.~\eqref{e:dmu-4-exact.linear-approximation} can be simplified by realising that
%
\begin{subequations}
 \begin{align}
  \Tr{ 
    \left[ 
         {\mathbb{M}} {\BM\Delta} {\bf D}^{(0)}  
    \right] }
%
  +\Tr{ 
    \left[ 
         {\mathbb{M}} {\bf D}^{(0)} {\BM\Delta}^{\dagger}
    \right] }
  &=
2 \Tr{ 
    \left[ 
         \widetilde{\mathbb{M}} \cdot \delta {\bf C}
   \right] }  \;,
%
\\
%
  \Tr{ 
    \left[ 
         {\mathbb{M}} {\bf D}^{(0)} {\BM\Delta} {\bf D}^{(0)}
    \right] }
%
 +\Tr{ 
    \left[ 
         {\mathbb{M}} {\bf D}^{(0)} {\BM\Delta}^{\dagger} {\bf D}^{(0)}
    \right] }
  &=
2 \Tr{ 
    \left[ 
         \widetilde{\mathbb{K}} \cdot \delta {\bf C}
   \right] } \;.
%
 \end{align}
\end{subequations}
%
Here we introduced the auxiliary tensors
%
\begin{subequations}
 \begin{align}
   \widetilde{\mathbb{M}}  &\equiv {\bf C}^\dagger {\mathbb{M}}     \;,           \\
   \widetilde{\mathbb{K}}  &\equiv {\bf C}^\dagger {\mathbb{M}} {\bf D}^{(0)} \;, \\
   \widetilde{\mathbb{L}}  &\equiv \widetilde{\mathbb{M}} - \widetilde{\mathbb{K}} \;.
 \end{align}
\end{subequations}
%
to simplify the notation. Hence, we have
%
\begin{equation} \label{e:dmu-l-vector}
  \frac{1}{4} 
 \delta {\BM{\upmu}} %({\bf r}_0) 
   =
   \Tr{ 
    \left[ 
         \widetilde{\mathbb{L}} \cdot \delta {\bf C}
    \right] }
   %
   \equiv \sum_{i}^{\rm occ} \sum_\alpha {\bf l}_{i\alpha} \delta C_{\alpha i} \;,
\end{equation}
%
where the Cartesian vector
%
\begin{equation}\label{e:l-vector}
 {\bf l}_{i\alpha} \equiv \left[ \widetilde{\mathbb{M}} - \widetilde{\mathbb{K}} \right]_{i\alpha} 
      = \left[  {\bf C}^\dagger \mathbb{M} \left[ \bf{1} - \bf{D}^{(0)} \right] \right]_{i\alpha}
\end{equation}
%
and $i$ runs over occupied unperturbed MO's.
Note that the intermediate quantity $\widetilde{\mathbb{L}}$ contains the projector onto
the unoccupied MO space at the absence of the external electric field. It means that the
perturbed solution is constructed from the unperturbed virtual orbitals as well.
Although Eq.~\eqref{e:dmu-l-vector}
is exact up to first order in ${\BM\Delta}$, it cannot be inverted at this moment to obtain $\delta {\bf C}$
due to the presence of trace operation. 

\subsection{Distributed Multipole Approach}

The above problem can be simplified if the trace operation in Eq.~\eqref{e:dmu-l-vector} is broken down into
two separate summations and then compared with the MO\hyp{}distributed partitioning 
of the polarization\hyp{}induced dipole moment. Formally,
%
\begin{equation} \label{e:mu-ind-distributed-general}
 \delta {\BM{\upmu}} = \sum_{i'}^{\rm occ} \delta {\BM{\upmu}}_{i'}  \;.
\end{equation}
%
The molecular orbitals in Eq.~\eqref{e:mu-ind-distributed-general}
are generally different from molecular orbitals in Eq.~\eqref{e:dmu-l-vector} because
any unitary transformation of MO's preserves the total dipole moment of the system. In other words,
Eq.~\eqref{e:dmu-l-vector} requires the following generalization
%
\begin{equation} \label{e:dmu-l-vector-X}
  \frac{1}{4} 
 \delta {\BM{\upmu}} %({\bf r}_0) 
   =
   \Tr{ 
    \left[ {\bf X}^\dagger
         \widetilde{\mathbb{L}}  \delta {\bf C}
          {\bf X}
    \right] } \;,
\end{equation}
%
in which ${\bf X}$ is, until now, an arbitrary unitary transformation of unperturbed molecular orbitals
%
\begin{equation}
 C_{\alpha i'} = \sum_i  C_{\alpha i} X_{ii'} \;,
\end{equation}
%
and is, in general, dependent on the external electric field.
Unfortunately, such an approach is greatly limited because now the target quantity $\delta {\bf C}$
is involved in summations that are second order in ${\bf X}$. However,
one can consider an opposite but much simpler problem when we look for such unitary transformation
that will result in the MO\hyp{}distributed dipole moments equivalent to 
$\sum_\alpha {\bf l}_{i\alpha} \delta C_{\alpha i}$ for $i$ running over \emph{known} set of unperturbed orbitals.
It means that we require that
%
\begin{equation} \label{e:dmu-l-vector-mo-transform}
 \frac{1}{4} \delta {\BM{\upmu}}_{i'}
   \cong
   \sum_\alpha {\bf l}_{i\alpha} \delta C_{\alpha i} \;.
\end{equation}
%
%which is to be understood that ${\BM{\upmu}}^{(i)}$ refers to the part of the total induced dipole moment
%that is compatible with the contribution to $\delta {\bf C}$ constructed from the known set of MO's.
%(which is different from ${\BM{\upmu}}_i$ which 
$\delta {\BM{\upmu}}_{i'}$ can be found from the coupled\hyp{}perturbed Hartree\hyp{}Fock theory,
according to which the molecular polarizability is given as 
%
\begin{equation} \label{e:pol-cphf}
 {\BM{\alpha}} = \Tr{{\mathbb{A}}} = \sum_i^{\rm occ} {\BM{\alpha}}_i  \;,
\end{equation}
%
where
%
\begin{equation} \label{e:pol-cphf-A}
 \left[{\mathbb{A}}\right]_{ij} = 2 \sum_r^{\rm vir} {\mathbb X}_{ir} \otimes {\mathbb{M}}_{rj}
\end{equation}
%
and ${\BM{\alpha}}_i$ is the distributed polarizability associated with the $i$th occupied MO.
In Eq.~\eqref{e:pol-cphf} the trace is performed along the occupied MO axes of ${\mathbb{A}}$
whereas the tensor outer product ($\otimes$) in Eq.~\eqref{e:pol-cphf-A} refers to the Cartesian components of
the perturbation operator $\hat{x}$ and dipole moment operator, projected onto the 
occupied and virtual MO subspaces. Note also, that ${\mathbb{A}}$ is a square matrix with
dimension $(N_{\rm occ} \times N_{\rm occ})$ of $(3 \times 3)$ matrices.

Now, the induced dipole moment associated with the $i$th occupied
MO is given by
%
\begin{equation} \label{e:dmu-dpol}
 \delta {\BM{\upmu}}_i = {\BM{\alpha}}_i \cdot {\bf F}({\bf r}_i) \;.
\end{equation}
%
In Eq.~\eqref{e:dmu-dpol}, ${\bf F}({\bf r}_i)$ is the electric field evaluated at the $i$th MO centroid of charge
${\bf r}_i = \tbraket{i}{\hat{\bf r}}{i}$.
It becomes apparent that
%
\begin{equation} \label{e:dmu-dpol-expl}
 \delta {\BM{\upmu}}_{i'} = \left[ {\bf X}^\dagger {\mathbb A} {\bf X} \right]_{i'i'} \cdot {\bf F}({\bf r}_{i'}) \;.
\end{equation}
%
We can now insert the above result into Eq.~\eqref{e:dmu-l-vector-mo-transform} and, after dropping off the primed indices
and recasting the right hand side in a matrix form,
we get
%
\begin{equation} \label{e:dmu-l-vector-mo-transform-X}
 \frac{1}{4} \sum_{jk}^{\rm occ} X_{ji} \left[ {\bf A}_{jk} \cdot {\bf F}({\bf r}_i) \right] X_{ki}
   =
   \delta {\bf c}_i^T \cdot {\bf L}_i \;.
\end{equation}
%
Note that now ${\bf L}_i$ is a matrix of size $(n \times 3)$ and $\delta{\bf c}_i$ is just the $i$th column of the
change in the LCAO-MO matrix due to the polarization process.
%
%The above equation can be re-written in a matrix form as
%
%\begin{equation} 
% \left[ {\mathbb L} \right]_{\alpha \zeta \beta} = L_{\beta\alpha}^{(\zeta)}
%\end{equation}
%%
%of size $m \times 3 \times m$.
%Note that, in this notation, ${\BM{\upmu}}_i^T$ and $\overline{\BM\Delta}_i^T$ are row vectors
%whereas $\overline{\BM\Delta}_i^p$ is a column vector. 
The solution for $\delta {\bf C}$, given that ${\bf X}$ is known, is
%
\begin{equation} \label{e:mu-ind.matrix.linear.solution}
  \delta {\bf c}_i^T = \frac{1}{4}
                    \sum_{jk}^{\rm occ} X_{ji} X_{ki}
                    \left[ {\bf A}_{jk} \cdot {\bf F}({\bf r}_i) \right] \cdot
                    \left[ {\bf L}_i  \right]^{-1}_{\rm Left} \;,
\end{equation}
%
where $\left[ {\bf L}_i  \right]^{-1}_{\rm Left}$ is a left inverse
of ${\bf L}_i$ matrix 
%
\begin{equation} 
      \left[ {\bf L}_i  \right]^{-1}_{\rm Left}   \equiv
       \left[ {\bf L}_i^T {\bf L}_i \right]^{-1} {\bf L}_i^T 
\end{equation}
%
and has size $(3\times n)$. Note that the \emph{right inverse} of ${\bf L}_i$
does not exist because ${\bf L}_i {\bf L}_i^T$ is singular.
%Clearly, in the above approximation, ${\bf X}$ depends on the choice
%of unperturbed orbitals.


\subsection{Density Matrix Change}

Now we can obtain the final expression for the density matrix change 
due to external electric field that reproduces the 
induced dipole moment of the molecule, provided ${\bf X}$ is known. 
The change in the LCAO\hyp{}MO coefficient is parametereized as
%
\begin{equation} \label{e:dC-dmatpol}
 \delta C_{\alpha i} = {\bf b}_{\alpha i}[{\bf X}] \cdot {\bf F}({\bf r}_i)  \;,
\end{equation}
%
where the susceptibility tensor is defined as
%
\begin{equation} \label{e:susceptibility-b}
  b_{\alpha i;w}[{\bf X}] = \frac{1}{4} \sum_{jk}^{\rm occ} \sum_u^{x,y,z} X_{ji} X_{ki} A_{jk;uw} 
   \left[ \left[ {\bf L}_i  \right]^{-1}_{\rm Left} \right]_{u;\alpha}  
\end{equation}
%
for $w=x,y,z$. The result from Eq.~\eqref{e:dC-dmatpol} can be now inserted into 
the expression for $\delta {\bf D}$ from Eq.~\eqref{e:dD-2} to finally give
%
\begin{equation}\label{e:final-model}
 \delta D_{\alpha\beta} \approx \sum_i^{\rm occ} {\bf B}^{(i)}_{\alpha\beta} \cdot {\bf F}({\bf r}_i)  \;,
\end{equation}
%
where the \emph{density matrix polarization susceptibility tensor} is
%
\begin{equation}  \label{e:susceptibility-B}
 {\bf B}^{(i)}_{\alpha\beta} = \left\{
%
                               C_{\alpha i} {\bf b}_{\beta i} + C_{\beta i} {\bf b}_{\alpha i}
                               - \sum_\gamma 
                                 \left( 
               D_{\alpha\gamma} C_{\beta i} + D_{\beta\gamma} C_{\alpha i}
                                 \right)
                                           {\bf b}_{\gamma i}
%
\right\}  \;.
\end{equation}
%
Hence, the 3D vector ${\bf B}^{(i)}_{\alpha\beta}$ describes the polarization affinity
of the density matrix element associated with $\alpha$th and $\beta$th orthogonal AO's.
Transforming these vectors to a non\hyp{}orthogonal AO basis is straightforward.

It is emphasized here that all elements of ${\bf B}_{\alpha\beta}^{(i)}$ except of ${\bf X}$
are unique properties of electronically unperturbed molecule. 
However, Eq.~\eqref{e:final-model} is an approximation because (i) it is based on the approximate
relation in Eq.~\eqref{e:dmu-l-vector-mo-transform} and (ii) only total dipole moment (but not quadrupole
and higher moments) is exactly
reproduced by the current model. Due to the above reasons, the existence of ${\bf X}$ that
reproduces the exact perturbed density matrix with vanishing error is not guaranteed. 
Moreover, as it was pointed out before, ${\bf X}$ almost certainly depends on the external electric field. 
Next section adresses the problem of finding the optimal ${\bf X}$ that is treated as an adjustable
matrix.

%unperturbed molecule. This makes them perfect candidates for general effective fragment parameters.

\subsection{Determination of Optimal Unitary Transformation}

In this Work the major task is then brought to examining if it is possible to find such ${\bf X}$ that optimizes
the perturbed density matrix with the minimal error. If it would be the case, the next important question
would be how much the effective unitary transformation depends on the external electric field.

We start with recasting Eq.~\eqref{e:final-model} in a form that is simpler for further analysis, mainly
%
\begin{equation}
 \delta D_{\alpha\beta}^{(N)} \approx
  \sum_{ijk}^{\rm occ} X_{ji} X_{ki} S_{ijk}^{\alpha\beta(N)}  \;,
\end{equation}
%
where 
%
\begin{equation}
  S_{ijk}^{\alpha\beta(N)} \equiv 
   \frac{1}{4} \sum_{uw}^{x,y,z} A_{jk;uw} 
   \left\{
     C_{\alpha i} G_{i\beta ;u} + C_{\beta i} G_{i\alpha ;u} 
   - \sum_\gamma \left(
        D_{\alpha\gamma} C_{\beta i} + D_{\beta\gamma} C_{\alpha i} 
    \right) G_{i\gamma ;u}
   \right\}
   F_w^{(N)}({\bf r}_i)  \;,
\end{equation}
%
$N$ denotes the sample taken for analysis and
%
\begin{equation}
 G_{i\alpha ;u} \equiv  \left[ \left[ {\bf L}_i  \right]^{-1}_{\rm Left} \right]_{u;\alpha}  \;.
\end{equation}
%
%for notational clarity. 
Now, let us assume that we have gathered $N$ samples with $N$ known exact estimates
of density matrix (each for different distribution of electric field) denoted as $\delta D_{\alpha\beta}^{(N),{\rm ref}}$. 
We define the objective, positive definite
error function
%
\begin{equation} \label{e:Z-func}
 Z[{\bf X}] = \sum_N \sum_{\alpha\beta} 
    \left( 
      \delta D_{\alpha\beta}^{(N)} - \delta D_{\alpha\beta}^{(N),{\rm ref}}
    \right)^2
\end{equation}
%
that is subject to minimization under unitary constraints for $\bf X$.
Writing $Z$ function explicitly we can see that it is fourth\hyp{}order with respect to
the unitary transformation
%
\begin{equation} \label{e:Z-func-explicit}
 Z[{\bf X}] = \sum_{ijklmn} X_{ji} X_{ki} X_{ml} X_{nl} R_{ijklmn}
    + \sum_{ijk} X_{ji} X_{ki} P_{ijk} 
    + Z_0
\end{equation}
%
where 
%
\begin{subequations}
 \begin{align}
  R_{ijklmn} &=  \sum_N \sum_{\alpha\beta} S_{ijk}^{\alpha\beta(N)} S_{lmn}^{\alpha\beta(N)} \\
  P_{ijk}    &=-2\sum_N \sum_{\alpha\beta} \delta D_{\alpha\beta}^{(N),{\rm ref}} S_{ijk}^{\alpha\beta(N)} \\
  Z_0        &=  \sum_N \sum_{\alpha\beta} \left[ \delta D_{\alpha\beta}^{(N),{\rm ref}} \right]^2
 \end{align}
\end{subequations}
%
We developed the iterative algorithm to find the global optimum of $Z$ and the Reader 
is referred to the Appendix~\ref{a:X-mat} for more details. 

\section{\label{s:3}Calculation Details}

The theory that is necessary to validate the density matrix susceptibility model from Section~\ref{s:2}
was implemented in our in\hyp{}house plugin to Psi4 quantum chemistry program.
A few simple molecules were selected as test systems, that were subject to 
perturbations due to uniform electric fields as well as a collection of point charges
generating a non\hyp{}uniform electric fields. For finding a general unitary transformations,
100 random point charge distributions were selected for which an exact density matrices
were calculated.

Throughout the Work, the Hartree\hyp{}Fock level was assumed and the aug-cc-pVTZ basis set
was used. For all the calculations of the density matrix polarization susceptibility tensors, 
atomic basis was orthogonalized by using the L{\"o}wdin symmetric orthogonalization scheme.


\section{\label{s:4}Results and Discussion}


%The critical assessment of the proposed model is discussed in two stages: (i) first
%the capability of the model in capturing the polarization 
It is instructive to analyze first the capability of the model in capturing the polarization
process by studying the 3D electronic density distortions due to the electric field perturbation.
For this purpose we consider a special case of the model in which ${\bf X}={\bf 1}$, i.e., without
optimization of the particular density matrix elements with respect to benchmark. In such a case,
density matrix polarization susceptibility tensors are implicitly independent on the electric field
distribution and can be computed for an isolated molecule.

For this purpose we studied few molecules
in a uniform external electric field. Although the density matrix elements in a chosen AO basis
can be inconsistent with other quantities (such as Fock matrix and so on) the polarization 


\section{\label{s:5}Summary and a few concluding remarks}
Test summary bla bla blaaaaa dfn fdj gf ads;jha g;ja gf; p.

\begin{acknowledgments}
This project is carried out under POLONEZ programme which has received funding from the European Union's
Horizon~2020 research and innovation programme under the Marie Skłodowska-Curie grant agreement 
No.~665778. This project is funded by National Science Centre, Poland 
(grant~no. 2016/23/P/ST4/01720) within the POLONEZ 3 fellowship.
\end{acknowledgments}

%
\appendix

\section{\label{a:orig-dep} Origin-Dependence of Induced Dipole Moment}

We need to be aware that multipole integrals depend on the origin with respect to which they are evaluated.
However, the induced dipole moments defined in Eq.~\eqref{e:mu-ind-distributed-general} 
have to be origin independent. 
%Let us check if this is the case.

Let us start with noticing that
%
\begin{equation}
 \left[ {\bf M} \right]_{i\alpha} ({\bf r}_0) 
 = \left[ {\bf M} \right]_{i\alpha} ({\bf 0}) - {\bf r}_0 \left[ {\bf C}^\dagger \right]_{i\alpha}  \;.
\end{equation}
%
Further we have
%
\begin{equation}
 \left[ {\bf K} \right]_{i\alpha} ({\bf r}_0) 
 = \left[ {\bf K} \right]_{i\alpha} ({\bf 0}) - {\bf r}_0 \left[ {\bf C}^\dagger {\bf D} \right]_{i\alpha} \;.
\end{equation}
%
But ${\bf C}^\dagger {\bf D}={\bf C}^\dagger {\bf C} {\bf C}^\dagger={\bf C}^\dagger$ in orthogonal
AO basis. This implies that
%
\begin{equation}
   \left[ {\bf L} \right]_{i\alpha} ({\bf r}_0) 
 = \left[ {\bf M} \right]_{i\alpha} ({\bf r}_0) - \left[ {\bf K} \right]_{i\alpha} ({\bf r}_0)
 = \left[ {\bf M} \right]_{i\alpha} ({\bf 0})   - \left[ {\bf K} \right]_{i\alpha} ({\bf 0})
 = \left[ {\bf L} \right]_{i\alpha} ({\bf 0}) \;.
\end{equation}
%
Therefore, it is proved that the polarization\hyp{}induced distributed dipole moments 
defined in Eq.~\eqref{e:mu-ind-distributed-general} 
are origin independent.
Thus, one can compute dipole integrals with respect to any origin and resulting
susceptibilities ${\bf B}_{\alpha\beta}^{(i)}$ from Eq.~\eqref{e:susceptibility-B} will be uniquely defined.

\section{\label{a:X-mat}Algorithm for Optimal Unitary Transformation}

The error function from Eq.~\eqref{e:Z-func-explicit} is optimized by using the generalized Jacobi iteration algorithm
which is an extension of the Edmiston\hyp{}Reudenberg optimization scheme to tackle high\hyp{}order
complicated tensor equations. 

Here we discuss a general algorithm that minimizes or maximizes the following function
%
\begin{equation} \label{e:a.Z}
 Z[{\bf X}] = \sum_{ijklmn} X_{ki} X_{lj} X_{mi} X_{nj} R_{ijklmn}
             +\sum_{ijk}    X_{ji} X_{ki} P_{ijk}
\end{equation}
%
under unitary constraint for ${\bf X}$. ${\bf P}^{(3)}$ and ${\bf R}^{(6)}$ are the real, arbitrary three\hyp{} and
six\hyp{}rank tensors of size $N^3$ and $N^6$, respectively. Note that $Z_0$ term 
in Eq.~\eqref{e:Z-func-explicit} is constant
and therefore we omit this term in the present section.

Optimization of $ {\bf X} $ is factorized into a sequence of 2-dimensional Jacobi rotations with one real parameter
%
\begin{equation}
 {\bf X}^{\rm New} = {\bf X}^{\rm Old} \cdot {\bf U}^{IJ}(\gamma)
\end{equation}
%
where 
%
\begin{equation} \label{e:a.U}
 U_{ij}^{IJ}(\gamma) = \delta_{ij}
                       \left( 1-\delta_{iI} \right)
                       \left( 1-\delta_{jJ} \right) 
                       +
                       \cos(\gamma) \delta_{ij} \left( \delta_{iI} + \delta_{jJ} \right)
                       + 
                       \sin(\gamma) \left( 1-\delta_{ij} \right) 
                       \left( \delta_{iI} \delta_{jJ} - \delta_{iJ} \delta_{jI}\right)
\end{equation}
%
is the Jacobi transformation matrix constructed for the appropriately chosen $ I$th and $ J$th element ($I\neq J$)
from the entire $ N$-dimensional set. Each step is performed by finding the optimal pair
$(I,J)$ (out of all $N(N-1)/2$ pairs) for which optimal $\gamma*$ gives the largest decrease (increase) of $Z$ 
in the case of minimization (maximization). 
For the sake of algirithmic simplicity, 
every iteration after $ {\bf U}^{IJ}(\gamma) $ has been formed, $ {\bf X}^{\rm Old} $ 
is for a while assumed to be an identity matrix and the $ {\bf R} $ as well as $ {\bf P} $ tensors 
are transformed according to the following formulae
%
\begin{subequations}\label{e:a.rp-transformations}
 \begin{align}
 R_{ijklmn} &\rightarrow \sum_{k'l'm'n'} R_{ijk'l'm'n'} X_{k'k} X_{l'l} X_{m'm} X_{n'n} \;, \\
 P_{ijk}    &\rightarrow \sum_{j'k'}     P_{ijk} X_{j'j} X_{k'k}  \;.
 \end{align}
\end{subequations}
%
The full transformation matrix is accumulated in the memory buffer until convergence.

To find the optimal $\gamma^*$ for a chosen $(I,J)$ pair
all the roots of the first derivative of $Z$ with respect to $\gamma$
need to be found first, i.e.,
%
\begin{equation} \label{e:a.Z-grad}
 \frac{\partial Z[{\bf U}^{IJ}(\gamma)]}{\partial \gamma} = 0 \;.
\end{equation}
%
From expression for ${\bf U}^{IJ}$ in Eq.~\eqref{e:a.U} it is clear that Eq.~\eqref{e:a.Z-grad}
adopts the truncated Fourier series
%
\begin{equation} \label{e:a.Z-grad-Fourier}
 \frac{\partial Z[{\bf U}^{IJ}(\gamma)]}{\partial \gamma} = 
  a_0 + \sum_{q=1}^4 \left\{ a_q\cos(q\gamma) + b_q\sin(q\gamma) \right\} = 0\;.
\end{equation}
%
In general, if $Z$ function subject to optimize contains $Y$ as maximal power of ${\bf X}$
then the Fourier series will have $2Y+1$ coefficients and maximally $2Y$ roots. Therefore,
in the case of $Z$ in Eq.~\eqref{e:a.Z} there will be eight roots, from which only one
is to be chosen.

Finding all the Fourier coefficients is straightforward though tedious if attempted analytically. 
In the Supporting Information the analytical formulae for the Fourier coefficients are given. 
Once all the Fourier coefficients are found, Eq.~\eqref{e:a.Z-grad-Fourier} can be solved
by utilizing the Boyd's method.\cite{Boyd.JEngMath.2006} 
In our case, the roots are given by
%
\begin{equation}
 \gamma_n = \Re\left[-i\ln{\varepsilon_n}\right] \;,
\end{equation}
%
where $\varepsilon_n$ are the complex eigenvalues of the Frobenius\hyp{}Jacobi matrix
%
\begin{equation}
\begin{pmatrix}
0 & 1 & 0 & 0 & 0 & 0 & 0 & 0\\
0 & 0 & 1 & 0 & 0 & 0 & 0 & 0\\
0 & 0 & 0 & 1 & 0 & 0 & 0 & 0\\
0 & 0 & 0 & 0 & 1 & 0 & 0 & 0\\
0 & 0 & 0 & 0 & 0 & 1 & 0 & 0\\
0 & 0 & 0 & 0 & 0 & 0 & 1 & 0\\
-\frac{a_4+ib_4}{a_4-ib_4} &
-\frac{a_3+ib_3}{a_4-ib_4} &
-\frac{a_2+ib_2}{a_4-ib_4} &
-\frac{a_1+ib_1}{a_4-ib_4} & 
-\frac{2a_0}{a_4-ib_4} &
-\frac{a_1-ib_1}{a_4-ib_4} &
-\frac{a_2-ib_2}{a_4-ib_4} &
-\frac{a_3-ib_3}{a_4-ib_4} &
\end{pmatrix}  \;.
\end{equation}
%
The roots can also be determined numerically. 
It is anticipated here that the above described method is not limited to Eq.~\eqref{e:a.Z} but can be directly applied 
to design unitary optimizations for more complicated tensor equations. In fact, the whole problem
reduces to finding the roots of the Fourier series.
It is also worth mentioning that the orbital localization method of Edmiston and Reudenberg
is the special case of the above algorighm. In their case the most of the Fourier coefficients
vanish leaving only those for $\cos(4\gamma)$. The according $Z$ function reaches 
the same maxima (minima) for angles that are constant through the iterative process,
in contrast to the general case when they need to be updated every iteration,
after the transformations similar to those in Eqs.~\eqref{e:a.rp-transformations}.

\bibliography{references}

\end{document}
